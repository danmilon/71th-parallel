\chapter{Day 5 -- Aarhus - Hirtshals (DK) - Lysebotn (NO) - 540 km}

Το φως που εμπαινε απο το μισοκλειστο παραθυρο δεν αφηνε περιθωρια διαφωνιας: σηκω.
Δεν θα διαφωνουσα. Το σωμα μου ειχε πλεον αρχισει να συνηθιζει στις καταπονησεις του ταξιδιου και οι 4 ωρες υπνου μου ηταν αρκετες.
Σημερα εφτανα στο ακρωτηρι Hirtshals της Δανιας και μετα... ολα τα ενδεχομενα ανοιχτα. Ολα αγνωστα, terra incognita.
Πως ειναι να πλαθεις μια εικονα στο μυαλο σου για εκεινη, τη μια Μοναδικη Στιγμη που ονειρευεσαι; Για την ωρα που περνας τη πυλη με τη ροζαλια, για τη στιγμη που γυρνας το κλειδι στη μιζα της πρωτης σου Μηχανης, για τη στιγμη που Εκεινη ερχεται κοντα και την παιρνεις στην αγκαλια σου, για την στιγμη που αντικρυζεις τα ματια του πρωτου σου παιδιου, για τη στιγμη που κανεις το ονειρο επιτελους πραξη;

Και να... Να που τωρα ημουν στο ηλιολουστο μικρο δωματιακι ενος ξενωνα στη Δανια, σε αποσταση αναπνοης απο αυτο που τοσα χρονια στριφογυριζα στο μυαλο μου. Ενα τετοιο ταξιδι. Απο τα Μεγαλα. Τα καλα.

\photo{136.jpg}

``Εχουμε μεγαλη μερα σημερα'' ειπα και χαμογελασα. Μαζεψα τα πραγματα και κατεβηκα στο σαλονι για το πρωινο. 
Ο ξενωνας \href{http://www.citysleep-in.dk/}{City Sleep-in} ηταν πραγματικα εξαιρετικος. Πολυ νεανικος, πολυχρωμος, καθαρος, με αυτη την αψογη απερριτη βορεια αισθητικη που συνδιαζε το συγχρονο με το παραδοσιακο, πετρα, ξυλο και με μια πολυ ομορφη αυλη για τους ενοικους. 
Ακομα πιο εντυπωσιακο ομως ηταν το ποσο οικονομικος ηταν για τα μετρα της Δανιας: μολις 25 ευρω τη βραδια! 
Αν τυχει και βρεθειτε στα περιξ σας τον συστηνω ανεπιφυλακτα!

\photo{137.jpg}
\photo{138.jpg}

Ο μπουφες ηταν ολα τα λεφτα! Πρωινο με βιολογικα προιοντα, ανοιχτα χρωματα, φυσικα υλικα και μοντερνες πινελιες απο ενα ...κανονικοτατο φαναρι της τροχαιας! Λατρευω τους ξενωνες! Τρελα μερη με εξισου τρελους ανθρωπους.

\photo{139.jpg}

Λατρευω ομως και ενα καλο πρωινο, οποτε οταν βρισκομαι σε τετοια μερη δεν χανω ευκαιρια να το απολαυσω! 
Η ωρα ηταν 7 παρα και το πλοιο εφευγε σε 3 ωρες αλλα δεν αγχωνομουν. Εμεναν 130 χιλιομετρα μεχρι το Hirtshals και με ευθειες της εθνικης οδου θα ημουν εκει πολυ ευκολα. 
Ποσο μου αρεσε αυτη η στιγμη! Εξω ο ηλιος ελαμπε ηδη πανω σε ενα καταγαλανο ουρανο, βρισκομουν σε ενα κουκλιστικο περιβαλλον που μεχρι προτινος εβλεπα μονο σε καταλογους του ΙΚΕΑ και στο τραπεζι ειχα οτι θα μπορουσα να θελω για ενα βασιλικο πρωινο -γιαουρτι φρουτοχυμο (πολυ διαδεδομενο ποτο στη Δανια), φρεσκα φρουτα, καφε, ομελετα, σαντουιτς...

Οσο καθομουν εκει και χαζευα γυρω τις παρεες των ενοικων να τιτιβιζουν αγουροξυπνημενοι, σκεφτομουν για αλλη μια φορα τις διαφορες στη ζωη των ανθρωπων σε αυτες τις χωρες με τις ζωες μας στην Ελλαδα -η τουλαχιστον με τη δικη μου τη ζωη. 
Αν ηταν μια λεξη που μου ερχοταν στο μυαλο ηταν Κουλτουρα. Λεμε οτι οι Ελληνες εχουμε πολιτισμο, κουλτουρα κλπ, αλλα που ηταν στην καθημερινοτητα τελικα; 
Εδω η κουλτουρα φαινοταν και στα πιο απλα πραγματα: οριστε ενας τυπικος Δανεζικος ξενωνας που προσεφερε μονο φρεσκα βιολογικα προιοντα για πρωινο, σε ενα χωρο που θα ζηλευε και το πιο κυριλε εστιατοριο της Αθηνας, με προσοχη ακομα και στη παρουσιαση και μερακι στη διακοσμηση. 
Οι ανθρωποι γυρω μπορει να ηταν διερχομενοι ταξιδιωτες αλλα δεν ακουγες φωνες, φασαρια και κακο χαμο: ηξεραν να σεβονται ο ενας το διπλανο του και να χαιρονται χωρις να ενοχλουν.
Οταν δε τελειωναν το πρωινο τους, μαζευαν τους δισκους τους, καθαριζαν το τραπεζι τους και πηγαιναν τα υπολοιπα χαρτια, πλαστικα κλπ στους αντιστοιχους καδους ανακυκλωσης που βρισκονταν στο χωρο. Πολιτισμος. 
Εμεις εδω στην Ελλαδα ομως σε αυτα κολλαμε μια ταμπελα ``ελα μωρε οι ξενερωτοι/δεν ξερουν να χαιρονται τη ζωη τους'' και ο μηνας εχει 9... Και ας αφηνουμε στη παραλια φραπεδες, τσιγαρα, αντηλιακα, χαρτια και κουτακια μπυρας... Και ας γινεται πανικος σε καθε μερος που παμε χωρις να μας ενδιαφερει τι κανει ο διπλανος μας. Ειμαστε οι καλυτεροι ετσι και αλλιως...

Δεν θα ψυχοπλακωνομουν ομως! Στο χερι του καθενα μας ειναι να κανει οτι μπορει ωστε να δωσει στις νεες γενιες που ερχονται τα εφοδια να γινουν εφαμιλλοι και καλυτεροι απο αυτους τους λαους εδω πανω. Γιατι τα υλικα υπαρχουν. Απλα πρεπει να τα φροντισουμε και να τα φτιαξουμε με σωστο τροπο. 

Τελειωσα το πρωινο και βγηκα εξω να φορτωσω τη μηχανη με ενα μεγαλο χαμογελο. 
Το Århus με αποχαιρετουσε πανεμορφο. Η γραφικη παλια πολη, το μεγαλο λιμανι, η πολυ ομορφη αρχιτεκτονικη των σπιτιων -νεων και παλιων- κατα μηκος της προκυμαιας...

\photo{140.jpg}
\photo{141.jpg}

Τι ομορφη μερα! Στο ταξιδι μου ειχα καποια σημεια ``κομβικα'' οπου ευχομουν να βρω καλες συνθηκες. Ε λοιπον, αυτο ηταν ενα τετοιο σημειο και δεν θα μπορουσα να ζητησω καλυτερο καιρο ουτε στα ονειρα μου...
Εθνικη οδος-ποιημα!

\photo{142.jpg}

Παρα τη μικρη καταιγιδα που βρηκα στη πορεια (εδω δεν ηταν νοτια Ευρωπη -ο καιρος σε αυτα τα μερη ηταν εντελως απροβλεπτος και παρανοικος!) τιποτα δεν μπορουσε να μετριασει τη χαρα μου! Η λιακαδα εμφανιστηκε ξανα και με παρεα μια ακομα μηχανη συντομα εφτασα στο λιμανι του Hirtshals με τη Βορεια Θαλασσα να ανοιγεται μπροστα μου... Η πυλη για εναν αλλο εντελως εξωπραγματικο κοσμο...

\photo{143.jpg}

Το Fjord Cat, ενα υπερσυγχρονο catamaran τεραστιων διαστασεων, ηταν εκει και με περιμενε για να με περασει στα μερη που τοσα χρονια σκεφτομασταν σαν ενα μακρυνο ονειρο... Δεν μπορουσα να πιστεψω ακομα οτι τωρα ημουν εκει πλεον!

\photo{144.jpg}

Μπηκα στο αμπαρι και εδεσα τη μηχανη γερα με τους θηριωδεις ιμαντες του πλοιου, ενω γυρω μου περνουσε πολυς κοσμος που πηγαινε προς το σαλονι. Ανεβηκα κι εγω και το θεαμα δεν θυμιζε καθολου τα πλοια που ξερουμε: ενας πολυ μεγαλος hi-tech χωρος σε δυο οροφους (!) με δερματινες πολυθρονες και τραπεζια παντου και στο κεντρο στο ισογειο μαγαζια που σερβιραν καφε και φαγητο.

\photo{145.jpg}

Επιασα μια θεση και αραδιασα τα πραγματα να στεγνωσουν απο τη μπορα που ειχα βρει πριν λιγο στη διαδρομη, ενω το catamaran ξεκινησε για το Kristiansand της Νορβηγιας! 

Ανοιξα τους χαρτες και αρχισα να σχεδιαζω τη διαδρομη της ημερας μου. Ο αρχικος σχεδιασμος μου ηταν να ξεκινησω απο το λιμανακι του Kristiansand και να ανεβω περιπου 300 χιλιομετρα βορειοδυτικα με προορισμο το βουνο Forsand, οπου θα εμενα σε ενα ορεινο καταφυγιο για να επισκεφτω την επομενη μερα ενα απιστευτο γεωλογικο θαυμα της φυσης, το διασημο Preikestolen. 

Στη Γερμανια ο Κωστης οταν εμαθε για τα σχεδια μου ειχε να μου προτεινει διαφορες μικρες αλλαγες, και εδω ηταν μια απο αυτες: ``Νικο, αν μπορεις απεφυγε την εθνικη. Ο Ε39 δεν εχει τιποτα σπουδαιο να δεις. Παρε καλυτερα τον μικροτερο αλλα απειρως πιο γραφικο δρομο ``9'' προς το Evje και θα σε βγαλει στο πανεμορφο λιμανακι Lysebotn που πραγματικα δεν πρεπει να το χασεις. Απο εκει μπορεις να παρεις το φερυ και να σε περασει κοντα στο Forsand και να ανεβεις στο βουνο.''

Εχοντας διαβασει ενα προσφατο ταξιδιωτικο του Κωστα στη νοτια Νορβηγια και εχοντας δει φωτογραφιες απο τα μερη που μου προτεινε ημουν σιγουρος οτι η δικη του προταση ηταν απειρως καλυτερη απο τη δικη μου!

\photo{146.jpg}

Συντομα ομως τα σχεδια και οι σκεψεις για τις διαδρομες κοπηκαν αποτομα. Το πλοιο βγηκε στα ανοιχτα και εκει ξεκινησε και το μαρτυριο μου. Γενικα με τα πλοια δεν τα παω καλα και ζαλιζομαι ποσο μαλλον τωρα που μια απιστευτη θαλασσοταραχη τιναζε το θηριωδες σκαρι σαν καρυδοτσουφλο απο τη μια μερια στην αλλη. 
Και δεν μιλαμε για ενα τυχαιο σκαφος. Το Fjord Cat μπορει να μεταφερει 900 επιβατες και 240 αυτοκινητα, ενω ειναι ενα απο τα πιο γρηγορα επιβαταγωγα στο κοσμο και κατεχει το παγκοσμιο ρεκορ της πιο γρηγορης διασχισης του Ατλαντικου ωκεανου!
Σε διαφορα σημεια στο σαλονι ειχε οθονες οπου εδειχνε το στιγμα του πλοιου μεσω GPS καθως και την ταχυτητα του: ακομα και στη θαλασσοταραχη τωρα κρατουσε σταθερη ταχυτητα 76 χλμ/ωρα, με μεγιστη ταχυτητα τα 89 χιλιομετρα/ωρα!

Ισως λογω ομως και της ταχυτητας του, το πλοιο εσκαγε με βια στα κυματα και εγω υπεφερα. Εκατσα στη πολυθρονα και εβαλα το κεφαλι πανω στο τραπεζι προσπαθωντας να ηρεμησω το στομαχι που ειχε σκαρφαλωσει στη πλατη μου! Η Αυρα τι να εκανε αραγε τωρα; Θα την εβρισκα στη θεση της η ξαπλωμενη φαρδια πλατια πανω σε κανενα αυτοκινητο; 

Μετα απο δυο μαρτυρικες ωρες το μαρτυριο εδειχνε να ειχε τελειωσει... Η θαλασσα ειχε καλμαρει και το στιγμα εδειχνε οτι μπαινουμε στο λιμανι του Kristiansand. 
Βγηκα στο μικροσκοπικο καταστρωμα οπου με περιμενε το πιο ομορφο θεαμα! Η Νορβηγια με καλοσωριζε με τον πιο ομορφο τροπο! 

\photo{147.jpg}

Κατεβηκα στο αμπαρι με το αγχος του πως εβρισκα τη μηχανη. Τελικα ηταν στη θεση της και με περιμενε να εξερευνησουμε αυτη τη αγνωστη γη. 
Σε λιγο οι ροδες της Αυρας πατουσαν Ν ο ρ β η γ ι α ! 
Ο καιρος ηταν καταπληκτικος! 25 βαθμοι και λιακαδα! Τι αλλο θα μπορουσα να ζητησω; 

\photo{148.jpg}

Εξω απο το λιμανι εκανα μια σταση για ανεφοδιασμο που ηξερα οτι θα ...πονεσει. Εδω η απλη βενζινη ειχε 2 ευρω το λιτρο! Εβγαλα απο το μυαλο της σκεψη οτι η βενζινη στη πιο πλουσια χωρα της Ευρωπης ειχε μολις 10-15 λεπτα διαφορα με τη τιμη στη Μπανανια και χαιρετησα τα 2 παιδια με τις Ducati που ηρθαν στο πρατηριο για να γεμισουν. Δανοι που γυριζαν απο μια ολιγοημερη βολτα στα περιξ, μου προτειναν και αυτοι την Οδο 9 και χαμογελασα: ``Ο γκουρου ειχε δικιο!''

Πηρα το δρομο προς την περιφημη επαρχιακη οδο και μπορει να ηταν η ιδεα μου αλλα ολα εδω πανω μου εμοιαζαν εντελως διαφορετικα. Τα σπιτια, οι δρομοι, η φυση, η καθαριοτητα... 

\photo{149.jpg}

H διαδρομη ανοιγοταν μπροστα μεσα σε μια Φυση απεριγραπτη! Σαν απο αλλο πλανητη! Δρομοι που χανονταν στο βαθος του οριζοντα, κινηση μηδενικη, ασφαλτος αψεγαδιαστη σαν ψευτικη, καταφυτα δαση, λοφοι, νερα...
Και ακομα δεν ειχα δει τιποτα.... 

\photo{150.jpg}

Οκ, εδω χρειαζεται λιγο προλογο....
Σε αυτη τη χωρα υπαρχει νερο. ΠΟΛΥ νερο. Λιμνες, ποταμια, καταρρακτες, θαλασσες... 
Ενας καλος φιλος καποτε ειπε οτι ``η Νορβηγια σου δινει την αισθηση οτι ειναι μικρες νησιδες γης που επιπλεουν πανω στο νερο'' και ειχε απολυτο δικιο.

Και εκεινη η πρωτη φορα που θα αντικρυσεις μια απο τις αμετρητες λιμνες στη Νορβηγια ειναι πολυ ιδιαιτερη... 
Καθως περνουσα αναμεσα στο δασος, τα δεντρα ανοιξαν για να φανερωσουν ενα α π ε ρ ι γ ρ α π τ ο θεαμα.
.
.
.
.
.
.
.
.
.
.
.
.
.
.
.
.
.
.
.
.
.
.
.
.

\photo{152.jpg}

Αφησα την Αυρα στην ακρη του δρομου.... 

\photo{153.jpg}

...και καθισα να κοιταζω αφωνος το μαγευτικο σκηνικο! Αρχισα να γελαω αμηχανα και η μονη σκεψη μου ηταν ``ΤΙ βλεπω θεε μου....''

\photo{154.jpg}

Δεν ξερω ποση ωρα εκατσα εκει και τραβουσα φωτογραφιες... Δεν χορταινα τις εικονες που εβλεπα μπροστα μου!
Συνεχισα και οσα υπερθετικα και να χρησιμοποιουσα δεν εφταναν για να περιγραψουν την ομορφια αυτο του τοπου...
Περικυκλωμενος απο ατελειωτα δαση....

\photo{155.jpg}

Λιμνες μεσα σε τοπια σαν απο καρτ-ποσταλ ...

\photo{156.jpg}

Εικονες τοσο ομορφες που ελεγα οτι δεν μπορει να ειναι αληθινες...!

\photo{157.jpg}

Αυτο ομως ηταν η Νορβηγια. Εκει που νομιζες οτι τα ειχες δει ολα στην επομενη στροφη ερχοταν μια νεα εικονα να σε αφησει αναυδο!

\photo{158.jpg}
\photo{159.jpg}

Μεσα σε αυτο το απεριγραπτο σκηνικο σταματησα στην ακρη του δρομου και εσβησα τη μηχανη. 

\photo{160.jpg}

Εκανα να βγαλω το κρανος και ενοιωσα τα ματια μου υγρα... Το μυαλο μου παλευε να χωρεσει τα οσα μετεφεραν οι αισθησεις... Τα τοπια, οι μυρωδιες, οι ηχοι... Ημουν σε ενα κοσμο αληθινα παραμυθενιο και αυτο δεν το ειχα ζησει ποτε πριν. Θελω να πιστευω οτι εχω ταξιδεψει λιγο στη ζωη μου, ομως ΠΟΤΕ και πουθενα δεν ειχα δει κατι σαν ολα αυτα που εβλεπα τωρα εδω μπροστα μου....
Πως γινοταν να υπαρχουν τετοια μερη στο πλανητη μας και να μην εχουμε ιδεα καν;
Κοιταξα την καλη μου κατακοκκινη Αυρα και χαμογελασα περηφανα. Κοιτα μεχρι που με εχεις φερει καλη μου...

Καπου εκει η ησυχια του τοπου εσπασε απο τον χαρακτηριστικο μπασο ηχο δικυλινδρων μηχανων. Γυρισα και ειδα δυο μηχανες που ηρθαν και αραξαν διπλα μου: μια λευκη BMW GS 800 και ενα πορτοκαλι Kawasaki ER6. Ηταν ενα ζευγαρι σαρανταρηδων Σουηδων, ο Christian και η Asa, που ειχαν ξεκινησει τη προηγουμενη μερα απο το Örebro της Σουηδιας με τις μηχανες τους και θα κανανε ενα ταξιδακι μερικων ημερων στη Νοτια Νορβηγια. Πολυ ευχαριστα παιδια! Αφου ανταλλαξαμε τις κλασσικες αβροφροσυνες για τις μηχανες μας, με ρωτησαν με μεγαλο ενδιαφερον απο που ερχομουν και που πηγαινα και συντομα πιασαμε τη κουβεντα περι ταξιδιων και των χωρων που ειχαμε δει... 
Ο Christian ηταν σχετικα νεος μηχανοβιος μιας που ειχε τη μηχανη του μολις 3 χρονια αλλα ειχε καταφερει να γραψει πολλα χιλιομετρα σε ταξιδια εντος και εκτος δρομου σε ολες τις Σκανδιναβικες χωρες. Δικοι μου ανθρωποι! 
Ποσο χαιρομουν τετοιες μοτοσυναντησεις απο το πουθενα! Αυτοι οι ανθρωποι ειχαν σταματησει και ειχαμε πιασει κουβεντα απλα και μονο γιατι καβαλουσαμε μηχανη. Δεν χρειαζονταν συστασεις και περιττα λογια. Ειχαμε τη κοινη αγαπη της Μηχανης και του Ταξιδιου να μας ενωνει.

Συντομα τα παιδια καβαλησαν τις μηχανες τους και συνεχισαν. Οπως μου ειχαν πει πηγαιναν στην ιδια διαδρομη με εμενα, οποτε ισως τους εβρισκα παρακατω στο δρομο! 

Η διαδρομη τωρα επιασε να περναει μεσα απο μικρα χωρια οπου ειχα την ευκαιρια να κανω μερικες πρωτες διαπιστωσεις για τους Νορβηγους. 
Καταρχας οι Νορβηγοι αγαπουν τη φυση.
Η αρχιτεκτονικη των σπιτιων και μαγαζιων τους ηταν σε απολυτη αρμονια με το περιβαλλον γυρω τους. 
Πετρα και ξυλο παντου με ομορφες λεπτομερειες και στους πιο μικρους οικισμους.

\photo{161.jpg}

Επισης, οπως ολοι οι Σκανδιναβοι, λατρευουν το camping. Η Νορβηγια εχει κυριολεκτικα εκατονταδες χωρους κατασκηνωσης. Αριστερα και δεξια στο δρομο εβλεπα συνεχως αναμεσα στις συσταδες των δεντρων campings, αλλα πιο οργανωμενα και αλλα πιο ...ελευθερα.

\photo{162.jpg}

Αυτο ομως δεν ηταν τυχαιο. 
Βλεπετε στη Νορβηγια (οπως και στη Σουηδια και τη Φινλανδια) η διαβιωση στη φυση ηταν τροπος ζωης απο αρχαιων χρονων. 
Οι παντες εδω εχουν το δικαιωμα να μπαινουν, να διασχιζουν, ακομα και να διαμενουν σε ακαλλιεργητη γη οπουδηποτε στη χωρα -απο την ακρη του δρομου, μεχρι οποιαδηποτε ακτη, και οποιοδηποτε βουνο μεχρι ακομα και σε προστατευμενες περιοχες οπως Εθνικους Δρυμους!

Αυτο το δικαιωμα γνωστο στη χωρα ως \textbf{allemannsrett} (All man's right / Το δικαιωμα καθε ανθρωπου) υπηρχε ως καθιερωμενο εθιμο απο αρχαιων χρονοων και το 1957 περασε και ως νομοθεσια στο συνταγμα της Νορβηγιας. 
Βασιζεται στη φροντιδα για τη φυση και ολοι οι επισκεπτες αναμενονται να δειχνουν τον απαραιτητο σεβασμο προς τους αγροτες, ιδιοκτητες γης, αλλους επισκεπτες καθως και για το περιβαλλον. 
Η φιλοσοφια του σεβασμου του περιβαλλοντος φαινεται και στις πιο μικρες λεπτομερειες. Στη καλλιεργημενη γη δεν ειναι επιτρεπτο για ευνοητους λογους να διασχιζεται, ομως επιτρεπεται να τη διασχιζει κανεις οταν ειναι παγωμενη και σκεπασμενη με χιονι! 

Στην ουσια ο περιηγητης σε αυτες τις χωρες μπορει -ελευθερα και δια νομου- να διασχισει οποια περιοχη θελει χωρις κανενα περιορισμο και να διανυκτερευσει απολυτως οπουδηποτε μεχρι δυο 24ωρα, ενω ειναι ελευθερος να φαει τους καρπους των δεντρων και λοιπων εδωδιμων και φρουτων που μπορει να βρει, αρκει να σεβαστει το περιβαλλον και τυχον αλλους περιηγητες και να αφησει το μερος που εμεινε καλυτερο και απο οτι το βρηκε.

Χαρακτηριστικο του ποσο σοβαρα παιρνουν την ελευθερια του καθε ανθρωπου να περιηγειται ελευθερα στη φυση ειναι το οτι απαγορευεται αυστηρα η κατασκευη φρακτων και αλλων εμποδιων σε κοινοχρηστη γη, και αν αυτο συμβει πεφτουν πολυ βαρια προστιμα.

Τι να λεμε... Τι συγκρισεις να εκανα τωρα... Απιστευτοι λαοι, ετη φωτος μπροστα!

Ακομα και στις πιο μικρες λεπτομερειες οι ανθρωποι εδειχναν το σεβασμο τους στη φυση και το περιβαλλον.
Ναι, αυτο ηταν σταση λεωφορειου και οχι αυτη η ζουγκλα στην σκεπη ΔΕΝ ειχε γινει τυχαια, ουτε ειχε λογο υπαρξης -ηταν εκει απλα για διακοσμηση! Η πρακτικη του να βαλεις ...χωμα(!) στη οροφη του σπιτιου και να σπειρεις λουλουδια και πρασιναδα θεωρειται συνηθισμενη και ομορφη για τους Σκανδιναβους... 

\photo{162.jpg}

Αφησα πισω μου τα μικρα χωριουδακια και ο δρομος τωρα μπροστα ανοιχτηκε, ατελειωτος, αδειος μεχρι εκει που εβλεπε το ματι...

\photo{163.jpg}

Εβαλα στα ακουστικα το καλυτερο soundtrack \href{http://goo.gl/xUsd7}{Sivert Høyem - Give it a whirl} για αυτο εδω τον απιστευτο τοπο και η μπαλα χαθηκε... 
Δυναμωστε και απολαυστε!

\photo{164.jpg}

Ο δρομος περνουσε συνεχως τωρα διπλα σε ατελειωτες λιμνες και δεν ηξερα που να πρωτοκοιταξω! 

\photo{165.jpg}

Απο τη μια παντου νερα και απο την αλλη θεορατοι βραχοι και βουνα που υψωνονταν σαν γιγαντες διπλα μου...

\photo{166.jpg}

Τα δεκαδες τουνελ στη διαδρομη τρυπουσαν τους βραχους και εδιναν την αισθηση οτι με μεταφερουν σε ενα παραλληλο συμπαν...
Ωσπου μετα απο ενα ακομα τουνελ....

\photo{167.jpg}

...εμφανιστηκε μπροστα μια ακομα εικονα απο αυτες που παιρνεις μαζι σου μεχρι να κλεισεις τα ματια σου απο αυτο τον κοσμο...!
.
.
.
.
.
.
.
.
.
.
.
.
.
.
.
.
.
.
.
.
.
.
.
.

\photo{168.jpg}

Οh yeah baby, now we're talkin'....
Now ΜΥ waiting was done...

\photo{169.jpg}

Συντομα αφησα πισω τον οντως καταπληκτικο Δρομο 9 και μπηκα σε ενα μικρο δρομακι μεσα στο δασος που μου εδειχνε το GPS. 
Ο δρομος αρχισε να ανηφοριζει προς τα πανω και ειχε στενεψει απιστευτα! Η Αυρα επιανε ολο το πλατος του δρομου και δεν ηθελα καν να σκεφτω τι θα γινοταν αν 2 αυτοκινητα συναντιοντουσαν εδω περα -ναι, αυτο που βλεπετε ηταν δρομος διπλης κατευθυνσης!

\photo{170.jpg}

Συνεχισα τη πορεια προς το βουνο και στην οθονη το GPS ελεγε ``Αγνωστη Οδος''. Γαμω! Τη πρωτη μερα μου στην Νορβηγια και ηδη διεσχιζα αγνωστους δρομους που δεν υπαρχουν καν στο χαρτη! Αυτα ειναι! 

\photo{171.jpg}

Ο δρομος ολο και ανηφοριζε και η θεα απο εκει πανω ηταν μοναδικη...

\photo{172.jpg}

Εδω ομως αρχιζαν τα πραγματικα απιστευτα!
Συντομα η διαδρομη με εβγαλε σε ενα υψιπεδο, που ομως δεν εμοιαζε με ΟΤΙΔΗΠΟΤΕ αλλο ειχα δει στο παρελθον!
Αγρια πετρα, πληρης ελλειψη οποιασδηποτε βλαστησης, κανενα δεντρο, ομως παντου νερο και χιονι! 

\photo{173.jpg}

Το τοπιο ηταν εντελως εξωπραγματικο! Εμοιαζε σαν κατι τοπια που ειχα δει σε φωτογραφιες στα 4000+ μετρα σε βουνα οπως το Εβερεστ, ομως εδω το GPS διαφωνουσε: βρισκομουν μολις στα 1000 μετρα υψομετρο! 
Τοτε πως; Που ημουν; Τι ηταν ολο αυτο εδω το πραγμα;

\photo{174.jpg}

Συνεχισα να προχωραω μεσα σε αυτο το εξωγηινο τοπιο μεχρι που ειδα ενα γνωριμο θεαμα!

\photo{175.jpg}

Ο Christian και η Asa ηταν και αυτοι εδω πανω στην ερημια, φωτογραφιζοντας τα ...αξιοθεατα!
Τι φοβερη συμπτωση! Χαιρομουν παρα πολυ που εβλεπα δυο γνωριμα προσωπα σε αυτο το τοσο αλλοκοτο περιβαλλον. 

\photo{176.jpg}

Αρχισα να συζηταω με τον Christian για το μερος. Τι ηταν ολο αυτο; 
Ο Christian γυρισε και μου ειπε: ``Στη Νορβηγια οπως υπαρχουν τα Fjord υπαρχουν και τα Fjell.''
Κυριολεκτικα Fjell σημαινει ``βουνο'' στα Νορβηγικα και στην ουσια ηταν τεραστιες επιπεδες εκτασεις στις κορυφες των οροσειρων της χωρας που ειχαν ισοπεδωθει πριν εκατομμυρια χρονια απο τους ιδιους τεραστιους παγετωνες που ειχαν δημιουργησει και τα fjord. Μαλιστα τα βουνα αυτα αρχικα ειχαν ως και ΠΕΝΤΕ φορες μεγαλυτερο υψομετρο απο οτι τωρα φτανοντας σε υψος και τα 10.000 μετρα.
Κατα το μεγαλυτερο μερος του χρονου τα fjell ηταν σκεπασμενα απο χιονι, ενω οταν το χιονι υποχωρουσε αφηνε παντου λιμνες και ελαχιστη βλαστηση.

Ωραια! Πριν καν να εβλεπα το πρωτο μου fjord θα εβλεπα τα fjell! Μα ποιος ημουν επιτελους;; 
Και παλι ομως αυτο δεν εξηγουσε το γιατι το τοπιο εδειχνε τοσο εξωπραγματικο. Και στην Ελλαδα ειχαμε βουνα που σκεπαζονται για μεγαλα διαστηματα με χιονι αλλα δεν ηταν ετσι ακομα και σε υψομετρα πανω απο 2000 μετρα. 

Καπου εκει θυμηθηκα κατι που ειχα διαβασει στον οδηγο του Lonely Planet. Εδω πανω δεν επαιζε ρολο το υψομετρο των βουνων. Μπορει το GPS να εδειχνε σχετικα χαμηλο υψομετρο αλλα εδω πανω ημασταν ψηλα πλεον στο πλανητη και τα βουνα ειχαν χαρακτηριστικα που σε μικροτερα πλατη εβρισκες σε πολυ μεγαλυτερα υψομετρα. Γεωγραφικο πλατος και οχι υψος...

Αν μου ελεγε καποιος οτι βρισκομουν σε αλλο πλανητη αυτη τη στιγμη θα τον πιστευα...

\photo{177.jpg}

Ρωτησα τα παιδια τι εκαναν εδω πανω. Μου ειπαν οτι πηγαιναν στο Lysebotn οπου και θα εμεναν! Τι συμπτωση! Απο ολα τα μερη που θα μπορουσαν να διαλεξουν, ειχαν επιλεξει να κανουν ακριβως την ιδια διαδρομη με μενα!\\

\dialogue{Τι λες; Παμε παρεα μεχρι εκει;}
\dialogue{Φυγαμε!}\\

Ετσι ξεκινησαμε πορεια ολοι μαζι μεσα απο αυτο το εκπληκτικο τοπιο και σκεφτομουν ολα τα απιθανα πραγματα που ειχαν γινει μεχρι στιγμης! Μερη που δεν μπορουσα να φανταστω οτι υπαρχουν στο κοσμο και τωρα ημουν παρεα με δυο Σουηδους που ειχα γνωρισει πριν μερικες ωρες ετσι απλα στο δρομο!
Η ασφαλτος ξετυλιγοταν μπροστα μου ατελειωτη, σε μια διαδρομη που ομοια της δεν ειχα ξανακανει ποτε στη ζωη μου...

\photo{178.jpg}

Τι τοπιο....! Εκει που νομιζα οτι δεν θα μπορουσα να δω κατι ακομα καλυτερο απο οσα ειχα δει μεχρι τωρα, ερχοταν ΑΥΤΗ η εικονα και με εκανε να κοιταζω χωρις να ξερω τι να πω... Τα συννεφα απλωνοντας απο πανω μεχρι εκει που εβλεπε το ματι και εφτιαχναν μια αληθινα Βιβλικη σκηνη... 

\photo{179.jpg}
\photo{180.jpg}

Δεν ηξερα τι να πω για ολο αυτο που εβλεπα... Πολλες φορες εχω θαυμασει την ομορφια της φυσης σε μια διαδρομη αλλα ηταν ελαχιστες οι φορες που πραγματικα δεν μπορουσα να πιστεψω στα ιδια μου τα ματια! 

Ο δρομος ανεβαινε και κατεβαινε συνεχως ακολουθωντας το αναγλυφο του παραξενου τοπιου και μου χαριζε εικονες μοναδικης ομορφιας μεχρι το ακρο του οριζοντα! 
Ηθελα να καθομαι να κοιταω το τοπιο με τις ωρες...

\photo{181.jpg}
\photo{182.jpg}

Δεν ειχα ξαναδει ποτε κατι παρομοιο σαν τις εικονες αυτες! 

\photo{183.jpg}
\photo{184.jpg}

Ενοιωθα απειροελαχιστος μπροστα στο Μεγαλειο μιας συγκλονιστικης Φυσης...
Τι να ελεγα αλλο για τετοιες εικονες...;

\photo{185.jpg}

Ο δρομος οσο πηγαινε στενευε και αλλο και ανεβοκατεβαινε το βουνο στριφογυριζοντας, με μια νεα εικονα πισω απο καθε στροφη...

\photo{186.jpg}

...ομως δεν τιποτα δεν μπορουσε να με προιδεασει για αυτο που θα εβλεπα τωρα μπροστα μου....!
.
.
.
.
.
.
.
.
.
.
.
.
.
.
.

\photo{187.jpg}

Καποια στιγμη φτασαμε στο ψηλοτερο σημειο του βουνου και αντικρυσα το πιο αλλοκοτο θεαμα: παντου γυρω μου απειρες μικρες και μεγαλες στιβες απο πετρες! Και οταν λεμε παντου εννοουμε παντου!

\photo{188.jpg}

Εσβησα τη μηχανη και κατεβηκα. Περιπλανηθηκα στο χωρο χαζευοντας το περιεργο αυτο θεαμα... Τι ηταν ολα αυτα; Τοτε εφερα στο μυαλο μου ενα παλιο ταξιδιωτικο στην Ανατολια που ανεφερε οτι οι ταξιδιωτες που περνουσαν απο μερη καποιας σημασιας ηθελαν να αφησουν το σημαδι τους για καλη τυχη και ετσι εστηναν μικρους η μεγαλους σωρους απο πετρες για να δειξουν οτι περασαν απο εδω.

Διαβασα στον οδηγο: "Αυτες οι στιβες με πετρες που βρισκονται διασπαρτες σε ολη τη χωρα ειναι γνωστες ως Cairns. Η λεξη προερχεται απο τη Σκοτσεζικη διαλεκτο càrn και τετοιες στιβες βρισκονται παντου στο κοσμο, οπου υπαρχει καποιος μεγαλος λοφος, κορυφη βουνου, καταρρακτες η ακρωτηρια, αλλα και σε ανυδρα μερη οπως ερημοι και τουνδρες. Διαφερουν σε μεγεθος, απο μικρες στιβες μεχρι μεγαλοι τεχνητοι λοφοι, αλλα και σε περιπλοκοτητα, απο χαλαρες πετρες στιβαγμενες η μια πανω στην αλλη ως και επιτευγματα ογκολιθικης τεχνικης. Τα Cairns μπορει να ειναι βαμμενα η διακοσμημενα για μεγαλυτερη ορατοτητα η ακομα και για θρησκευτικους λογους.

Αυτη η πρακτικη ακολουθειται απο την προιστορια, οπου τα cairns χρησιμοποιουνταν για κυνηγι, αμυντικους, θρησκευτικους η ακομα και αστρολογικους σκοπους. Σημερα χρησιμοποιουνται απλα ως μνημεια σηματοδοτοντας τη παρουσια καποιου στη περιοχη."
Δεν θα μπορουσα λοιπον να παραλειψω να φτιαξω και εγω τη δικη μου μικρη στιβα απο πετρες αναμεσα στις αλλες. Τη σημασια της την ηξερα εγω και αρκουσε...

\photo{189.jpg}

Βεβαια ακομα πιο κλασσικη ηταν η πρακτικη των ταξιδιωτων να βαζουν ενα αυτοκολλητο η ενα σημειωμα στις πινακιδες του δρομου και εδω μαλλον το ειχαν παει ενα βημα παραπερα...

\photo{190.jpg}

Η καλη μου Αυρα παντα διπλα μου συντροφος πιστος, ετοιμη να με ταξιδεψει ακομα παραπερα...

\photo{191.jpg}

...μεσα σε τοπια απιστευτης αγριας ομορφιας...

\photo{192.jpg}

...και εκει που νομιζα οτι δεν υπηρχε κατι αλλο πλεον να με εντυπωσιασει ο δρομος επιασε να κατηφοριζει προς το Lysebotn και για αλλη μια φορα εμεινα να κοιταω σαν χαζος...
Το αναγλυφο του τοπιου και ο δρομος που χανοταν και στριφογυριζε σαν φιδι μεσα σε ολο αυτο ηταν ενα θεαμα απλα μοναδικο... 
Τοσα ταξιδια και τετοια ομορφια δεν ειχα ποτε μου και πουθενα αλλου...!

\photo{193.jpg}
\photo{194.jpg}

Εδω και ωρα ταξιδευα μεσα στο απιστευτο αυτο τοπιο παρεα με το ζευγαρι των Σουηδων, ομως συντομα στη παρεα προστεθηκαν και καποια αλλα παιδια απο τη Δανια... Δεν ηθελα να τελειωσει ποτε αυτη η διαδρομη!

\photo{195.jpg}

...Καπως ετσι αρχισαμε να κατηφοριζουμε ολοι μαζι το πασο προς το λιμανακι του Lysebotn, στη μυτη του διασημου Lysefjord (ελληνιστι το Φιορδ Lyse).

\photo{196.jpg}

Οι εκπληξεις ομως δεν ελεγαν να τελειωσουν! Η Νορβηγια εκτος ολων των αλλων ηταν διατρητη απο αμετρητα τουνελ ολων των μεγεθων, διατομων και σχεδιων. Αλλα λιγων εκατονταδων μετρων και αλλα πολλων χιλιομετρων, αλλα μεσα απο βραχο σε μεγαλο υψομετρο και αλλα υποθαλασσια... Λιγα ομως τουνελ μπορουσαν να συγκριθουν με το τουνελ που βρισκοταν τωρα εδω... 
Μπηκα στο στενο τουνελ. 
Σταγονες νερου επεφταν απο το γυμνο βραχο και ο χαμηλος κιτρινος φωτισμος εδινε στο μερος μια σχεδον τρομακτικη οψη.
Ξαφνικα ο δρομος αρχισε να κατεβαινε αποτομα προς τα κατω και ο ηχος του V4 της Αυρας αντηχουσε πανω στα τοιχωματα του τουνελ σαν ουρλιαχτο. Και τοτε το ειδα: μια στροφη 180 μοιρων ΜΕΣΑ στο τουνελ! Εκοψα οσο μπορουσα και πηρα τη στροφη προσεκτικα... Απιστευτο και ομως αληθινο...

Και εκει που τα λογια ειναι πολυ φτωχα για να περιγραψουν μια κατασταση, ισως η εικονα τα καταφερει καλυτερα...
Δειτε, ακουστε, απολαυστε και ...μην τρομαξετε! \href{http://www.youtube.com/watch?v=Gt4kouWnKh0}{Lysebotn Tunnel}

Και τελικα... φως! Πηρα μια βαθεια ανασσα και το μαγευτικο τοπιο με καλοσωριζε στη γη των Φιορδ....

\photo{197.jpg}

Εδω το λογο ειχαν οι τιτανιοι ογκοι βραχου που κρεμονταν πανω απο τα κεφαλια μας αριστερα και δεξια του δρομου.... 
Νερα ετρεχαν απο παντου γυρω και επεφταν με δυναμη απο υψος εκατονταδων μετρων!

\photo{198.jpg}

Τα μεγεθη σε αυτη τη χωρα ειχαν πλεον χασει καθε νοημα!
Μπορειτε να βρειτε τη μηχανη σε αυτη την εικονα;

\photo{199.jpg}

Ο Κωστης στο δικο του ταξιδιωτικο ειχε περιγραψει με τα πιο ομορφα λογια το Lysebotn. Εφτασα στο μικροσκοπικο λιμανακι και αφησα τη κατακοκκινη κουκλα μου να ξεκουραστει και να γινει και αυτη μερος αυτου του μαγικου τοπου...

\photo{200.jpg}

Και εκει ηταν που Το Αντικρυσα. 
Το Ονειρο. 
Το πρωτο μου φιορδ. 
Αυτο που τοσα χρονια ειχα σαν αμυδρη εικονα στο μυαλο μου ανοιγοταν τωρα μπροστα μου και ειχα παραλυσει.
Εμεινα αναυδος και το μυαλο μηδενισε. 
.......
... Ημουν εδω...!
.
.
.
.
.
.
.
.
.
.
.
.
.
.
.
.
.
.
.
.
.


\photo{201.jpg}

Ο Christian και η Asa στο μεταξυ ειχαν παει παραδιπλα στο τοπικο camping, οπου ειχαν αποφασισει να διανυκτερευσουν σημερα. 
Εγω ειχα αρχικο σκοπο να παρω το καραβακι για την αλλη ακρη του φιορδ και το Forsand, αλλα τωρα αληθινα δεν ηθελα καν να σκεφτω να φυγω απο εδω! 
Και να ηθελα ομως δεν θα μπορουσα τελικα: βλεπετε το τελευταιο φερυ της μερας ειχε μολις φυγει και το επομενο ηταν αυριο το πρωι.
Το αρχικο προγραμμα μου πηγαινε περιπατο αλλα ποιος θα παραπονιοταν σε ενα τετοιο μερος και με τετοιο καιρο; Εξαλλου εδω θα ειχα και τους νεους μου φιλους παρεα! 

Παμε να στησουμε και να βολευτουμε λοιπον!
Το camping ηταν ακριβως διπλα στη θαλασσα και πολυ φροντισμενο αν και η βλαστηση περιοριζοταν σε μερικα λιανα δεντρακια στην εισοδο του... Σκια στο χωρο των σκηνων; you've got to be kidding!

\photo{202.jpg}
\photo{203.jpg}

Εστησα τη σκηνη σε ενα επικο τοπιο... Στο βαθος πισω ενας τεραστιος καταρρακτης επεφτε απο τα καθετα βραχια και στα αριστερα μου η θαλασσα του φιορδ με τον ηλιο να πεφτει σιγα σιγα...

\photo{204.jpg}

Δεν υπηρχε περιπτωση να αφησω την ευκαιρια να εξερευνησω αυτο το μερος! Τα παιδια ειχαν πιασει ενα ξυλινο σπιτακι και κανονισαμε να βρεθουμε σε καμια ωριτσα για φαγητο, οποτε ειχα χρονο μπροστα μου να απολαυσω το απιστευτο σκηνικο. 

Κατευθυνθηκα προς τον τεραστιο καταρρακτη και ενοιωθα μια αγρια χαρα σαν το μικρο παιδι με το καλυτερο παιχνιδι του κοσμου!
Δεν ειχα ξαναδει τετοιο θεαμα απο τοσο κοντα... Ο θεορατος βραχος πρεπει να ειχε υψος τουλαχιστον 200 μετρα και το νερο ετρεχε απο τη κορυφη φτιαχνοντας ενα σωρο μικροτερους καταρρακτες που εσκαγαν κατω με την εκκωφαντικη βουη του νερου που χτυπαει στα βραχια...

\photo{205.jpg}

Το μερος ηταν απεριγραπτο...! 
Περπατουσα στη σκια του καταρρακτη στη μικρη παραλια και τριγυρω μια παρεουλα τουριστων απολαμβαναν και αυτοι το μερος με μπαρμπεκιου, μπυριτσες και μουσικη απο το αμαξι τους...

Μπροστα μου ο ηλιος ειχε παρει να δυει και εβαφε τα καθετα τεραστια βραχια με χρυσα χρωματα, ενω το φως που επεφτε απο το πλαι στο βαθος εμοιαζε σαν ασπρογαλανες κουρτινες ριγμενες στον οριζοντα... Μια εικονα μοναδικη...

\photo{206.jpg}

Ποσα απεριγραπτα πραγματα ειχα δει μεσα σε μια μονο μερα; 
Ποσες φορες ειδα εικονες που ελεγα οτι δεν υπαρχει περιπτωση να δω κατι καλυτερο; 
Και ομως... Καπου εκει η Φυση βαλθηκε να με διαψευσει για αλλη μια φορα χαριζοντας μου μια απο τις καλυτερες εικονες της ζωης μου...


Lysefjord Norway, 9.50 μμ

\photo{207.jpg}

Με αυτη την εικονα στα ματια γυρισα στο camping και η μερα εκλεισε με τον καλυτερο τροπο: με τη παρεα φιλων!
Μαζι με τους γλυκυτατους Σουηδους ζευγαρι πηγαμε σε ενα γραφικοτατο τοπικο εστιατοριο και απολαυσαμε τις τοπικες λιχουδιες, με μπυριτσα και κουβεντουλα για τα παντα. Μηχανες, φιλια, σχεσεις, ταξιδια.
Ποσο μου αρεσει αυτη η Σκανδιναβικη αρχιτεκτονικη!

\photo{208.jpg}

Εβλεπα τους νεους φιλους μου να μιλανε, να γελανε και να αλληλοπειραζονται και χαμογελασα... Σκεφτομουν οτι τελικα ειδικα εμεις οι μηχανοβιοι ειμαστε παντου απο την ιδια ``παστα''! Θα μπορουσα καλλιστα να ειμαι αναμεσα σε παλια φιλαρακια σε καποια εκδρομη στα περιξ των Αθηνων και να γελαμε κουβεντιαζοντας ακριβως με τον ιδιο τροπο...
Στερεοτυπα χρονων για κρυους και απομακρους Σκανδιναβους που δεν ανοιγονται ευκολα σε αγνωστους ανθρωπους τωρα χανονταν σαν καπνος. Αυτο τελικα ειναι και το μεγαλυτερο δωρο ενος ταξιδιου: σου ανοιγει τα ματια στην πραγματικη ζωη που περιμενει εκει εξω...

Γυρισα στη σκηνη και επεσα να κοιμηθω με ενα χαμογελο μεχρι τα αυτια.
Η ωρα ηταν 11.30 το βραδυ αλλα το φως εξω ηταν οπως το σουρουπο στην Ελλαδα! 
Καλωσηρθες στη γη που δεν νυχτωνει ποτε φιλε μου....
