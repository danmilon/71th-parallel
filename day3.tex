\chapter{Day 3 -- Guebwiller - Colmar - Leverkuzen - 504km}

Ο ηλιος που περναγε μεσα απο τις κουρτινες στο παραθυρο επεσε στα ματια μου και με ξυπνησε. Χαμογελασα. Οταν βρισκεσαι στο δρομο η παρουσια του ηλιου ειναι το καλυτερο δωρο που μπορεις να ζητησεις.
Ανοιξα τα ματια και κοιταξα γυρω το σαλονι καλυτερα. Απεναντι μου ενα τεραστιο καλλιτεχνικο κολαζ απο Νεα Υορκη -κατι αναμεσα σε πινακα ζωγραφικης και φωτογραφιες- πλαισιωνοταν απο μικες κορνιζες και κουκλακια κρεμασμενα στο τοιχο.
Στο τραπεζακι του σαλονιου και το περβαζι του παραθυρου υπηρχαν γλαστρακια και διαφορα παλια αντικειμενα στολισμενα. Φοβερη διακοσμηση! Παντου τριγυρω στο σπιτι υπηρχαν μικρες εξαισιες λεπτομερειες: παλια αντικειμενα, πινακες ζωγραφικης, κολαζ, χειροτεχνιες, κουκλακια, κατασκευες και χρωματα. Πολλα χρωματα.

\photo{80.jpg}

Ηταν νωρις ακομα και η Christie κοιμοταν. Σηκωθηκα και χαζεψα λιγο τριγυρω.
Το σπιτι ηταν πολυ ομορφο. Πολυ παλιο (πανω απο 200 ετων οπως εμαθα αργοτερα) με μεγαλα ξυλινα δοκαρια να προεξεχουν εκτεθειμενα στην οροφη και τους τοιχους και πολλα δωματια με μικρα περασματα χωρις πορτες. Η αρχιτεκτονικη ηταν στο στυλ των Γερμανικων χωριατοσπιτων, λογικο αν σκεφτει κανεις οτι η Αλσατια ηταν επι πολλους αιωνες Γερμανικη επαρχια.
Η Lolotte ειχε ηδη ξυπνησει και με κοιτουσε με περιεργεια απο απεναντι: ``τι γυρευεις στο σπιτι του αφεντικου μου;''

\photo{81.jpg}

Συντομα ξυπνησε και η Christie και αρχισε να ετοιμαζει πρωινο ενω εγω εκατσα να δω το προγραμμα της ημερας.
Σημερα θα πηγαιναμε μαζι με τη Christie να δουμε το περιφημο Colmar και μετα θα αναχωρουσα για το Leverkusen της Γερμανιας οπου με περιμενε ο πολυ αγαπημενος μου φιλος Κωστας -ο διεθνως διασημος συνταξιδευτης Sieche- με μια μικρη σταση πρωτα στη Χαιδελβεργη για να δω τον παντα χαμογελαστο και εξαιρετικα ευγενικο δεν-με-λενε-Στεφανο-αλλα-Σταυρο akaSteven ER.
Διεθνης συναντηση mybike λεμε -η μερα ηταν αφιερωμενη στους φιλους!

\photo{82.jpg}

Εξω η λιακαδα ηταν απλα υπεροχη! Η Christie με περιμενε ηδη εξω και καθισαμε για πρωινο στο μακραν πιο ομορφο μπαλκονι που εχω δει ποτε!

\photo{83.jpg}

Πραγματικα ηταν απεριγραπτο. H φοβερη διακοσμηση του σπιτιου εδω εδινε ρεστα!
Παντου κουκλακια, γλαστρες, λουλουδια και φανταστικες χειροτεχνιες και τοσο μα τοσο χρωμα...

\photo{84.jpg}
\photo{85.jpg}

Καθως ο πρωινος ηλιος επεφτε ελουζε το μικρο μπαλκονι εκανε τα παντα να μοιαζουν ακομα πιο εντονα και ζωντανα: Μια απιστευτη εκρηξη χρωματων!

\photo{86.jpg}

Καθως τρωγαμε πρωινο και συζητουσαμε για τη καθημερινοτητα εκει μια επιμονη σκεψη μου ηρθε στο νου: κοιτα που ειμαι τωρα! Αν δεν ηταν το Couchsurfing τωρα θα βρισκομουν σε ενα τυπικο camping, hostel η χειροτερα σε ενα απροσωπο ξενοδοχειο, αγνωστος μεταξυ αγνωστων.
Αλλα αντι για αυτο καθομουν εδω και απολαμβανα το καλυτερο πρωινο που θα μπορουσα να σκεφτω σε ενα πανεμορφο Γαλλικο σπιτι με εξαιρετικη παρεα.
Τα λογια ειναι φτωχεια... Κοιταχτε απλα που και πως ζουν αυτοι οι ανθρωποι! (και να σκεφτει κανεις οτι για εμενα η ιδεα του πρωινου αρχιζε και τελειωνε σε ενα νες καφε στο ποδι)

\photo{87.jpg}

Ποσο φοβερο ειναι αυτο πραγματικα... Ανθρωποι χωρις να με γνωριζουν μου ανοιγουν τα σπιτια τους, μου προσφερουν απλοχερα την παρεα και με βαζουν στις ζωες τους ετσι απλα. Με καλη διαθεση και πιστη ακομα στη καλοσυνη των ανθρωπων.
Φιλια, καλες προθεσεις, Εμπιστοσυνη... Ποσο πιο ομορφος θα ηταν ο κοσμος μας αν θυμομασταν ξανα τετοιες εννοιες που εχουμε χασει τωρα πια...

Καποτε κοιμομασταν με το κλειδι στη πορτα του σπιτιου για να μπορει να μπαινει ο γειτονας μας, ο φιλος, ο συγγενης. Τωρα ``καγκελα ασφαλειας μεταλουμιν'', συναγερμοι και διπλοκλειδωμα στην εξωπορτα της πολυκατοικιας...
Πας να μιλησεις σε εναν ανθρωπο στο δρομο και η καχυποψια κυριαρχει: τι θελει αυτος ο περιεργος απο εμενα; Γιατι μου μιλαει; Θελει να με κλεψει/βιασει/σκοτωσει;
Η απολυτη ειρωνια! Ο Ελλην ο ``φιλοξενος'': Εξω η φιλοξενια τυπου couchsurfing ειναι πολυ διαδεδομενη και ο κοσμος σε δεχεται στο σπιτι τους σαν να εισαι μερος της οικογενειας τους. Εδω οι περισσοτεροι το να φιλοξενησουν καποιον αγνωστο το θεωρουν αδιανοητο! ``Και που ξερω εγω οτι δεν θα με κλεψει; Που ξερω εγω αν θα με σκοτωσει το βραδυ;''

Ζουμε ζωες μιζερες, βαρετες και ασχημες γαμωτο -ποτισμενες απο το δηλητηριο του φοβου και της ελλειψης πιστης στο διπλανο μας- και για παρηγορια κανουμε shopping therapy: νεο κινητο, νεα τσαντα, νεες μποτες, νεο κρανος, νεο GSXR... Γιατι;
Αυτο εγω δεν το λεω ζωη: Καταντια ειναι. Λιγη θεληση χρειαζεται και μια δοση τρελας: να εμπιστευτουμε ξανα τον κοσμο γυρω μας. Να δωσουμε με χαμογελο σε οσους ζητανε ωστε οταν ζητησουμε και εμεις να εισπραξουμε το ιδιο... Τοσο πολλα ζηταω ρε παιδια; Εγω ειμαι ο τρελος η οι αλλοι χαζοι τελικα που δεν βλεπουν αυτα που βλεπω εγω;

``Παμε να δουμε τη πολη;'' η Christie με ξυπνησε απο τις σκεψεις μου. ``Φυγαμε''.
Η Αυρα ειχε και αυτη ...σπεσιαλ θεση για παρκινγκ κατω απο τις σκαλες του κουκλοσπιτου και ηταν κριμα να την ενοχλησω.

\photo{88.jpg}

Πηραμε λοιπον το -εννοειται Γαλλικο- αμαξακι της Christie για να παμε παρεα στη πολη.
Το Colmar μας περιμενε! (θεα με special guests τι αλλο; κουκλακια!)  

\photo{89.jpg}

Συντομα φτασαμε στη πολη... Το Colmar εδειχνε να στεκεται στο υψος της φημης του!

\photo{90.jpg}

Παντου μεγαλα παρκα με πολυ πρασινο, αψογη αισθητικη, καθαριοτητα και ατελειωτα παρτερια με λουλουδια...

\photo{91.jpg}

Οι ανθρωποι εδω ειναι πολυ ζωντανοι: φωνακλαδες, χαμογελαστοι, εγκαρδιοι. Καπως ετσι τους ειχα φανταστει τους Γαλατες...
(Μη ξεχνιομαστε! Ποδηλατο Πεζο ετσι; )

\photo{92.jpg}

Το Κολμάρ είναι πόλη της Αλσατίας στη βορειοανατολική Γαλλία. Η πόλη βρίσκεται στο Route de Vin (δρομο του κρασιου) της Γαλλιας αποτελώντας την πρωτεύουσα των περιφημων αλσατικών κρασιών.

Η πόλη ιδρύθηκε τον 9ο μ.Χ. αιώνα: Αναφέρεται για πρώτη φορά το 823, ως Columbarium -από όπου φαίνεται ότι προέρχεται και το σημερινό της όνομα- σε διάταγμα του Λουδοβίκου του Α' γιου του Καρλομαγνου, αλλά της δόθηκε το δικαίωμα να υφίσταται ως ελεύθερη αυτοκρατορική πόλη της Αγίας Ρωμαϊκής Αυτοκρατορίας το 1226 (civitatis). Το 1354 το Κολμάρ συμμετέχει στη δημιουργία της Δεκαπόλεως, μιας ομοσπονδίας δέκα αυτοκρατορικών πόλεων της Αλσατίας.

Το 1679 με τη συνθήκη του Νιμάγκεν το Κολμάρ αποδόθηκε στη Γαλλία και αποτέλεσε ``Βασιλική Γαλλική πόλη''. Στο καθεστώς αυτό παρέμεινε μέχρι το 1871, οπότε ολόκληρη η Αλσατία, με τη λήξη του Γαλλογερμανικού πολέμου, αποδόθηκε στη Γερμανία. Υπό γερμανική διοίκηση παρέμεινε μέχρι το τέλος του Α΄ Παγκοσμίου Πολέμου, οπότε με τη Συνθήκη των Βερσαλλιών η Αλσατία επεστράφη στη Γαλλία. Το Κολμάρ αναπτύχθηκε και το 1934 οι κάτοικοι φθάνουν σχεδόν τους 50.000.
Δυστυχώς όμως στον Β΄ Παγκόσμιο Πόλεμο οι Γερμανοί εισβάλλουν και καταλαμβάνουν εκ νέου την Αλσατία, την οποία προσαρτούν στο Γ' Ράιχ. Η πόλη υφίσταται καταστροφές μνημείων της και υφίσταται κατοχή με έντονα στοιχεία εκγερμανισμού και ναζιστικοποίησης. Η Γαλλία τελικά ανέκτησε εκ νέου τον έλεγχο της Αλσατίας ύστερα από τη μάχη του ``θύλακα του Κολμάρ'' το 1945.

\photo{93.jpg}

Η πόλη, βασιζόμενη στην αμπελοκαλλιέργεια αλλά και στη βιομηχανική της ανάπτυξη, ευημερεί από τον Πόλεμο ως σήμερα. Είναι μια πλούσια πόλη, με κύρια δύναμη της τον αναπτυσσόμενο τουρισμό, αλλά αποτελεί και έδρα γνωστών εταιριών όπως η Leitz και η Leibherr.
Βρίσκεται στο διαμέρισμα του Άνω Ρήνου σε απόσταση 68 χλμ. νοτιοδυτικά του Στρασβούργου. Ο ποταμός Ρήνος περνά 15 χλμ ανατολικά της, αλλά η πόλη συνδέεται με αυτόν μέσω ενός καναλιού, που συνδέει το Ρήνο με τον ποταμό Lauch, ο οποίος διασχίζει την πόλη, ενώ δυτικά της βρίσκεται η οροσειρά των Βοσγίων. Το θερμό και ξηρό μικροκλίμα της περιοχής του Κολμάρ είναι ιδιαίτερα ευνοϊκό για την καλλιέργεια αμπέλου, εξηγώντας γιατί η πόλη είναι το επίκεντρο παραγωγής και διακίνησης των Αλσατικών κρασιών.

\photo{94.jpg}

Ομως το κυριο αξιοθεατο εδω και ο λογος που η πολη ειχε αποκτησει τη φημη της πιο ομορφης στην Ευρωπη ηταν η ``la Petite Venise'' -η Μικρη Βενετια: μια ακρως γραφικη περιοχη στη καρδια της πολης που διασχιζεται απο μικρα καναλια του ποταμου Lauch, που στο παρελθον αποτελουσε το κεντρο κρεοπωλων, αλιεων και βυρσοδεψων.

\photo{95.jpg}

Η αρχιτεκτονικη εδω αντανακλα 8 αιωνες Γερμανικης και Γαλλικης σχεδιαστικης κουλτουρας η οποια εχει συνδιαστει περιφημα με τα τοπικα στυλιστικα εθιμα και υλικα κατασκευης: κοκκινος και κιτρινος ψαμμιτης και ξυλινα πλαισια με μεγαλα τμηματα τους να ειναι εκτεθειμενα ετσι ωστε να καταληγουν μερος του αρχιτεκτονικου σχεδιασμου.

\photo{96.jpg}
\photo{97.jpg}

Μπηκαμε μεσα στα μικρα σοκκακια και δεν ηξερα που να πρωτοκοιταξω!
Η ημερα ηταν γεματη φως και χαρουμενες φωνες απο τους επισκεπτες -το μερος εσφυζε απο ζωη.

\photo{98.jpg}

Πλακοστρωτα, μικρα καφε, bistrot...

\photo{99.jpg}
\photo{100.jpg}

...συντριβανια, καταπρασινα δεντρα, γεφυρακια και απειρα πολυχρωμα λουλουδια που αντικατοπτριζονταν στα νερα του μεγαλου καναλιου που κοβει την παλια πολη στα δυο.

\photo{101.jpg}
\photo{102.jpg}

Τα παλια παραδοσιακα σπιτια ηταν γεματα χρωματα βαμμενα ροζ, γαλαζια, κιτρινα, γαλαζια, καφε...
Μια σκετη πανδαισια!

\photo{103.jpg}
\photo{104.jpg}

Και η φοβερη λεπτομερεια: πολλα παραθυροφυλλα ειχαν στη μεση μια μικρη καρδια!

\photo{105.jpg}

Περπατησαμε για πολυ ωρα στα στενα, μη μπορωντας να χορτασω τις εικονες που εβλεπα. Σκεφτηκα το πως θα ηταν να ζουσα σε ενα τετοιο μερος...

\photo{106.jpg}

Θα κρατουσα αυτο τον ενθουσιασμο ανεπαφο; Ενας ντοπιος που μενει εδω αραγε αντιλαμβανεται οτι ζει σε ενα απο τα ομορφοτερα μερη του κοσμου η τα θεωρει ολα αυτα δεδομενα, αναξια λογου, βαρετα στην τελικη;

\photo{107.jpg}

Δεν ηξερα την απαντηση αλλα βλεποντας τη καθαριοτητα παντου και την πληρη ελλειψη γκραφιτι, αυτοκολλητων, αφισσων και λοιπων αισθητικων παρεμβασεων που συναντουσα συνηθως σε μεγαλες πολεις, μαλλον αν μη τι αλλο ηξεραν να σεβονται αυτο που εχουν...

\photo{108.jpg}

Ομως δεν ειχα δει ακομα αυτο για το οποιο ειχα ερθει...
Στο μυαλο μου απο τοτε που πρωτοδιαβασα ενα αρθρο για το Colmar υπηρχε μια συγκεκριμενη εικονα. Πολυχρωμα σπιτακια στις οχθες του ποταμου και απειρα λουλουδια να καθρεφτιζονται στα νερα.
Χαθηκαμε ξανα στα στενακια ψαχνοντας μια εικονα που μεχρι τοτε ειχα δει μονο στο αψυχο χαρτι.
Και ξαφνικα την ειδα εκει μπροστα μου. Ζωντανη. Και εχασα καθε μετρο συγκρισης... Απιστευτο!

\photo{109.jpg}

Τι να ελεγα για ΑΥΤΗ την εικονα που εβλεπα τωρα μπροστα μου; Πως να περιγραψω μια τοσο ομορφη εικονα που εμοιαζε να ειναι απο παραμυθι βγαλμενη;
Εμεινα απλα εκθαμβος να κοιταζω την ομορφια της φυσης και το μερακι των ανθρωπων...

\photo{110.jpg}
\photo{111.jpg}

Ειχε πλεον μεσημεριασει για τα καλα και η Christie με οδηγησε σε ενα ωραιο μαγαζακι που ηξερε για να δοκιμασω μια διασημη γαστρονομικη λιχουδια της Γαλλιας, το \href{http://en.wikipedia.org/wiki/Tarte_flamb%C3%A9e}{Tarte flambée}. Μια μεγαλη ανοιχτη πιτα με πολυ λεπτη ζυμη γεματη με φρεσκο λευκο τυρι, ζαμπον, ψιλοκομμενο κρεμυδι και κρεμα γαλακτος.

Ο θρυλος λεει οτι οι δημιουργοι αυτης της λιχουδιας ηταν γερμανοφωνοι αγροτες της Αλσατιας που συνηθιζαν να ψηνουν φουρνιστο ψωμι μια φορα τη βδομαδα. Η tart flambee χρησιμοποιουνταν για να δοκιμασουν τη θερμοκρασια του φουρνου. Στη καταλληλη θερμοκρασια ο φουρνος μπορουσε να ψησει τη ταρτα σε 1 με 2 λεπτα. Η κρουστα γυρω απο τη ταρτα κατεληγε να καει σχεδον απο τις φλογες (εξ' ου και flambee).

\photo{112.jpg}

Καθισα πισω και απολαμβανα τις στιγμες με ολες μου τις αισθησεις. Με τη πεντανοστιμη ταρτα, τη συνοδεια ενος εξαιρετικου λευκου κρασιου και τον υπεροχο ηλιο τωρα πραγματικα ημουν στο παραδεισο! Τι αλλο να ηθελα για να ειμαι ευτυχισμενος; Τα πιο ομορφα ειναι τα πιο μικρα παντα...

Η ωρα ομως πλεον ειχε περασει και οσο και να μην το ηθελα τωρα επρεπε να πηγαινω.
Ειχα 500 χιλιομετρα για το Leverkusen και ο καλος μου φιλος Κωστας με περιμενε. Δεν ελεγε να φτασω μεσανυχτα.
Φευγοντας αγορασα τα καθιερωμενα αυτοκολλητα μου και πηραμε το δρομο της επιστοφης για να ετοιμαστω. Στο βαθος οι καταπρασινοι αμπελωνες απλωνονταν αγερωχοι...

\photo{113.jpg}

Αρχισα να μαζευω τις βαλιτσες και να βαζω τη στολη αλλα ενας κομπος στο λαιμο δεν με αφηνε ησυχο. Η Christie ηταν ενας γλυκυτατος ανθρωπος -εξαιρετικα φιλοξενη, απιστευτα ζωντανη, χαρουμενη και αψογη οικοδεσποινα- και στεναχωριομουν πολυ που επρεπε να φυγω. Ομως δεν γινοταν αλλιως και το ηξερα. Η μοιρα του ταξιδιωτη. Τα παντα ρει και ουδεν μενει. 
Ως ειθισται, ανταλλαξαμε καποια μικρα δωρακια της χωρας μας ο καθενας και δωσαμε ραντεβου καπου καπως καποτε για το μελλον: 
A bientot Christie! Au revoir!

\photo{114.jpg}

Καβαλησα τη μηχανη. Τη σηκωσα απο το stand και τη ζυγισα πανω στα ποδια μου. Γνωριμη και οικεια η αισθηση, σαν μια υποσχεση οτι οσο ημασταν μαζι δεν μπορουσε να παει τιποτα στραβα. Εσυ και εγω. Εμεις.
Καπως ετσι θα ενοιωθε ενας καβαλαρης που ανεβαινει πανω στο αλογο του μετα απο καιρο. 
Κατεβασα τη ζελατινα και γυρισα το κλειδι στη μιζα. Ο ηχος του μοτερ γεμισε τη μικρη αυλη και ενοιωσα τη χαρα του ταξιδιου να διωχνει τη θλιψη της αποχωρησης. Η ωρα ηταν 3 το μεσημερι. Μπορει να ειχα αργησει λιγο αλλα ποτε δεν αντεχα να με κυνηγαει ο χρονος! Αυτες σημερα ηταν στιγμες πολυ σημαντικες για να τις ειχα χασει.

Ηλιος και 28 βαθμοι -ιδανικες συνθηκες. Επομενη σταση Χαιδελβεργη.

\photo{115.jpg}

Κατευθυνθηκα βορεια, παραλληλα με το ποταμο Ρηνο -το φυσικο συνορο μεταξυ Γαλλιας και Γερμανιας- απολαμβανοντας ενα μικρο κομματι της Γαλλικης επαρχιας. 
Σιγουρα ενα επομενο ταξιδι θα με βρει να εξερευνω καλυτερα αυτη την υπεροχη χωρα. 

\photo{116.jpg}

Λιγο εξω απο το Στρασβουργο η διαδρομη με οδηγησε πανω απο το πασιγνωστο ποταμο και ακριβως στη μεση του αλλο ενα συνορο: \textbf{Willkommen in Deutschland!}

\photo{117.jpg}

Αφησα πισω μου τους μικρους επαρχιακους δρομους καθως ηταν καιρος να γραψω χιλιομετρα. Η autobahn ανοιγοταν μπροστα μου και η Αυρα ηταν πανετοιμη να κανει αυτο για το οποιο φτιαχτηκε: ``παμε να κυνηγησουμε τον οριζοντα μωρο μου...''

\photo{118.jpg}

Μεχρι τη Χαιδελβεργη τα χιλιομετρα εφυγαν πολυ γρηγορα. 
Συναντησα το Σταυρο εξω απο τη δουλεια του και τα ειπαμε για λιγη ωρα. Ειχε ερθει στη Γερμανια πριν λιγο καιρο και θα καθοταν για ενα μικρο διαστημα δουλευοντας εκει. Πιασαμε κουβεντα περι ταξιδιων, νοοτροπιας των λαων και βεβαια -τι αλλο;- για μηχανες... Ηταν ενα γλυκυτατο παλικαρι, χαμογελαστος και πολυ ευγενικος. 
Πολυ ωραιος τυπος!

\photo{119.jpg}

Η πενταμορφη και το τερας η ``πως να χαλασετε μια ωραιοτατη φωτογραφιας μοτοσυκλετας''!

\photo{120.jpg}

Κοιταζε με λαχταρα τη μηχανη και με ρωτουσε για το ταξιδι που ειχα μπροστα μου, ομως εγω ηθελα να του πω οτι το ``ταξιδι'' που εκεινος ειχε ξεκινησει ηταν πολυ καλυτερο απο το δικο μου. Xαμογελασα και δεν ειπα τιποτα. Καποια πραγματα απλα δεν λεγονται με λογια.

Θα ηθελα να καθομουν περισσοτερο να τα λεγαμε αλλα ο χρονος οσο και να μην ηθελα πιεζε. 
``Κριμα ρε φιλε!'' -το ειδα οτι στεναχωρηθηκε αλλα ηξερα οτι καταλαβαινε. Βαρυ πραγμα η ξενιτια, ακομα και για λιγο καιρο. Το ειχα νοιωσει στο πετσι μου καλα τα χρονια που ζουσα στο βροχονησο...

Απο πανω μας ο ουρανος ειχε μαζεψει συννεφα αλλα κρατουσε ακομα. Λες να ημασταν τοσο τυχεροι μεχρι το Leverkusen; Ο Κωστης στο τηλεφωνο μου εκοψε τα ποδια: ``Βαλε αδιαβροχα. Εδω βρεχει ασταματητα.''

\photo{121.jpg}

Ο Σταυρος με συνοδεψε μεχρι την εξοδο της πολης και αποχαιρετηθηκαμε. Βγηκα στην εθνικη και ανοιξα το γκαζι. Τα απειλητικα γκριζα συννεφα μπροστα εδειχναν τις διαθεσεις του καιρου...

\photo{122.jpg}

Το φως χανοταν πλεον και η βροχη αρχισε να πεφτει ανελεητη. Κατεβασα κεφαλι και εσφιξα το δεξι γκριπ -ηταν ωρα να σοβαρεψουν τα πραγματα. 
Περνουσα μεσα απο τη καταιγιδα με 160+. Η Αυρα δεν καταλαβαινε τιποτα: παρατεταμενες στροφες με 140 και ευθειες με κλειδωμενο γκαζι -απολυτη σταθεροτητα σαν να ηταν το πιο φυσικο πραγμα του κοσμου.

Δεν μπορουσα ομως να πω το ιδιο και για τον αναβατη της.... Η διαδρομη ηταν δυσκολη. Πισω απο τη βρεγμενη και θολη ζελατινα προσπαθουσα να δω μεσα στο σκοταδι, ενω το κρυο πλεον να ειχε αρχισει να γινεται πολυ ενοχλητικο παρα τα αδιαβροχα και τη στολη. Ειχα ηδη κανει 350 χιλιομετρα απο το μεσημερι και αρχισα να νοιωθω τη κουραση της ημερας βαρια πανω μου. 
Σκυφτος πισω απο τα φερινγκ, εβλεπα τα χιλιομετρα στο οδομετρο να περνανε ενα ενα και δυσκολα και το γκαζι ανοιγε ακομα πιο πολυ για να φτασω συντομα. Ευτυχως αν υπηρχε ενα μερος στο κοσμο να πηγαινω ετσι μεσα στη βροχη ηταν εδω στην autobahn.

\photo{123.jpg}

Η πινακιδα που εγραφε Leverkusen ηρθε σαν σανιδα σωτηριας. Ειχα φτασει. 

\photo{124.jpg}

Στο σπιτι ο Κωστας και η γλυκυτατη Σαρα με υποδεχτηκαν με μεγαλη χαρα. 
Δεν θα μπορουσα να ζητησω κατι καλυτερο για το παγωμενο και βρεγμενο τομαρι μου: δυο ζεστα χαμογελα και μια στεγη πανω απο το κεφαλι μου αυτη τη κρυα νυχτα.
Κατσαμε να φαμε και το βραδυ περασε με κουβεντες για τα παντα: για τα πραγματα στην Ελλαδα, στη Γερμανια, για ταξιδια περασμενα και επομενα, για ονειρα και στοχους... 

Εγω ομως ενοιωθα πραγματικα πολυ χαρουμενος και γιατι επιτελους συναντουσα το Κωστα απο κοντα! Ηταν κατι που ηθελα να γινει εδω και χρονια γιατι πραγματικα με ειχε σκλαβωσει με την καλοσυνη του, την ευγενεια και το χαρακτηρα του. 

Βλεπετε στα τελη του 2008 εψαχνα να αγορασω VFR. Τα ελαχιστα που ειχα βρει στην Ελλαδα ηταν για πεταμα οποτε πλεον κοιτουσα απο Γερμανια. Ειχα βρει μια σε αριστη κατασταση σε μια αντιπροσωπεια Honda στη κεντρικη Γερμανια περιπου μια ωρα ανατολικα του Leverkusen, ομως δεν μπορουσα να παω να τη δω απο κοντα. 
Ετσι ενας φιλος προσφερθηκε να παει αυτος 200 χιλιομετρα πηγαινελα να τη δει και να τη δοκιμασει για παρτη μου, οπως και εγινε τελικα. 
Αυτος ο φιλος ηταν ο Κωστας και αυτη η VFR ηταν η Αυρα μου...

Και να που τωρα ημασταν εδω σε ενα ομορφο σπιτι συνοδεια μπυριτσας και τσιμπολογηματος να τα λεμε ενω εξω η βροχα επιπτε ραιτ θρου...
Ομορφα...
