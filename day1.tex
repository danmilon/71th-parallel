\chapter{Day 1 -- Ancona (IT) - Como (IT) -- 492km}

Η ωρα πανω στα πλοια περναει αργα. Οσοι εχουν ταξιδεψει ετσι ξερουν. Κοιταξα το ρολοι μου. Με τη διαφορα ωρας θα επρεπε να κοντευουμε στο λιμανι. Ομως τριγυρω το μονο που βλεπω ειναι το απειρο μπλε της θαλασσας. \textit{Τι ωρα φτανουμε;} Η απαντηση απο τον λογιστη του πλοιου με ξαφνιασε: \textit{Μια το μεσημερι;!} Εγω πως νομιζα οτι φτανουμε 10.30 το πρωι; Η ιδεα μου να χαζεψω στους επαρχιακους δρομους της κεντρικης Ιταλιας τωρα πηγαινε απατη...

Η ωρα τελικα περασε χαζευοντας τους χαρτες για να δω τις διαδρομες που θα ακολουθησω για το προορισμο μου. Τι πιο ομορφο πραγμα απο αυτο; Να ξεκινας τη μερα σου χαζευοντας ολες αυτες τις μικρες πολυχρωμες γραμμουλες πανω στους χαρτες και να προγραμματιζεις αγνωστες διαδρομες για να νυχτωθεις οπου σε βγαλει ο δρομος...

Εγω ομως σημερα ειχα προορισμο: τη λιμνη Como, τη τριτη μεγαλυτερη λιμνη της Ιταλιας στα βορειοδυτικα της χωρας, ακριβως πανω στα Ιταλο-ελβετικα συνορα. Εκει ειναι και το ομωνυμο φημισμενο χωριο που αποτελει ενα (αρκετα κυριλε/ακριβο) τουριστικο θερετρο για τους Ευρωπαιους και οχι μονο, καθως ειναι και το μερος που ζει ο ...George εδω και χρονια. (Μη με ρωτησετε ποιος ειναι ο George ετσι;) Ηθελα να επισκευτω την περιοχη εδω και χρονια και να που ειχε ερθει η ευκαιρια να το κανω. Βεβαια με το πλοιο να δενει τοσο αργα στην Ancona και εχοντας να διασχισω ουσιαστικα ολη την Ιταλια δεν θα εβλεπα και πολλα απο το Como αλλα δεν πειραζει. Ακομα και ετσι θα επαιρνα μια μικρη γευση!

\photo{6.jpg}

Το πλοιο ειχε πιασει να μπαινει στο λιμανι της Ancona και βρηκα την ευκαιρια να χαζεψω τριγυρω...
Δεν θα το ελεγα ακριβως ωραιο το μερος. Ενα μικρο, μαλλον αδιαφορο λιμανακι ηταν αλλα ακομα και ετσι μερικα κτηρια διατηρουσαν την περιφημη Ιταλικη φινετσα και ομορφια...

\photo{7.jpg}

Κατεβηκα στο γκαραζ για να ετοιμαστω. Οσοι με ξερουν γνωριζουν οτι με το χρονο εχω μια πολυ περιεργη σχεση: μια ζωη αργω σε οτι και να κανω, και αυτη η φορα ΔΕΝ ηταν η εξαιρεση.

Τα παιδια με το Multistrada που ηταν παρκαρισμενο διπλα μου ειχαν ηδη κατεβει κατω και περιμεναν υπομονετικα να μαζεψω το τσαντιρι για να φυγουν. Οι παρκαδοροι του πλοιου θελοντας να διευκολυνουν την κατασταση επιασαν να λυνουν τους ιμαντες απο τις μηχανες για να φυγουμε μια ωρα αρχιτερα.. Ποσο μα ποσο θα το μετανοιωνα αυτο πολυ συντομα.... Για να βοηθησω με τη σειρα μου το ζευγαρι να φυγει μετακινησα την Αυρα 2 μετρα πιο πισω. Ομως εκει που την πηγα το πατωμα ειχε ενα μικρο εξογκωμα. 
Οχι και τοσο σημαντικο θα μου πειτε. Θα συμφωνουσα αν δεν υπηρχαν δυο μικρες λεπτομερειες:

\smallskip
\begin{enumerate}
\item Η μηχανη λογω βαρους απο τις βαλιτσες (και λογω εγκεφαλικης βλακειας δικης μου που ΔΕΝ εσφιξα την προφορτιση πριν το ταξιδι) εκανε την αναρτηση να βυθιζεται ελαφρα, ετσι ωστε οταν καθοταν στο stand πλαγιαζε απο λιγο εως ελαχιστα.
\item Οι ιμαντες δεσιματος ειχαν λυθει εντελως.
\end{enumerate}
\smallskip

Ετσι εβαλα το stand αφου ειχα κανει πισω να φυγουν τα παιδια και πηγα πισω στη βαλιτσα να κλεισω απλα το καπακι. Καπου εκει εκανα την ενδιαφερουσα παρατηρηση οτι η μηχανη στηριζοταν σε οριακα ορθια θεση και καθως ακουμπαω το καπακι της βαλιτσας.... ο Τιτανικος βυθιζεται! 340+ κιλα πλαστικων και μεταλλων εφυγαν προς τα δεξια με χαρη που θα ζηλευε και ο ελεφαντας του Circo Medrano. Προφανως οι προσπαθειες να σταματησω το θηριο τραβωντας απο αριστερα ηταν εντελως αναξιες λογου και ετσι βρεθηκα να κοιταω τη μηχανη φαρδια πλατια στο καταστρωμα του πλοιου!

Τι πιο ωραια αρχη για το μεγαλυτερο ταξιδι που θα εκανα ποτε; Τελικα αν δεν πεσει η μηχανη μια φορα σε ενα ταξιδι μου δεν θα παει καλα. Το 2010 ειχε συμβει το ιδιο ακριβως 2 φορες αλλα το ταξιδι ειχε παει υπεροχα. Απο την αλλη να πω τωρα οτι χαιρομουν θα ημουν ψευτης... Σηκωσαμε τη μηχανη πανω μαζι με ενα νταλικιερη που προθυμοποιηθηκε και ...ευτυχως σχεδον καμια ζημια! Η μηχανη ειχε κατσει πανω στη δεξια βαλιτσα σωζοντας τα χειροτερα και το μονο αλλο που ειχε βρει ηταν ενα πολυ μικρο πλαστικο στο δεξι φλας που γεμισε γρατζουνιες. Παλι καλα!

Αυτο που ειχε παθει ομως καλη ζημια ηταν το μεγαλο δαχτυλο μου στο δεξι χερι που πρηστηκε σχεδον αμεσως και δεν μπορουσε να κλεισει. Κακωση; Καταγμα; Το σιγουρο ηταν οτι ποναγε διαολεμενα. Πως θα ταξιδευα ετσι 12.000 χιλιομετρα; Αν ηξερα ομως τι θα ακολουθουσε μεχρι να τελειωσει αυτη η απιστευτη περιπετεια μαλλον θα το προσπερνουσα αυτο ως κατι αναξιο λογου βρε αδερφε...

Καθως ομως η Αυρα ρολαριζε τις ροδες της για μια ακομα φορα επι Ιταλικου εδαφους τα ειχα ξεχασει ολα και χαμογελουσα σαν χαζος...

\photo{8.jpg}

Σταματησα στο πρωτο βενζιναδικο που βρηκα για να ταισω τα αλογα και να αποφασισω για τη διαδρομη. Ηθελα πολυ να βγω στους επαρχιακους δρομους αλλα το GPS μου εκοψε τα ποδια: 10+ ωρες συνεχους οδηγησης και η ωρα ηταν ηδη 2 το μεσημερι. Στην καλυτερη των περιπτωσεων και χωρις καμια σταση (κατι πρακτικα αδυνατον) θα εφτανα τις 12 το βραδυ και βεβαια χωρις να εχω κανονισει καπου να μεινω δεν ηταν και η καλυτερη ιδεα να ψαχνω μεσα στη μαυρη νυχτα καπου να κοιμηθω...

Συνεπως autostrada και ξερο ψωμι για σημερα δυστυχως.
Και το δυστυχως οχι γιατι ειναι κανενας παλιοδρομος (αν και δεν μπορει φυσικα να συγκριθει με το ΥΠΕΡΤΑΤΟ μεγαλειο της Κορινθου - Πατρων) αλλα γιατι εγω βαριεμαι τρομερα τις εθνικες οδους και τις ευθειες και τις αποφευγω οπως ο Lemmy τη Britney Spears. Αναγκαιο ομως κακο σημερα η εθνικη αν ηθελα να φτασω σε καποια λογικη ωρα στο προορισμο μου. Περα απο τη πλακα βεβαια η autostrada ειναι δρομαρα και αυτο δεν κρυβοταν με τιποτα...

\photo{9.jpg}

...ενω και οι εικονες της επαρχιακης ζωης τριγυρω εκαναν οτι μπορουσαν για να μην πληττω.

\photo{10.jpg}

Συντομα επιασα ρυθμο και η μηχανη αρχισε να τρωει τα χιλιομετρα γουργουριζοντας χαρουμενη, κατι που δεν μπορουσα να πω και για εμενα. Τρελη ζεστη, ηλιος και δερματινα δεν ειναι και ο καλυτερος συνδιασμος. Οι στασεις για νερο ηταν συνεχεις και παροτι ανεβαινα ολοενα και πιο βορεια η καψα του μεσογειακου καλοκαιριου δεν ελεγε να κοπασει παρα τα συννεφα που εβλεπα να μαζευονται πλεον στον οριζοντα. 
Τωρα ειχε και ζεστη και μουνταδα!

\photo{11.jpg}

Βγηκα απο την autostrada και προετοιμαστηκα για τις πραξεις ΧΧΧ που θα μου εκαναν στα διοδια: 27 ευρω για 400+ χιλιομετρα; Χμμμ! Περιμενα χειροτερα.
Το απογευμα ειχε πιασει να πεφτει απο ωρα και ανηφοριζα πλεον προς το Como. Ειχε βρεξει εδω πριν λιγο... Ο ουρανος γκριζος πλεον με βαρια συννεφα -θα προτιμουσα λιακαδα, αλλα οι μυρωδιες της φυσης και η ατμοσφαιρικη εικονα που εφτιαχνε ο ουρανος ηταν εξισου ομορφες...

Καθως κατεβαινα απο την εξοδο του δρομου προς τη λιμνη, το Como αρχισε να αποκαλυπτει το λογο γιατι θεωρειται ενας απο τους πιο ομορφους καλοκαιρινους προορισμους στην Ευρωπη. Εκανα μια σταση στην ακρη του δρομου και τα ματια δεν μπορουσαν να χορτασουν τις εικονες... Εικονες μιας ταινιας προσεχως...

\photo{12.jpg}
\photo{13.jpg}

To τοπιο ηταν απλα μαγευτικο! Με καποιο παραξενο τροπο η μουνταδα του ουρανου ταιριαζε απολυτα στη στιγμη εκεινη και εγω καθισα σιωπηλος χαμενος στη στιγμη...

\photo{14.jpg}

Θα ηθελα να κανω το γυρο της λιμνης αλλα η ωρα δεν το επετρεπε. Η νυχτα επεφτε πλεον και εγω ακομα δεν ειχα βρει που θα εμενα το βραδυ. Παρολαυτα, η διαδρομη που ξεκινουσε γυρω απο το Como εδειχνε πολλα υποσχομενη και στο μυαλο μου ανανεωσα το ραντεβου μου με την περιοχη για καποια αλλη φορα στο μελλον. Τελικα αν κατσω και βαλω ενα σημαιακι στα μερη που εχω αφησει στα ``προσεχως'' στο τελος δεν θα βλεπω χαρτη για χαρτη!

\photo{15.jpg}
\photo{16.jpg}

Το GPS εδωσε τη λυση για καποιο κοντινο camping: Camping Europa, ενα συμπαθητικο καμπινγκ ακριβως στην εξοδο του χωριου.

\photo{17.jpg}

Δυστυχως ομως η ρεσεψιον οσον αφορουσε την εξυπηρετηση και τη συννενοηση βρισκοταν στο απολυτο μηδεν: ενας βαριεστημενος πιτσιρικας (που ειμαι σιγουρος οτι θα προτιμουσε να ειναι οπουδηποτε αλλου εκεινη την ωρα) μετα βιας εδειχνε ενδιαφερον να εξυπηρετησει ενα ταλαιπωρημενο μηχανοβιο που στεκοταν μπροστα του.
Με τα ελαχιστα ιταλικα μου ευτυχως συνεννοηθηκαμε και μετα τις τυπικες διαδικασιες ξεκινησα να στησω τα συμπραγκ... Επ! Οχι τοσο γρηγορα φιλε μου! Ο καιρος δεν ειχε πει την τελευταια του λεξη ακομα...
Ισα που ειχα προλαβει να απλωσω το εσωτερικο της σκηνης, οταν αρχισαν να πεφτουν δυνατοι κεραυνοι πανω απο τη λιμνη και η ατμοσφαιρα γεμισε τη γνωριμη μυρωδια του οζοντος -σημαδι οτι εντος δευτερολεπτων θα ξεκινουσε ο κατακλυσμος του Νωε!
Τα πυκνα φυλλωματα των δεντρων απο πανω μου θα κρατουσαν μια μικρη βροχουλα αλλα κατι μου ελεγε οτι αυτο που ερχοταν μονο μικρο δεν θα το ελεγες... 

Μαζεψα αρον αρον τη σκηνη και ετρεξα πισω στη ρεσεψιον ενω η βροχη αρχισε να δυναμωνει. 
Καλυβακι; Λαστ γιαρ! Και τωρα τι κανουμε; Να στησω υπο βροχη ηταν κατι που θα προτιμουσα να αποφυγω...
Εν μεσω βροχης το πιτσιρικι μου προτεινε ενα τροχοσπιτο. Πως ειπατε; 
Το καμπινγκ ειχε μερικα παναρχαια, ταλαιπωρημενα τροχοσπιτα στημενα εκει μονιμα ως λυση αναγκης για οσους ηθελαν καλυβακι αλλα δεν εβρισκαν.
Μου εδωσε ενα κλειδι και μου ειπε το νουμερο που επρεπε να βρω. Που ειναι ρε παιδια το νουμερο 12; Δεν υπηρχε πουθενα! Μετα απο τρια τεσσερα πανω κατω στη ρεσεψιον φορωντας τα δερματινα και ιδρωνοντας σαν γουρουνι στο σακι ΠΑΡΑ τη βροχη που επεφτε τελικα καταφερα να εντοπισω το περιφημο τροχοσπιτο. 

Κατι που τελικα μακαρι να μην το ειχα κανει και ποτε! Ολη μου τη ζωη την εχω περασει σε καμπινγκ -ε λοιπον δεν εχω δει ΠΟΤΕ αλλοτε ενα τροχοσπιτο σε τοσο αθλια κατασταση. Τα παντα ηταν σκουριασμενα και σπασμενα. Πομολα, μεντεσεδες στις πορτες, μπρατσα στηριξης στα παραθυρα... Το μεγαλυτερο μερος του τροχοσπιτου εκτελουσε χρεη (σκουπιδ)αποθηκης με οτι σαβουρα μπορει να φανταστει κανεις πεταμενη μεσα ενω το κρεβατι ειχε επανω κατι τεραστιους λεκεδες που θα προτιμουσα να μην ξερω απο τι εγιναν και μια εκλεκτη κολεξιον απο ``Πεθαμενα ζωυφια αγνωστου ταυτοτητας'' του 2012. 
Σαν να μην εφταναν αυτα, το τροχοσπιτο εδειχνε να ...κατοικειται ηδη! Στο τοιχο υπηρχαν κατι ρωσικα εικονισματα και σε μια καρεκλα κατι παλια ρουχα και παντοφλες καποιου αντρα.

Μεσα σε χρονο μηδεν ημουν και παλι στη ρεσεψιον: αστο φιλε, θα το ρισκαρω με τη καταιγιδα χιλιες φορες παρα αυτο το πραγμα!
Εξαλλου με ολα αυτα η μπορα ειχε σταματησει πλεον οποτε χιλιες φορες η σκηνη μου και η φυση παρα να κλειστω σε ενα αθλιο κουτι.
Αραξα τη μηχανη σε ενα ωραιο σημειο που δεν ειχε βραχει πολυ και αρχισα να στηνω. 
Λιγα μετρα πιο περα δυο αλλες ταξιδιαρες μηχανες ηταν αραγμενες: ενα RT1100 και ενα GS1150, που αν εκρινα απο τα αυτοκολλητα και τη κατασταση τους μαλλον ειχαν να πουν πολλες ιστοριες για ταξιδια. 
Βρετανικες πινακιδες με Ουαλλικα διακριτικα! Συντοπιτες.

Καθως το βραδυ ειχε πεσει πλεον το τσαρδι ηταν ετοιμο και ειχα ηδη πιασει να μαγειρεψω κατι για δειπνο...
Καπου εκει ηρθαν και οι Ουαλλοι και κατσαμε να τα πουμε υπο τη συνοδεια παγωμενης μπυριτσας. 
Πολυ ωραιοι τυποι! Παλιοσειρες μηχανοβιοι ταξιδιωτες που ειχαν φαει την Ευρωπη με το κουταλι. Ειχαν ξεκινησει απο Αγγλια πριν μια βδομαδα και αυριο θα ξεκινουσαν να δουν τα πασα της Ελβετιας οπως και εγω! Οι ιστοριες και οι διαδρομες στους χαρτες εδιναν και επαιρναν μεχρι αργα το βραδυ...

Κρατηστε τα πιο κυριλε εστιατορια και τα πιο γκουρμε φαγητα σας. Εγω ετσι τωρα ημουν ο πιο ευτυχισμενος ανθρωπος του κοσμου. Μια σκηνη, η μηχανη διπλα, οι μποτες μου, αναπαντεχες συναντησεις και ενα ζεστο πιατο φαι στο τελος της μερας. Παραδεισος! 
Αυριο ξημερωνε μια νεα μερα...

\photo{18.jpg}
