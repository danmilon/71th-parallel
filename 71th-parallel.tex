\documentclass[11pt, letterpaper]{book}
\usepackage{fontspec}
\usepackage{hyperref}
\usepackage{graphicx}

\usepackage{xgreek}    % greek hyphenation
\usepackage{titlepic}  % picture at title


\hypersetup{pdfborder=0 0 0, colorlinks=true, linkcolor=red}
\setmainfont[Ligatures=TeX]{Linux Libertine}

\newcommand\dialogue[1]{\par\noindent--~\textit{#1}}

\newcommand\photo[1]{\begin{center}\noindent\includegraphics[width=0.9\textwidth]{photos/#1}\end{center}}

\titlepic{\includegraphics[width=\textwidth]{photos/front.jpg}}
\title{72ος Παράλληλος: Η μοναξιά του αναβάτη}
\author{Ταξιδευτής}

\begin{document}
\maketitle
\tableofcontents

\chapter{Εισαγωγή}

Αυτο το ταξιδι το χρωσταω σε ολους τους υπεροχους ανθρωπους που συναντησα καθ οδον, σε φιλους παροντες και εκεινους που δεν ειναι πια εδω... Christie, Κωστα, Γιαννη, Ion, Christian, Asa, Οle, Marko, Oili, Annika, Marek, Αlex, Dragan, Πετρο, Θαλεια, Κωστη, Μαθο, Πανο, Μανωλη, Γωγω... Χωρις εσας πολυ απλα δεν θα υπηρχε το ταξιδι τουτο και γι' αυτο και σας ανηκει.

Στο ραφι απεναντι μου, εκει, αναμεσα σε βιβλια παρτιτουρες και καθε λογης ματζαλα, εχω ενα σημειωματαριο με τους χαρτες απο το ταξιδι, γεματο με σημειωσεις για σκεψεις, εικονες, συναισθηματα. Ειναι καιρος τωρα που προσπαθω να το αποφυγω το ατιμο. Το εχω αφησει εκει και κανω οτι δεν το βλεπω ομως νοιωθω να με κοιταει απο απεναντι επικριτικα σαν να μου λεει ``Αντε! Ξεκινα! Γραψε!''

Ειναι αληθεια οτι καθε φορα που καθομαι να γραψω για ενα ταξιδι δυσκολευομαι. Αυτη τη φορα ομως ειναι διαφορετικα. Δεν ξερω....
Δεν ξερω αν μπορω να γραψω αρκετα καλα ωστε να μεταφερω το τι εζησα. Νοιωθω λιγος αυτη τη φορα γαμωτο!
Γραφω, σβηνω και ξαναγραφω, ξανασβηνω και τα παραταω. Γιατι ειναι τοσο δυσκολο να παρει η ευχη;

Αυτη η φωνη μεσα στο μυαλο δεν λεει να σωπασει.

\dialogue{Ξερεις το λογο Νικο.}
\dialogue{Σκασε.}
\dialogue{Ξερεις τι πρεπει να κανεις για να γραψεις.}
\dialogue{Ασε με!}
\dialogue{Για να γραψεις για αυτο το ταξιδι πρεπει να μπορεσεις να τα πεις ολα. Να παραδεχτεις αληθειες δυσκολες, να αντιμετωπισεις φαντασματα, να μηδενισεις κοντερ. Αντεχεις ρε φιλε; Εχεις τα αντερα να μην κρυφτεις απο τον ιδιο σου τον εαυτο; Εδω σε θελω....}

Ναι λοιπον ειναι καιρος να γραψω γιατι ολα αυτα μεσα δεν χωρουν πια...
Δεν γινεται να το αποφευγω πια... Στο ποτηρι εχει απο ωρα μπει ενα καλο single malt και απ' τα ηχεια η μοναδικη φωνη του Sivert Hoyem ερχεται να βαλει φωτια στα συναισθηματα, να πυρωσει μνημες και εικονες. Το soundtrack του ταξιδιου αυτου δεν θα μπορουσε να ξεκιναει αλλιως.... \href{http://goo.gl/xUsd7}{Sivert Hoyem - Give it a whirl}

\begin{verse}
\textit{It's my time to be waiting\\
Locked down, chained in the hole\\
But when my waiting is done I'm gonna give it a whirl...}\\
\end{verse}

Αυτο το ταξιδι ηταν οτι πιο δυσκολο εχω ζησει. Γιατι ομως; Ουτε ο πρωτος ουτε ο τελευταιος που πηγε στο Βορειο Ακρωτηρι ημουν.
Αλλοι πηγανε και περασαν ζαχαρη και ολα καλα και ωραια και χαλαρα και σουπερ γουαου.
Οχι εγω. Μπορει να ηταν Ονειρο ζωης, μια μοναδικη εμπειρια που θα κουβαλαω μεχρι να χτυπησω καρτα απο το ματαιο τουτο κοσμο, να ηταν κατι που χρωστουσα σε εμενα και σε αδερφια που δεν μπορεσαν να πανε ποτε, αλλα ευκολο ΔΕΝ ηταν...
Σωματικα, νοητικα, ψυχικα. Το ταξιδι αυτο ζητησε τα παντα και ακομα περισσοτερα. Καθε οριο, καθε κολλημα, καθε φοβια, καθε νοοτροπια. Οσο το αγαπησα τοσο το μισησα και οσο το θελησα αλλο τοσο το μετανοιωσα. Ψαχνετε να βρειτε λογικη; Διαβαστε κατι αλλο καλυτερα γιατι εδω δεν παιζει να τη βρειτε...

Αγκαλιασα το αγνωστο γιατι πολυ απλα δεν ειχα αλλες επιλογες και ειχα το χαμογελο του παλαβου καθως ο Μερφυ εκανε οτι μπορουσε για να μου χαλασει τη διαθεση.
Ειδα μερη αποκοσμα που δεν μπορουσα να φανταστω οτι υπαρχουν σε αυτο το πλανητη.
Εζησα καταστασεις τρελες και παλεψα με καθε ειδους συνθηκες. Απο τον καυσωνα των 42 βαθμων στο πολικο ψυχος του αρκτικου κυκλου, με ηλιο, με βροχη, αερα, χιονι και οτι αλλο μπορει να ριξει η φυση πανω σε ενα ταλαιπωρο μηχανοβιο ταξιδιωτη.
Ατελειωτα χιλιομετρα, στο ελεος και τις διαθεσεις του καιρου, κατω απο ενα βαρυ ουρανο, με συντροφο τη μηχανη μου... Εκει ειναι που κανεις φιλη τη μοναξια δικε μου γιατι αλλιως εισαι χαμενος απο χερι.

Ποσο ειρωνικο ομως ειναι οτι αναμετρηθηκα με οτι μπορουσα να φανταστω, για να ανακαλυψω οτι ο μεγαλυτερος εχθρος μου απο την αρχη ηταν ενας: Εγω.
Στο ταξιδι αυτο ομως τον εκανα μαγκα: τον γκρεμισα κομματι κομματι για να ανακαλυψω αυτο που κρυβοταν πισω απο τη σκουρια και τη μπιχλα της καθημερινοτητας και του βολεματος... Εγινα κατι αλλο απο αυτο που ημουν. Αυτο το ταξιδι σε αλλαζει βαθεια. Ειναι βασανιστικο, σκληρο αλλα τελικα λυτρωτικο...
Στο καθενα δινει την απαντηση σε οσα αναζητα. Σε εμενα εδωσε την απαντηση που γυρευα να βρω, και ας ηταν δυσκολη να την παραδεχτω...

Ταξιδευοντας εμαθα να εμπιστευομαι τους ανθρωπους που συνανταω, να ακολουθω το ενστικτο μου και πανω απο ολα να μη φοβαμαι κανεναν και τιποτα στο κοσμο. Παντα στα ταξιδια γυριζα λιγο διαφορετικος απο οτι εφευγα. 
Ομως αυτη τη φορα κατι αλλαξε οριστικα μεσα μου...
Μετα απο αυτο το ταξιδι.... δεν ξερω... νοιωθω να μπορω να κανω τα παντα και τιποτα πια...

Αν ψαχνετε ενα ταξιδιωτικο με λυρικες περιγραφες για τα φοβερα μερη που ειδαμε, για το ποσο τελεια ηταν ολα και το ποσο ωραια περασαμε, καλυτερα να σταματησετε να διαβαζετε εδω. Αν μη τι αλλο, υπαρχουν ενα σωρο αλλα ταξιδιωτικα εκει εξω απο παιδια που εχουν κανει αυτο το ταξιδι και τα μπορουν να τα πουνε πολυ καλυτερα απο εμενα ολα αυτα.

Αυτη εδω η ιστορια ειναι απλα οι σκεψεις ενος μονου αναβατη που προσπαθησε να κανει πραγματικοτητα ενα μεγαλο ονειρο -εικονες, συναισθηματα, νοηματα, ολα ατακτως ερριμενα.
Μη με παρεξηγησετε... Ισως βαζοντας τα στο χαρτι να βρω μια ακρη και να ξετυλιξω το κουβαρι...
Τι; Νομιζατε οτι θα τη γλυτωσετε ετσι απλα; Γελαστηκατε! Θα παρει χρονο το ταξιδιωτικο αυτο και δεν θα ειναι ευκολο.
Θα βγει με κοπο αλλα ισως και να αξιζε στο τελος -οπως εγινε και με το ιδιο το ταξιδι.

Σχολια, ευχες, παρατηρησεις, μπινελικια, ευχες και καταρες, ριψατε εντος. Ολα καλοδεχουμενα, σημαντικα, απολυτως απαραιτητα τελικα.

``There was nowhere to go but everywhere, so just keep on rolling under the stars.''

\photo{1.jpg}

\chapter{Day 0 -- Αθήνα (GR) - Ancona (IT)}

Το ταξιδι αυτο ηταν κατι που υπηρχε παντα μεσα στο μυαλο μου. Απο τοτε που θυμαμαι τον εαυτο μου σε δυο ροδες, παντα αναφερομουν στο ταξιδι στο Βορειο Ακρωτηρι σαν Το Ταξιδι. Eνα απο αυτα τα ονειρα που πρεπει να κανει καποιος πραγματικοτητα πριν κλεισει τα ματια.

Η φωτια σιγοκαιγε χρονια μεσα στο μυαλο αλλα καθε φορα που πηγαινε να φουντωσει την εσβηνα λεγοντας ``αστο μωρε, εχουμε χρονο, τωρα εξαλλου εχουμε το Χ, το Ψ, το Ω''. Καθε φορα μια δικαιολογια, καθε φορα μια αναβολη. Ωσπου ηρθε εκεινο το τηλεφωνημα εκεινη την ρημαδα Κυριακη του Σεπτεμβριου απο τον κολλητο.

\dialogue{Ελα ρε Νικο, που εισαι ρε και σε ψαχνω;}
\dialogue{Ελα φιλε, ειμαι σε κατι συγγενικες υποχρεωσεις. Τι εγινε;}
\dialogue{Δεν τα εμαθες; Ο Πανος ρε... Εφυγε...! Συνορα Ουκρανιας με ενα διερχομενο φορτηγο.}
\dialogue{........}

Ο καλος μου φιλος ο Πανος... Αδερφος ταξιδευτης και αυτος ειχε φαει τα Βαλκανια και την Ευρωπη με το κουταλι, και πλεον εγραφε ροτες για νοτια. Με εκεινον ειχα ξεκινησει το πρωτο μεγαλο μου ταξιδι στην Ευρωπη το 2007, με εκεινος ειχα κανει ενα σωρο βολτες και ταξιδακια οσο βρισκοταν Ελλαδα πριν μετακομισει Βουλγαρια. Μια μερα πριν μιλουσαμε γαμωτο! Και τωρα....

Εκει κατι αλλαξε μεσα μου. Ο χρονος που πριν ηταν απλετος τωρα ειχε τελειωσει. Οι δικαιολογιες, οι αναβολες εγιναν χιλια κομματια με τον πιο βιαιο τροπο. Τωρα. ΤΩΡΑ.
Μην αναβαλλεις δικε μου το Ονειρο γιατι μια μερα θα ειναι πολυ αργα. Παντα υπαρχει τροπος, αλλα ειναι πιο ευκολο να βρισκεις δικαιολογιες για να λες ``δεν μπορω''... Τα ονειρα ζητανε βλεπεις πολλα: θυσιες, κοπο και αιμα που βραζει. Πως να το κανεις οταν εισαι βολεμενος στο τακτοποιημενο, καθαρο, αποστειρωμενο κουτακι της ζωης σου; Το κουτι μου ομως εγω το ειχα σπασει και δεν κοιταζα πισω πια.

Γυρισα σπιτι αμιλητος, μουδιασμενος. Ανοιξα τους παλιους χαρτες γιατι ξερω οτι αυτο λατρευε και ξεκινησα να σχεδιαζω ροτες... Το ταξιδι ειχε ξεκινησει ηδη. Ενα ταξιδι που δεν προλαβε γαμωτο αλλα θα το καναμε μαζι.

Δυσκολοι μηνες περασαν απο τοτε, ομως το ονειρο δεν ξεχαστηκε. Δεν μιλησα σε κανεναν. Θα το εκανα μονο οταν ολα θα ηταν πλεον ετοιμα. Ηταν τοσα που θα μπορουσαν να πανε στραβα. Δουλεια, αδεια, χρονος, χρημα, διαθεση, μηχανη και αλλα τοσα απροβλεπτα και ξαφνικα.

Σιγα σιγα ξεκινησαν ετοιμασιες. Ο εξοπλισμος εκστρατειας ηταν απο πριν απολυτα ικανος να αντεξει ενα τετοιο ταξιδι, οποτε απλα αρχισα να μαζευω τα λιγα που μου ελειπαν: ενα καλο αδιαβροχο, ολοσωμο ισοθερμικο, νεα δερματινα, ενα κουζινακι για το μαγειρεμα... Λιγο πριν το καλοκαιρι η αδεια κανονιστηκε, οι διαδρομες αρχισαν να βγαινουν, το προγραμμα κανονιστηκε, τα πλοια εκλεισαν. Οταν ελαβα το πρωτο confirmation απο το πλοιο που θα με πηγαινε απο Δανια - Νορβηγια ενοιωσα μαγικα: το ονειρο επαιρνε πλεον σαρκα και οστα! Θα το εκανα!

Οι ημερομηνιες που επελεξα συγκεκριμενες: απο τα μεσα Ιουλιου μεχρι αρχες Αυγουστου. Ο λογος απλος: εκει πανω ο καιρος απο τον Αυγουστο και μετα αγριευει και θελει προσοχη. Η καλυτερη περιοδος για να επισκευτει καποιος αυτες τις χωρες ειναι απο τελη Ιουνιου μεχρι μεσα Αυγουστου. Χα! Ετσι ελεγαν.... Καθως οι μερες περνουσαν και εφτανε ο καιρος, το αγχος για το τι πηγαινα να κανω ολοενα και μεγαλωνε. Το να λες ``παω Βορειο Ακρωτηρι'' ειναι τρεις απλες λεξουλες, αλλα οταν το κανεις πραγματικα, εκει δικε μου ειναι ΕΝΤΕΛΩΣ αλλη ιστορια...

Η τελευταια βδομαδα πριν το ταξιδι περασε μεσα σε ενα πανικο δουλειας, πιεσης να τα προλαβω ολα και αγχους για να μην ξεχασω τιποτα.

Πεμπτη μεσημερι στο γραφειο και κατι περισσοτερο απο 24 ωρες για την αναχωρηση! Παρασκευη βραδυ θα επαιρνα το πλοιο απο Ηγουμενιτσα για Ανκονα. Ολα ηταν ετοιμα. Μεχρι που χτυπησε το τηλεφωνο:\\

\dialogue{Ναι γεια σας, τηλεφωνω απο την Greek Ferries. Eχετε κλεισει ενα εισιτηριο απο Ηγουμενιτσα για Ανκονα για αυριο το βραδυ;}
\dialogue{Ναι.}
\dialogue{Θελουμε να σας ενημερωσουμε οτι το πλοιο ακυρωνεται!}\\


\noindent Αφου συνηλθα απο το μινι εγκεφαλικο εγινε χαμος:\\

\dialogue{Μα τι μου λετε;;;;! Αυριο φευγω για 23 μερες ταξιδι και ολα ειναι κλεισμενα ηδη! Και μου λετε μια μερα πριν οτι ακυρωνεται το πλοιο;}
\dialogue{Ε ναι, μας ενημερωσαν απο την Superfast οτι το πλοιο αυτο πλεον δεν θα κανει αυτο το δρομολογιο. Μπορειτε να παρετε ενα αλλο πλοιο απο Ηγουμενιτσα στις 5 }το απογευμα.
\dialogue{Αυτο ΔΕΝ γινεται κυρια μου! Μεχρι το μεσημερι αυριο εργαζομαι και δεν μπορω να το φυγω νωριτερα. Αυτος ηταν και ο λογος που επελεξα το βραδυνο δρομολογιο.}
\dialogue{Τι να σας πω, υπαρχει και αλλο ενα που φευγει 12 το μεσημερι απο Πατρα.}
\dialogue{.....}

Εκλεισα γιατι αν μιλουσα με ενα ντουβαρι θα ειχα καλυτερη συνεννοηση και αρχισα πανικοβλητος να κοιταω εναλλακτικες. Πως θα προλαβαινα αυτα τα δρομολογια; Θα επρεπε να φυγω απο πρωι και να καταφερω να μην παω καν στο γραφειο, κατι παρα πολυ δυσκολο! Ακομα δεν ξεκινησαμε και αρχισε ο Μερφυ να κανει παιχνιδι; Τη βαψαμε! Εκει ως απο (βαυαρικης) μηχανης θεος εμφανιστηκε ο αδερφος Κωστης. Του εξηγησα το προβλημα και ανελαβε δραση.

Με τα κοννε του στις ακτοπλοικες της Πατρας εμαθε οτι το ιδιο το πλοιο που θα επαιρνα απο Ηγουμενιτσα το βραδυ εφευγε 5 το απογευμα απο την Πατρα! Απλα δεν εκανε σταση πλεον Ηγουμενιτσα αλλα πηγαινε Ανκονα απευθειας, κατι που η ηλιθια του πρακτορειου δεν γνωριζε καν! Την καλω επι τοπου και τη βαζω να ψαξει με την Superfast. Αφου ενημερωνεται οτι οντως ετσι ειναι, κλεινουμε θεση στο νεο δρομολογιο και σημαινει ληξη συναγερμου! Επιστροφη στο σπιτι για ενα τελικο ελεγχο στα συστηματα της Αυρας: GPS, θερμαινομενα, MP3, ασφαλειες, φωτα... Εχετε ακουσει για κανεναν που απλα εκανε μια αλλαγη λαδια και πηγε Βορειο Ακρωτηρι; Οχι; Ε τωρα θα ακουσετε!

 Το τελευταιο σερβις που της ειχα κανει ηταν στις 88.000 χλμ οποτε τωρα που ειχε αισιως μπει στις 100.000 το μονο που χρειαζοταν ουσιαστικα ηταν μια αλλαγη λαδιων που της χρωστουσα. Ολα τα αλλα ηταν μια χαρα! Τι αλλο να χρειαζοταν για το ταξιδι; 2 το πρωι, μεσα σε μια ζεστη, υγρη νυχτα του Ιουλιου, και οι βαλιτσες κουμπωναν πανω στην Αυρα. Το μυαλο ακομα δεν μπορουσε να το χωρεσει αυτο που πηγαινα να κανω, αλλα τωρα δεν υπηρχε επιστροφη... Η μερα της μεγαλης φυγης ξημερωνε συντομα.

\photo{2.jpg}

Το πρωι στο γραφειο τρελη πιεση να κλεισουν οι εκκρεμοτητες και πανικος να φυγω στην ωρα μου. 
Το αγχος του φευγιου. Πως θα γινει μια φορα να φυγω με το πασο μου; Τρεχουμε και δεν φτανουμε μια ζωη! 
Ο χρονος που νομιζεις οτι εχεις και τελικα ανακαλυπτεις οτι ποτε δεν ειναι αρκετος.

Ο Κωστης μου στελνει μηνυμα: ``Μην ξεχαστεις! Το πλοιο φευγει απο το νεο λιμανι στη Πατρα.'' Ιδεα δεν ειχα! Ποτε εφτιαξαν νεο λιμανι; Το 2010 ειχα φυγει απο το παλιο κλασσικο στην εισοδο της πολης. Το αλλαξαν;; 

Ευτυχως που με ενημερωσε γιατι ιδεα δεν ειχα! Μου δινει σχετικες οδηγιες και κανονιζουμε να συναντηθουμε στην εξοδο της περιφερειακης Πατρων για να με παει μεχρι το πλοιο. 

Κατεβαινω στη μηχανη σαν τον κυνηγημενο. Αγχος, στρες, πιεση, ουφ! Βαζω το κλειδι στη μιζα και... κοντοστεκομαι. Αυτο ειναι! Γυριζω μιζα και ξεκιναω να ζησω ενα ονειρο. Ενα κουμπι με χωριζει. Κλικ. Το μπασο γρυλισμα της Αυρας αντηχει στο υπογειο γκαραζ και εγω χαμογελαω σαν χαζος! Φυγαμε!

Η μηχανη κυλουσε σβελτα στην Εθνικη για Πατρα και το χαμογελο δεν ελεγε να φυγει απο τα χειλη μου. Η διαδρομη εκανε οτι μπορουσε να μου χαλασει τη διαθεση αλλα δεν εδινα σημασια. Μια βδομαδα τωρα οι θερμοκρασιες ηταν πανω απο 40 βαθμους και σε τετοιες συνθηκες το ταξιδι ειναι δραματικο. Η ζεστη χτυπουσε κοκκινα και ο αερας ηταν τοσο καυτος που εκλεισα ζελατινα για να δροσιστω.  Η υπνηλια απο την ανυποφορη ζεστη και τη βαρεμαρα καραδοκουσε και τα χιλιομετρα παντα φαινονται περισσοτερα σε τετοιες συνθηκες..

Μετα απο εναν αιωνα περνουσα πλεον εξω απο τη Πατρα. Η ωρα ηδη περασμενες 4! Ειχα κατι λιγοτερο απο μια ωρα για το πλοιο.  Στη περιφερειακη Πατρων το αγχος για το αν προσπερασα τη σωστη εξοδο η οχι αρχισε να ερχεται ξανα. Ημουν οριακα σε χρονο. Αν επρεπε να γυρισω πισω και να ψαχνω το πλοιο το ειχα χασει!

Πανω που αρχισα να σκεφτομαι οτι ισως να ειχα κανει βλακεια ηρθε η θεα της σωστης πινακιδας να με ανακουφισει!

Ο Κωστης με το Φωτη με περιμεναν απο ωρα στη γεφυρα. Χαμογελο. Υπεροχο συναισθημα κ τοσο τιμητικο να σε μετρανε τοσο οι φιλοι που να καθονται να περιμενουν κατω απο το λιοπυρι για παρτη σου! 

Φυγαμε για το λιμανι συνοδεια. Ο Κωστης με την Μπεμπα μπροστα, ο Φωτης με το αγριμι πισω και η Αυρα φορτωμενη στη μεση. Μα ποιος ειμαι επιτελους; 

Το λιμανη ηταν χωμενο κυριολεκτικα στου διαολου το κερατο. Εγω απο μονος μου δεν θα το εβρισκα εγκαιρως με τιποτα. Ο Κωστης εσωσε το ταξιδι, κυριολεκτικα. Τυπικοτητες στο τσεκ ιν και φυγαμε για το πλοιο που ηταν ετοιμο για αναχωρηση.

Η ωρα ειχε περασει. Ο Κωστης γυρισε στο μερος μου και μου ειπε σε τονο υπηρεσιακο ``Φυγε, απο εδω και μετα ειναι ελεγχομενος χωρος. Ζησε το ονειρο σε καθε στιγμη και καλη ανταμωση.'' Ο χρονος ετρεχε και επρεπε να φυγω ομως ξερω τον Κωστη: λακωνικος, ομως δυο κουβεντες του ισοδυναμουν με χιλιες συζητησεις. Αντιο παιδια και θα τα ξαναπουμε.

\photo{3.jpg}

Δεσαμε τη μηχανη στο αμπαρι, διπλα σε ενα Ducati. Που ηταν ολοι οι αλλοι ταξιδιωτες; Ανεβηκα επανω και εκανα την καθιερωμενη βολτα στους οροφους του πλοιου...  Ομορφο, καινουργιο και καθαρο, ομως δεν εβρισκα μερος που να μπορει να ξαπλωσει κανεις. Που ειναι οι καναπεδες ρε παιδια; Βγηκα στο καταστρωμα ενω το πλοιο ειχε πιασει να σαλπαρει. Βλεπω την ακτη να χανεται κ μαζι της νοιωθω να χανονται και οι σκεψεις, το αγχος, τα προβληματα.. Το μονο που εχει σημασια πλεον ειναι το ταξιδι. Καθε μερα και αλλου, αλητεια κ περιπλανηση, αγνωστα μερη και περιπετεια. Να μην ξερεις τι σε ξημερωνει και να εχεις για μπουσουλα τον οριζοντα.

\photo{4.jpg}

Tο πλοιο ηταν σχεδον αδειο παροτι ημασταν στη μεση του καλοκαιριου. Προφανως λιγοι ειναι οι τρελοι που θελουν (και μπορουν πλεον!) καταμεσης του καλοκαιριου να φυγουν απο Ελλαδα. Φανταζομοαι οτι στην επιστροφη για Πατρα οι τουριστες θα κρεμονταν απο τα ρελια σαν τσαμπια! Ετσι ομως τωρα ηταν καπως καταθλιπτικο το θεαμα -μονο ενα τσουρμο πιτσιρικια γερμανακια φτιαχνουν λιγο την ατμοσφαιρα και μου θυμιζουν οτι ο κοσμος φευγει σε διακοπες.

Ο Νορβηγος φιλος Ole μου εστειλε μηνυμα, φτιαχοντας μου τη διαθεση: ο καιρος εκει πολλα υποσχομενος: ηλιος και 21 βαθμοι σημερα, το μεγαλυτερο που ειχαν δει λεει μεχρι τωρα! Τωρα τι του λες;

Η νυχτα επεσε σιγα σιγα και εγω αρχισα να ψαχνω ενα μερος να την πεσω. Μπρος γκρεμος και πισω ρεμα. Στο σαλονι, τηρωντας τις ενδοξες θαλασσινες παραδοσεις, το αircondition δουλευε τερμα γκαζια: οι θερμοκρασιες μπορει να χαροποιουσαν εναν πιγκουινο αλλα οχι εμενα. Απο την αλλη το καταστρωμα ηταν ζεστο αλλα τα πλαστικα παγκακια δεν τα ελεγες ακριβως και ιδανικα για υπνο. Λαγοκοιμαμαι δυο ωρες με τα φωτα φθοριου στα ματια και σηκωνομαι. Επιστροφη στο σαλονι οπου βρισκω μια καπως ευρυχωρη καρεκλα και κοιμαμαι σαν τελικο σιγμα. Superfast και υπνος ...ελευθερας βοσκης δεν πανε μαζι. 

Το πρωινο ξημερωσε με εναν λαμπρο ηλιο να ανατελλει πανω απο τη θαλασσα. Μοναδικο θεαμα να σε ξυπναει κατι τετοιο...

Σηκωθηκα και βγηκα στο καταστρωμα. Πηρα τον κλασσικο βαπορισιο καφε και αραξα σε μια μερια να απολαυσω τη στιγμη. Στο MP3 οι \href{http://goo.gl/HcaG4}{Starsailor} και οι \href{http://goo.gl/6sdDG}{Oasis} (κλικ ντεεεε!) εφτιαχναν το ιδανικο soundtrack της μερας:

\begin{verse}
There's a fever
On the freeway
In the morning

And the lover
Smiling for me
Without warning

There's an outlaw
On the highway
And she's falling

Man I must have been blind
To carry a torch
For most of my life

These days I'm hanging around
You're out of my heart
And out of my town...
\end{verse}

Αραχτος στο καταστρωμα, με το Zen and the art of Motorcycle Maintenance ανα χειρας, υπεροχος ηλιος, ζεστη, καφεδακι, μουσικη και κατω στο γκαραζ η Αυρα πανετοιμη να περιμενει να φυγουμε! Η Τελεια Στιγμη...

Σε λιγο το πλοιο πιανει λιμανι στην Ancona. Τι μας περιμενει καλη μου;

\photo{5.jpg}

\chapter{Day 1 -- Ancona (IT) - Como (IT) -- 492km}

Η ωρα πανω στα πλοια περναει αργα. Οσοι εχουν ταξιδεψει ετσι ξερουν. Κοιταξα το ρολοι μου. Με τη διαφορα ωρας θα επρεπε να κοντευουμε στο λιμανι. Ομως τριγυρω το μονο που βλεπω ειναι το απειρο μπλε της θαλασσας. Τι ωρα φτανουμε; Η απαντηση απο τον λογιστη του πλοιου με ξαφνιασε: 1 το μεσημερι;! Εγω πως νομιζα οτι φτανουμε 10.30 το πρωι; Η ιδεα μου να χαζεψω στους επαρχιακους δρομους της κεντρικης Ιταλιας τωρα πηγαινε απατη...

Η ωρα τελικα περασε χαζευοντας τους χαρτες για να δω τις διαδρομες που θα ακολουθησω για το προορισμο μου. Τι πιο ομορφο πραγμα απο αυτο; Να ξεκινας τη μερα σου χαζευοντας ολες αυτες τις μικρες πολυχρωμες γραμμουλες πανω στους χαρτες και να προγραμματιζεις αγνωστες διαδρομες για να νυχτωθεις οπου σε βγαλει ο δρομος...

Εγω ομως σημερα ειχα προορισμο: τη λιμνη Como, τη τριτη μεγαλυτερη λιμνη της Ιταλιας στα βορειοδυτικα της χωρας, ακριβως πανω στα Ιταλο-ελβετικα συνορα. Εκει ειναι και το ομωνυμο φημισμενο χωριο που αποτελει ενα (αρκετα κυριλε/ακριβο) τουριστικο θερετρο για τους Ευρωπαιους και οχι μονο, καθως ειναι και το μερος που ζει ο ...George εδω και χρονια. (Μη με ρωτησετε ποιος ειναι ο George ετσι; )
Ηθελα να επισκευτω την περιοχη εδω και χρονια και να που ειχε ερθει η ευκαιρια να το κανω. Βεβαια με το πλοιο να δενει τοσο αργα στην Ancona και εχοντας να διασχισω ουσιαστικα ολη την Ιταλια δεν θα εβλεπα και πολλα απο το Como αλλα δεν πειραζει. Ακομα και ετσι θα επαιρνα μια μικρη γευση!

\photo{6.jpg}

Το πλοιο ειχε πιασει να μπαινει στο λιμανι της Ancona και βρηκα την ευκαιρια να χαζεψω τριγυρω...
Δεν θα το ελεγα ακριβως ωραιο το μερος -ενα μικρο, μαλλον αδιαφορο λιμανακι ηταν αλλα ακομα και ετσι μερικα κτηρια διατηρουσαν την περιφημη Ιταλικη φινετσα και ομορφια...

\photo{7.jpg}

Κατεβηκα στο γκαραζ για να ετοιμαστω. Οσοι με ξερουν γνωριζουν οτι με το χρονο εχω μια πολυ περιεργη σχεση: μια ζωη αργω σε οτι και να κανω -και αυτη η φορα ΔΕΝ ηταν η εξαιρεση.

Τα παιδια με το Multistrada που ηταν παρκαρισμενο διπλα μου ειχαν ηδη κατεβει κατω και περιμεναν υπομονετικα να μαζεψω το τσαντιρι για να φυγουν. Οι παρκαδοροι του πλοιου θελοντας να διευκολυνουν την κατασταση επιασαν να λυνουν τους ιμαντες απο τις μηχανες για να φυγουμε μια ωρα αρχιτερα -ποσο μα ποσο θα το μετανοιωνα αυτο πολυ συντομα....

Για να βοηθησω με τη σειρα μου το ζευγαρι να φυγει μετακινησα την Αυρα 2 μετρα πιο πισω. Ομως εκει που την πηγα το πατωμα ειχε ενα μικρο εξογκωμα. 
Οχι και τοσο σημαντικο θα μου πειτε. Θα συμφωνουσα αν δεν υπηρχαν δυο μικρες λεπτομερειες:

1. Η μηχανη λογω βαρους απο τις βαλιτσες (και λογω εγκεφαλικης βλακειας δικης μου που ΔΕΝ εσφιξα την προφορτιση πριν το ταξιδι) εκανε την αναρτηση να βυθιζεται ελαφρα, ετσι ωστε οταν καθοταν στο stand πλαγιαζε απο λιγο εως ελαχιστα.
2. Οι ιμαντες δεσιματος ειχαν λυθει εντελως.

Ετσι εβαλα το stand αφου ειχα κανει πισω να φυγουν τα παιδια και πηγα πισω στη βαλιτσα να κλεισω απλα το καπακι. Καπου εκει εκανα την ενδιαφερουσα παρατηρηση οτι η μηχανη στηριζοταν σε οριακα ορθια θεση και καθως ακουμπαω το καπακι της βαλιτσας.... ο Τιτανικος βυθιζεται!
340+ κιλα πλαστικων και μεταλλων εφυγαν προς τα δεξια με χαρη που θα ζηλευε και ο ελεφαντας του Circo Medrano. Προφανως οι προσπαθειες να σταματησω το θηριο τραβωντας απο αριστερα ηταν εντελως αναξιες λογου και ετσι βρεθηκα να κοιταω τη μηχανη φαρδια πλατια στο καταστρωμα του πλοιου!

Τι πιο ωραια αρχη για το μεγαλυτερο ταξιδι που θα εκανα ποτε; 
Τελικα αν δεν πεσει η μηχανη μια φορα σε ενα ταξιδι μου δεν θα παει καλα. Το 2010 ειχε συμβει το ιδιο ακριβως 2 φορες αλλα το ταξιδι ειχε παει υπεροχα. Απο την αλλη να πω τωρα οτι χαιρομουν θα ημουν ψευτης...

Σηκωσαμε τη μηχανη πανω μαζι με ενα νταλικιερη που προθυμοποιηθηκε και ...ευτυχως σχεδον καμια ζημια! Η μηχανη ειχε κατσει πανω στη δεξια βαλιτσα σωζοντας τα χειροτερα και το μονο αλλο που ειχε βρει ηταν ενα πολυ μικρο πλαστικο στο δεξι φλας που γεμισε γρατζουνιες. Παλι καλα!

Αυτο που ειχε παθει ομως καλη ζημια ηταν το μεγαλο δαχτυλο μου στο δεξι χερι που πρηστηκε σχεδον αμεσως και δεν μπορουσε να κλεισει. Κακωση; Καταγμα; Το σιγουρο ηταν οτι ποναγε διαολεμενα. Πως θα ταξιδευα ετσι 12.000 χιλιομετρα; Αν ηξερα ομως τι θα ακολουθουσε μεχρι να τελειωσει αυτη η απιστευτη περιπετεια μαλλον θα το προσπερνουσα αυτο ως κατι αναξιο λογου βρε αδερφε...

Καθως ομως η Αυρα ρολαριζε τις ροδες της για μια ακομα φορα επι Ιταλικου εδαφους τα ειχα ξεχασει ολα και χαμογελουσα σαν χαζος...

\photo{8.jpg}

Σταματησα στο πρωτο βενζιναδικο που βρηκα για να ταισω τα αλογα και να αποφασισω για τη διαδρομη. Ηθελα πολυ να βγω στους επαρχιακους δρομους αλλα το GPS μου εκοψε τα ποδια: 10+ ωρες συνεχους οδηγησης και η ωρα ηταν ηδη 2 το μεσημερι. Στην καλυτερη των περιπτωσεων και χωρις καμια σταση (κατι πρακτικα αδυνατον) θα εφτανα τις 12 το βραδυ και βεβαια χωρις να εχω κανονισει καπου να μεινω δεν ηταν και η καλυτερη ιδεα να ψαχνω μεσα στη μαυρη νυχτα καπου να κοιμηθω...

Συνεπως autostrada και ξερο ψωμι για σημερα δυστυχως.
Και το δυστυχως οχι γιατι ειναι κανενας παλιοδρομος (αν και δεν μπορει φυσικα να συγκριθει με το ΥΠΕΡΤΑΤΟ μεγαλειο της Κορινθου - Πατρων) αλλα γιατι εγω βαριεμαι τρομερα τις εθνικες οδους και τις ευθειες και τις αποφευγω οπως ο Lemmy τη Britney Spears. Αναγκαιο ομως κακο σημερα η εθνικη αν ηθελα να φτασω σε καποια λογικη ωρα στο προορισμο μου.

Περα απο τη πλακα βεβαια η autostrada ειναι δρομαρα και αυτο δεν κρυβοταν με τιποτα...

\photo{9.jpg}

...ενω και οι εικονες της επαρχιακης ζωης τριγυρω εκαναν οτι μπορουσαν για να μην πληττω.

\photo{10.jpg}

Συντομα επιασα ρυθμο και η μηχανη αρχισε να τρωει τα χιλιομετρα γουργουριζοντας χαρουμενη, κατι που δεν μπορουσα να πω και για εμενα. Τρελη ζεστη, ηλιος και δερματινα δεν ειναι και ο καλυτερος συνδιασμος. 
Οι στασεις για νερο ηταν συνεχεις και παροτι ανεβαινα ολοενα και πιο βορεια η καψα του μεσογειακου καλοκαιριου δεν ελεγε να κοπασει παρα τα συννεφα που εβλεπα να μαζευονται πλεον στον οριζοντα. 
Τωρα ειχε και ζεστη και μουνταδα!

\photo{11.jpg}

Βγηκα απο την autostrada και προετοιμαστηκα για τις πραξεις ΧΧΧ που θα μου εκαναν στα διοδια: 27 ευρω για 400+ χιλιομετρα; Χμμμ! Περιμενα χειροτερα.
Το απογευμα ειχε πιασει να πεφτει απο ωρα και ανηφοριζα πλεον προς το Como. Ειχε βρεξει εδω πριν λιγο... Ο ουρανος γκριζος πλεον με βαρια συννεφα -θα προτιμουσα λιακαδα, αλλα οι μυρωδιες της φυσης και η ατμοσφαιρικη εικονα που εφτιαχνε ο ουρανος ηταν εξισου ομορφες...

Καθως κατεβαινα απο την εξοδο του δρομου προς τη λιμνη, το Como αρχισε να αποκαλυπτει το λογο γιατι θεωρειται ενας απο τους πιο ομορφους καλοκαιρινους προορισμους στην Ευρωπη.
Εκανα μια σταση στην ακρη του δρομου και τα ματια δεν μπορουσαν να χορτασουν τις εικονες...
Εικονες μιας ταινιας προσεχως...

\photo{12.jpg}
\photo{13.jpg}

To τοπιο ηταν απλα μαγευτικο! Με καποιο παραξενο τροπο η μουνταδα του ουρανου ταιριαζε απολυτα στη στιγμη εκεινη και εγω καθισα σιωπηλος χαμενος στη στιγμη...

\photo{14.jpg}

Θα ηθελα να κανω το γυρο της λιμνης αλλα η ωρα δεν το επετρεπε. Η νυχτα επεφτε πλεον και εγω ακομα δεν ειχα βρει που θα εμενα το βραδυ. Παρολαυτα, η διαδρομη που ξεκινουσε γυρω απο το Como εδειχνε πολλα υποσχομενη και στο μυαλο μου ανανεωσα το ραντεβου μου με την περιοχη για καποια αλλη φορα στο μελλον. Τελικα αν κατσω και βαλω ενα σημαιακι στα μερη που εχω αφησει στα ``προσεχως'' στο τελος δεν θα βλεπω χαρτη για χαρτη!

\photo{15.jpg}
\photo{16.jpg}

Το GPS εδωσε τη λυση για καποιο κοντινο camping: Camping Europa, ενα συμπαθητικο καμπινγκ ακριβως στην εξοδο του χωριου.

\photo{17.jpg}

Δυστυχως ομως η ρεσεψιον οσον αφορουσε την εξυπηρετηση και τη συννενοηση βρισκοταν στο απολυτο μηδεν: ενας βαριεστημενος πιτσιρικας (που ειμαι σιγουρος οτι θα προτιμουσε να ειναι οπουδηποτε αλλου εκεινη την ωρα) μετα βιας εδειχνε ενδιαφερον να εξυπηρετησει ενα ταλαιπωρημενο μηχανοβιο που στεκοταν μπροστα του.
Με τα ελαχιστα ιταλικα μου ευτυχως συνεννοηθηκαμε και μετα τις τυπικες διαδικασιες ξεκινησα να στησω τα συμπραγκ... Επ! Οχι τοσο γρηγορα φιλε μου! Ο καιρος δεν ειχε πει την τελευταια του λεξη ακομα...
Ισα που ειχα προλαβει να απλωσω το εσωτερικο της σκηνης, οταν αρχισαν να πεφτουν δυνατοι κεραυνοι πανω απο τη λιμνη και η ατμοσφαιρα γεμισε τη γνωριμη μυρωδια του οζοντος -σημαδι οτι εντος δευτερολεπτων θα ξεκινουσε ο κατακλυσμος του Νωε!
Τα πυκνα φυλλωματα των δεντρων απο πανω μου θα κρατουσαν μια μικρη βροχουλα αλλα κατι μου ελεγε οτι αυτο που ερχοταν μονο μικρο δεν θα το ελεγες... 

Μαζεψα αρον αρον τη σκηνη και ετρεξα πισω στη ρεσεψιον ενω η βροχη αρχισε να δυναμωνει. 
Καλυβακι; Λαστ γιαρ! Και τωρα τι κανουμε; Να στησω υπο βροχη ηταν κατι που θα προτιμουσα να αποφυγω...
Εν μεσω βροχης το πιτσιρικι μου προτεινε ενα τροχοσπιτο. Πως ειπατε; 
Το καμπινγκ ειχε μερικα παναρχαια, ταλαιπωρημενα τροχοσπιτα στημενα εκει μονιμα ως λυση αναγκης για οσους ηθελαν καλυβακι αλλα δεν εβρισκαν.

Μου εδωσε ενα κλειδι και μου ειπε το νουμερο που επρεπε να βρω. Που ειναι ρε παιδια το νουμερο 12; Δεν υπηρχε πουθενα! Μετα απο τρια τεσσερα πανω κατω στη ρεσεψιον φορωντας τα δερματινα και ιδρωνοντας σαν γουρουνι στο σακι ΠΑΡΑ τη βροχη που επεφτε τελικα καταφερα να εντοπισω το περιφημο τροχοσπιτο. 

Κατι που τελικα μακαρι να μην το ειχα κανει και ποτε! Ολη μου τη ζωη την εχω περασει σε καμπινγκ -ε λοιπον δεν εχω δει ΠΟΤΕ αλλοτε ενα τροχοσπιτο σε τοσο αθλια κατασταση. Τα παντα ηταν σκουριασμενα και σπασμενα. Πομολα, μεντεσεδες στις πορτες, μπρατσα στηριξης στα παραθυρα... Το μεγαλυτερο μερος του τροχοσπιτου εκτελουσε χρεη (σκουπιδ)αποθηκης με οτι σαβουρα μπορει να φανταστει κανεις πεταμενη μεσα ενω το κρεβατι ειχε επανω κατι τεραστιους λεκεδες που θα προτιμουσα να μην ξερω απο τι εγιναν και μια εκλεκτη κολεξιον απο ``Πεθαμενα ζωυφια αγνωστου ταυτοτητας'' του 2012. 
Σαν να μην εφταναν αυτα, το τροχοσπιτο εδειχνε να ...κατοικειται ηδη! Στο τοιχο υπηρχαν κατι ρωσικα εικονισματα και σε μια καρεκλα κατι παλια ρουχα και παντοφλες καποιου αντρα.

Μεσα σε χρονο μηδεν ημουν και παλι στη ρεσεψιον: αστο φιλε, θα το ρισκαρω με τη καταιγιδα χιλιες φορες παρα αυτο το πραγμα!
Εξαλλου με ολα αυτα η μπορα ειχε σταματησει πλεον οποτε χιλιες φορες η σκηνη μου και η φυση παρα να κλειστω σε ενα αθλιο κουτι.

Αραξα τη μηχανη σε ενα ωραιο σημειο που δεν ειχε βραχει πολυ και αρχισα να στηνω. 
Λιγα μετρα πιο περα δυο αλλες ταξιδιαρες μηχανες ηταν αραγμενες: ενα RT1100 και ενα GS1150, που αν εκρινα απο τα αυτοκολλητα και τη κατασταση τους μαλλον ειχαν να πουν πολλες ιστοριες για ταξιδια. 
Βρετανικες πινακιδες με Ουαλλικα διακριτικα! Συντοπιτες.

Καθως το βραδυ ειχε πεσει πλεον το τσαρδι ηταν ετοιμο και ειχα ηδη πιασει να μαγειρεψω κατι για δειπνο...
Καπου εκει ηρθαν και οι Ουαλλοι και κατσαμε να τα πουμε υπο τη συνοδεια παγωμενης μπυριτσας. 
Πολυ ωραιοι τυποι! Παλιοσειρες μηχανοβιοι ταξιδιωτες που ειχαν φαει την Ευρωπη με το κουταλι. Ειχαν ξεκινησει απο Αγγλια πριν μια βδομαδα και αυριο θα ξεκινουσαν να δουν τα πασα της Ελβετιας οπως και εγω! Οι ιστοριες και οι διαδρομες στους χαρτες εδιναν και επαιρναν μεχρι αργα το βραδυ...

Κρατηστε τα πιο κυριλε εστιατορια και τα πιο γκουρμε φαγητα σας. Εγω ετσι τωρα ημουν ο πιο ευτυχισμενος ανθρωπος του κοσμου. Μια σκηνη, η μηχανη διπλα, οι μποτες μου, αναπαντεχες συναντησεις και ενα ζεστο πιατο φαι στο τελος της μερας. Παραδεισος! 
Αυριο ξημερωνε μια νεα μερα...

\photo{18.jpg}

\chapter{Day 2 -- Como (IT) - Guebwiller (FR) 430 km}

Το ξημερωμα με βρηκε πτωμα. Ο αθλιος υπνος που ειχα κανει μεσα στο Superfast και τα 500+ χιλιομετρα μεσα στον πρωτοφανη για την Ιταλια καυσωνα ειχαν κανει αισθητη την επιδραση τους σε ενα σωμα που ειχε μεχρι προτινος καλομαθει σε ολες τις καθημερινες ανεσεις. Η ωρα ηταν περασμενες 8 και παροτι ηθελα να συνεχισω να κοιμαμαι ηξερα οτι επρεπε να σηκωθω αν ηθελα να βγαλω το προγραμμα της ημερας οπως ειχα σχεδιασει. Εξαλλου η πρωινη ζωη του camping που ειχε ξεκινησει εδω και ωρα δεν θα με αφηνε να κοιμαμαι για πολυ ακομα και να ηθελα...

\photo{19.jpg}

Ο προορισμος μου για σημερα ηταν η Γαλλια και συγκεκριμενα το Guebwiller στη νοτιοανατολικη Αλσατια οπου θα συναντουσα τη Christie: ενα μελος του Couchsurfing που μου ειχε προσφερει με χαρα να με φιλοξενησει στο σπιτι της αφου θα περνουσα απο την περιοχη. Γιατι ομως εκει; Διοτι σε αποσταση λιγων χιλιομετρων ηταν το περιφημο \href{http://en.wikipedia.org/wiki/Colmar}{Colmar}: καποιοι το ανεφεραν ως τη πιο ομορφη πολη της Ευρωπης και δεν θα αφηνα να παει χαμενη η ευκαιρια να διαπιστωσω αν ηταν αυτο ηταν αληθεια...

Πριν απο αυτο ομως σημερα θα επαιρνα την εκδικηση μου για τα βαρετα χιλιομετρα εθνικης που ειχα γραψει χθες! Ειχα να διασχισω την Ελβετια και σκοπευα να κινηθω οσο μπορουσα σε επαρχιακους δρομους και να απολαυσω τη μεγαλειωση φυση της χωρας αλλα βεβαια το μεγαλυτερο κερασακι ηταν αλλο: τα διασημα πασα Furkapass και Grimselpass στα 2.400 μετρα υψομετρο στην ραχη των Ελβετικων Αλπεων.
Οταν εχεις ενα τετοιο προγραμμα ξυπνας πιο ευχαριστα οσο να ναι...

\photo{20.jpg}

Πρωτα απο ολα ομως καφες. Οι Ουαλλοι διπλα ειχαν σηκωθει απο ωρα και επιναν ηδη το τσαι τους χαμογελαστοι, ενω δεν εχασαν ευκαιρια να μου την πουν οτι παρακοιμηθηκα κιολας! Οι παλιολυκοι μου εβαλαν τα γυαλια!
Ζεστανα νερο στο κουζινακι και εφτιαξα τον καφε μου και εκατσα να τον απολαυσω πλαι στους χαρτες μου και με παρεα αλλους μηχανοβιους ταξιδιωτες.
Το εχω πει πολλες φορες και θα το πω αλλες τοσες: δεν υπαρχει καλυτερος τροπος απο το να ξεκινας ετσι τη μερα σου!

\photo{21.jpg}

Οι Ουαλλοι μου ελεγαν οτι θα ακολουθησουν πανω κατω την ιδια διαδρομη με εμενα, τουλαχιστον μεχρι τα πρωτα πασα. Εκανα τη σκεψη να παμε μεχρι εκει παρεα ομως τους ειδα να βιαζονται να ξεκινησουν: εγω δεν ειχα προλαβει να πιω ακομα τον καφε μου και αυτοι ειχαν ξεστησει τις σκηνες τους και ειχαν σχεδον πακεταρει τα παντα! 
Αρχισα να μαζευω τα πραγματα βιαστικα αλλα καταλαβα οτι απλα θα τους καθυστερουσα χωρις λογο και εγω θα αγχωνομουν αδικα. Ανταλλαξαμε συμβουλες για τη διαδρομη και χαιρετηθηκαμε -μπορει να τα λεγαμε και στη πορεια εξαλλου!

Το πακεταρισμα παντα ηταν για εμενα ενα θεμα: μου τρωει χρονο. Πολυ χρονο. Μεχρι να μαζεψω υπνοσακους, υποστρωματα, σκηνη, τεντες, ρουχα και λοιπα συμπραγκαλα οι περισσοτεροι γειτονες ειχαν φυγει ηδη! Οκ, βεβαια πολλοι ηταν με τροχοσπιτα και αυτοκινητα αλλα οπως και να το κανεις ειναι αποκαρδιωτικο να παλευεις να σφηνωσεις ολα τα μπαγκαζια στις βαλιτσες και να βλεπεις τους αλλους να ξεστηνουν τα παντα χαμογελαστοι στο χρονο που σου παιρνει εσενα να κλεισεις την μια βαλιτσα! 
Βεβαια ειχα φερει μαζι μου και κατι παραπανω οσο να ναι... (Μιας που για να φας στη Νορβηγια πρεπει να παρεις ενα μικρο στεγαστικο δανειο ειχα προετοιμαστει καταλληλα)

\photo{22.jpg}

Βγηκα στους δρομους του Como και ο ηλιος εδειχνε οτι θα ειναι μια υπεροχη -και πολυ ζεστη- μερα. 
Απο ψηλα η λιμνη και τα σπιτια αμφιθεατρικα στο βουνο εφτιαχναν ενα πολυ ομορφο θεαμα -για αλλη μια φορα εδωσα υποσχεση στον εαυτο μου να ξαναρθω σε αυτα τα μερη για μια δευτερη αναγνωση...

\photo{23.jpg}

Ενω περνουσα μεσα στους δρομους του χωριου, ξαφνικα.... συνορα;!; Τα συνορα με την Ελβετια ειναι μεσα στη πολη! 
Δεν το ειχα ξαναδει αυτο ποτε. Οκ, η προσβαση ειναι ελευθερη στους Ευρωπαιους πολιτες, αλλα οπως και να το κανεις, το να βλεπεις συνοριακο φυλακιο, αστυνομικους στη μεση ενος δρομου και σημαιες μιας αλλης χωρας ειναι κατι εντελως ασυνηθιστο. 
Απο την μια μερα της πολης Ιταλια και απο την αλλη Ελβετια!

\photo{24.jpg}

Αν δειτε το χαρτη της περιοχης θα παρατηρησετε οτι τα συνορα κοβουνε το χωριο στη μεση ουσιαστικα. Απο τη μερια της Ιταλιας ειναι το Como και απο τη μερια της Ελβετιας το Chiasso. Βεβαια η γλωσσα παρεμενε η Ιταλικη, τα μαγαζια συνεχιζαν να παιρνουν ευρω αλλα καταλαβαινες οτι κατι ανεπαισθητο εχει αλλαξει... Πολυ παραξενο συναισθημα! 

Με κατι τετοια κτηρια ομως θα μπορουσα να πιστεψω οχι μονο οτι ειμαι σε αλλη χωρα αλλα και σε αλλο ...πλανητη! Αυτος ο ιπταμενος δισκος τεραστιων διαστασεων ηταν το εμπορικο κεντρο της περιοχης.

\photo{25.jpg}

Ομως εγω τωρα ηθελα να χαθω σε επαρχιακα δρομακια... 
Να δω χωρια, να χαρω τη φυση, να πιασω μια ροτα στο περιπου και να βγω οπου με παει ο δρομος.

\photo{26.jpg}

Ο δρομος αρχισε να ανηφοριζει στο βουνο με ποιοτητα ασφαλτου που θα ζηλευε και πιστα....

\photo{27.jpg}

...χωριουδακια τριγυρω σαν ψευτικα....

\photo{28.jpg}

...δαση και ατελειωτο πρασινο οπου και να γυριζε το ματι....

\photo{29.jpg}

....και μια θεα που σε εκανε απλα να στεκεσαι και να κοιτας σαν χαμενος!

\photo{30.jpg}

Ο δρομος συντομα με εβγαλε στις οχθες της λιμνης Lugano και της ομωνυμης πολης που βρισκεται εκει...

\photo{31.jpg}

Η αρχιτεκτονικη ηταν εμφανως Ιταλικη, κατι αναμενομενο αν σκεφτει κανεις οτι ολη αυτη η περιοχη ανηκε επι αιωνες στους Ιταλους. Ημουν σιγουρος οτι δεν θα ηταν και πολυ χαρουμενοι που τωρα ηταν μερος της Ελβετιας, αλλα ακομα και ετσι ηταν ουσιαστικα ιταλικη περιοχη. 
Κοιταζοντας αυτες τις διαχωριστικες γραμμες στο χαρτη να πασχιζουν να ορισουν περιοχες με εκανε να σκεφτομαι ξανα τη ματαιοδοξια και γελοιοτητα του να προσπαθεις να χωρισεις γη και πολιτισμο σε ταμπελες: ``Ιταλια'', ``Ελβετια'', εμεις, εσεις. Μονοι εμεις οι ανθρωποι καταφερνουμε κατι τοσο αδιαιρετο οσο η Γη να το κοβουμε σε κομματια και να σκοτωνομαστε κιολας για αυτα...

\photo{32.jpg}

Ολες ομως αυτες οι σκεψεις εξατμιστηκαν με μιας οταν η θεα της λιμνη ξεδιπλωθηκε μπροστα μου... 
Τι μαγευτικο θεαμα!

\photo{33.jpg}
\photo{34.jpg}

Το προγραμμα απο την αρχη σημερα ηταν να κανω οσο πιο πολλα χιλιομετρα μπορουσα σε επαρχιακους δρομους. 
Ομως τωρα ημουν λιγο εξω απο την Bellinzona και ειχα ηδη κανει μιαμιση ωρα διαδρομης για μολις 70 χιλιομετρα -τα χωριουδακια και οι φιδωτοι ορεινοι δρομοι μπορει να ηταν ακρως γραφικοι αλλα δεν βοηθουσαν να γραψω χιλιομετρα.

\photo{35.jpg}
\photo{36.jpg}

Κοιταξα το GPS. Αν ηθελα να ειμαι στη Γαλλια νωρις το βραδυ θα επρεπε να βαλω λιγο νερο στο κρασι μου και να πιασω την εθνικη, τουλαχιστον μεχρι τα πασα. Η Christie με περιμενε στο σπιτι της και θα ηταν μεγαλη αγενεια να φτασω μεσα στη μαυρη νυχτα και απλα να πεσω να κοιμηθω σε ενα καναπε για να αναχωρησω την επομενη μερα.

Ετσι μετα απο μια σταση για ανασυγκροτηση προμηθευτηκα την Ελβετικη βινιετα οσο και αν πονουσε -27 ευρω ειναι αυτα!- και βγηκα στην εθνικη...
Αν εκρινα ομως απο τη θεα που ανοιγοταν μπροστα μου κατι μου ελεγε οτι δεν θα βαριομουν καθολου!

\photo{37.jpg}

Ολα πηγαιναν περιφημα. Η Αυρα ειχε την ευκαιρια να ξεμουδιασει λιγο, τα χιλιομετρα εφευγαν γρηγορα και εγω χαζευα την καταπρασινη φυση γυρω μου με ενα χαμογελο μεχρι τα αυτια! Και ολα θα συνεχιζαν να πηγαινουν περιφημα αν δεν υπηρχε κατι που ακουγε στο ονομα Galleria stradale del San Gottardo. 

Καποια στιγμη στην εθνικη οδο ειδα στο βαθος κινηση. Το ειδος της κινησης που ξερεις οτι κατι συμβαινει. Τα αυτοκινητα ηταν ακινητοποιημενα σε ολες τις λωριδες και περιμεναν. Σταματησα στην ακρη δεξια και προσπαθησα να καταλαβω τι γινεται. Ατυχημα; 
Τι αλλο θα μπορουσε να μπλοκαρει ολες τις λωριδες σε μια τετοια εθνικη οδο; Η ζεστη του μεσημεριου ηταν ανελεητη και τα δερματινα δεν βοηθουσαν ιδιαιτερα τη κατασταση. 
Ευτυχως ομως συντομα καποιοι αλλοι μοτοσυκλετιστες αρχισαν να περνανε σιγα σιγα απο την ΛΕΑ με αλαρμ και αποφασισα να τους ακολουθησω. Φτιαχνοντας στιχακια στο μυαλο μου ("ΛΕΑ ΛΕΑ και η ζωη ειναι ωραια" κλπ) και μετα απο μερικα χιλιομετρα πορειας αναμεσα σε λεωφορεια και κολωνακια εφτασα στην κεφαλη της ουρας: τα αυτοκινητα περιμεναν σε φαναρι! 

Μπροστα μου ανοιγοταν η εισοδος ενος τουνελ και η κυκλοφορια γινοταν σε δοσεις. Αυτο το ειχα δει στο παρελθον στην Αυστρια οποτε δεν εδωσα ιδιαιτερη σημασια και περιμενα το πρασινο για να μπω στο τουνελ. Λαθος.
Το ποσο μεγαλο λαθος ηταν αυτο το ανακαλυψα καθως εμπαινα στο τουνελ και ειδα την πινακιδα: Galleria stradale del S. Gottardo, 16.400 m. Ποσα;; 16;;; Δεκαεξι χ ι λ ι ο μ ε τ ρ α;;;

Καπου εκει θυμηθηκα ενα βραδυ που κοιταζοντας τους χαρτες στο σπιτι ειχα δει μια τεραστια ευθεια γραμμη καπου στις Ελβετικες Αλπεις και σκεφτηκα οτι δεν μπορουσε να ειναι δρομος αυτο το πραγμα. Εντελως ευθεια και τοσο μεγαλο σε μηκος... Δεν μπορει. 
Και ομως. Αυτη τη στιγμη βρισκομουν στο τουνελ του St. Gottard, το τριτο μεγαλυτερο τουνελ στον κοσμο με μηκος πανω απο 16 χιλιομετρα απο ακρη σ' ακρη!

\photo{38.jpg}

Ζωντας εδω στην Ελλαδα που τα μεγαλυτερα τουνελ σπανια ξεπερνανε τα 1-2 χιλιομετρα δεν ειναι ευκολο να καταλαβει κανεις τι σημαινει να διασχιζεις ενα τοσο τερατωδες τουνελ. 
Τα χιλιομετρα αρχισαν να περνανε το ενα μετα το αλλο αργα και βασανιστικα και συντομα καταλαβα οτι θα ειχα προβλημα. Εκλεισα τη ζελατινα καλα και ολους τους αεραγωγους σε μια απεγνωσμενη προσπαθεια να προφυλαχτω. Το θερμομετρο εδειχνε 52 βαθμους εξωτερικη θερμοκρασια και τα καυσαερια απο τα οχηματα εκαναν την ατμοσφαιρα να θυμιζει θαλαμο αεριων. Δραματικη κατασταση...

Παρα ολ αυτα ομως δεν μπορουσα να μην θαυμασω το τι εφτιαξαν οι ανθρωποι. Ακρως εντυπωσιακο αν και εξαιρετικα δυσκολο στη διασχιση για ενα μηχανοβιο...

Μετα απο 20 λεπτα που φανηκαν σαν αιωνας βγηκα στην αλλη πλευρα! Ουφ! Βαθειες ανασσες και ο δροσερος πλεον αερας εμοιαζε τοσο γλυκος τωρα!

\photo{39.jpg}

Η πινακιδα που μολις ειχα περασει εγραφε τα μαγικα γραμματα: Furkapass δεξια. 
Χαμογελασα. Τα πασα με περιμεναν...

\photo{40.jpg}
\photo{41.jpg}

Το τοπιο πλεον ειχε αλλαξει δραματικα. Οι χαμηλες καταπρασινες πλαγιες της βορειας Ιταλιας ειχαν δωσει τη θεση τους σε τεραστιους βραχωδεις ογκους που ορθωνονταν γυρω μου σε ενα θεαμα πραγματικα μεγαλειωδες...
Κοιτες ποταμων περνουσαν αναμεσα σε πανυψηλες οροσειρες, ενω ο δρομος εμοιαζε με ενα γκριζο φιδακι που ξεδιπλωνοταν αναμεσα σε βραχους, δεντρα και νερο...

\photo{42.jpg}

Στο βαθος τα συννεφα χορευαν πανω στις καταπρασινες κορυφες, σε ενα αεναο παιχνιδι της γης με τον ουρανο, φτιαχνοντας μια μοναδικη εικονα απο αυτες που σε συνοδευουν για παντα και σου θυμιζουν οτι η ομορφια βρισκεται εκει εξω, στα πιο απλα, στα πιο μικρα... 
Και μεσα σε ολα αυτα, ο δρομος μπροστα που ανοιγοταν μεχρι τον οριζοντα... 

\photo{43.jpg}

Εδω ανοιγαν τα ματια, η ψυχη και το μυαλο και προσπαθουσαν να χωρεσουν μεσα τους το μεγαλειο αυτο... Τιποτε αλλο...
Και οπως εβλεπα δεν ημουν ο μονος που σκεφτοταν ολα αυτα τριγυρω... 

\photo{44.jpg}

O δρομος συντομα αρχισε να ανηφοριζει μεσα απο μικρους και γραφικους οικισμους, ολοενα και πιο στενος, αλλα τιποτα δεν μαρτυρουσε αυτο που βρισκοταν μπροστα...

\photo{45.jpg}

...ωσπου οι πρωτες στροφες του πιο διασημου Ελβετικου πασου ανοιχτηκαν μπροστα μου. 
``Stairway to heaven'', σκεφτηκα χαμογελοντας και το παρτυ ξεκινουσε!

\photo{46.jpg}

Αρχισα να ανεβαινω και η μια φουρκετα διαδεχοταν την αλλη σε ενα γιγαντιο rollercoaster, απανωτες ανηφορικες στροφες ξανα και ξανα και ξανα, με μια ασφαλτο που σε προκαλουσε να ξυσεις μεχρι και τα γκριπ! 
Και ολα αυτα ενω η θεα κατω πραγματικα εκοβε την ανασσα...

\photo{47.jpg}

Τρεχουμενα νερα παντου, ποταμια που κατεβαιναν απο τις πλαγιες του βουνου και μεγεθη που δεν θα χωρουσαν στον καλυτερο ευρυγωνιο φακο του κοσμου... Τι να σου κανει μια ταλαιπωρη φωτογραφικη οταν εχεις μπροστα σου τετοιες εικονες;

\photo{48.jpg}
\photo{49.jpg}

Ανεβαινοντας, απο ενα σημειο και μετα η μπαλα χαθηκε τελειως... Το μυαλο μηδενισε, κλειδωσε, πεταξε τα κλειδια στον αγυριστο και παρεδωσε στα ματια και την ψυχη. 
Αυτο που εβλεπα μπροστα μου απλα δεν μπορουσα να το περιγραψω με λογια... 
Εικονες βγαλμενες απο τα καλυτερα μου ονειρα...!

\photo{50.jpg}
\photo{51.jpg}

Το GPS εδειχνε ηδη 2.000 μετρα υψομετρο και τα συννεφα κυλουσαν συμπαγη πανω στην ασφαλτο σαν να ηταν κατι ζωντανο...

\photo{52.jpg}

... ενω εκαναν ολο το δρομο να μοιαζει σαν να εχει παρει φωτια...!

\photo{53.jpg}

Ωσπου τελικα.... Ο προορισμος:

\photo{54.jpg}

Εχοντας δει πολλες φορες ταξιδιωτικα με τα γνωστα parking και τα μεγαλα τουριστικα κιοσκια στα πασα των Αλπεων, θεωρουσα οτι ετσι θα ηταν και εδω. Οχι ακριβως. Στο υψηλοτερο σημειο του πασου που βρισκομουν αυτη τη στιγμη δεν υπηρχε τιποτα παρα μονο ενα παλιο και ερειπωμενο πλεον πανδοχειο, πραγματικα στη μεση του πουθενα... 

\photo{55.jpg}

Ποσες φορες στη ζωη μας μπορουμε να βαζουμε στοχους και να ειμαστε στη θεση να τους φτανουμε, να τους αγγιζουμε με τα ιδια μας τα χερια; Αυτο το ταξιδι ηταν ακριβως αυτο: στοχοι, σημεια στο χαρτη ενος μυαλου, που ηθελε να τα δει απο κοντα, να τα αγγιξει, να νοιωσει τη χαρα της κατακτησης, οτι και αυτος ηταν εκει! 
Το αν αξιζε η αναμονη και ο κοπος θα φαινοταν ``στο χειροκροτημα'' ομως αν νομιζατε οτι ειχαμε τελειωσει ετσι απλα, απατασθε...

\photo{56.jpg}

Το κρυο σε τετοιο υψος δεν αστειευοταν παρα τον ηλιο που εκανε φιλοτιμες προσπαθειες να με ζεστανει. Αν κατακαλοκαιρο με ηλιο ειχε τετοια ψυχρα, το χειμωνα ηθελα καν να σκεφτω πως θα ηταν εδω.

Καπου εκει ακουσα ηχο μηχανων και ειδα μια μεγαλη ομαδα μηχανοβιων να περνανε χαιρετωντας και αποφασισα να τους ακολουθησω κατεβαινοντας πλεον απο την αλλη πλευρα του βουνου. Δεν ειχα προλαβει να κανω πανω απο 3 χιλιομετρα οταν ειδα το λογο που οι περισσοτερες μηχανες δεν σταματουσαν στην κορυφη οπως εγω αλλα συνεχιζαν πιο κατω...

\photo{57.jpg}

Σταματησα στην ακρη της αλανας και καθισα εκει σωπηλος. Χαιδεψα το ντεποζιτο της Αυρας και θαρρεις την ακουσα να μου ψυθιριζει ``Σε τετοια μερη θελω παντα να με φερνεις, να γεμιζω απο τις ομορφιες του κοσμου...''

\photo{58.jpg}

Δεν ηξερα τι να πω. Η θεα ηταν τοσο εκθαμβωτικη που χαμογελουσα σαν χαζος μπροστα σε αυτο που δεν χωρουσε μεσα στα ματια μου...

\photo{59.jpg}

Στο mp3 αρχισε να παιζει το Freebird των Lynyrd Skynyrd και αρχισα να σιγοτραγουδαω...

\begin{verse}
If I leave here tomorrow
Would you still remember me?
For I must be traveling on, now,
Cause there's too many places I've got to see.
But, if I stayed here with you, girl,
Things just couldn't be the same.
Cause I'm as free as a bird now,
And this bird you can not change..."
\end{verse}

\photo{60.jpg}

Καθομουν εκει πολυ ωρα. Καποια στιγμη το μηνυμα στο κινητο με ξυπνησε απο το ονειρο. Ηταν η Christie που ρωτουσε που ημουν. Η ωρα πλησιαζε 4 το απογευμα και ειχα αλλα 250 χιλιομετρα μεχρι το Guebwiller -oσο και να μην ηθελα επρεπε να πηγαινω.
Καβαλησα τη μηχανη και ξεκινησα να κατεβαινω το πασο ομως ειχα πολλα ακομα να δω...

\photo{61.jpg}

Εξαλλου ο δρομος που τοση ωρα χαζευα απο ψηλα περνουσε ετσι και αλλιως απο το αλλο διασημο πασο της Ελβετιας, το Grimselpass.

\photo{62.jpg}

Οι ρυθμοι ραθυμοι πλεον -ηθελα να απολαυσω καθε στιγμη που βρισκομουν σε αυτα τα μοναδικα μερη και το ιδιο εκαναν και οι συνταξιδιωτες μου...

\photo{63.jpg}

Οσο ανεβαιναμε ομως ο καιρος αρχιζε πλεον να δειχνει τα δοντια του: ο ηλιος χαθηκε μεσα στα συννεφα και το κρυο τωρα ηταν πολυ τσουχτερο... 
Οντας μεσα σε αρκετη ομιχλη και με το ψιλοβροχο να πεφτει δεν ειχα πολλα να δω: ο δρομος συντομα με εβγαλε στη κορυφη οπου περιμενε το κλασσικο τουριστικο περιπτερο, με καποιους παραξενους εκπροσωπους της τοπικης πανιδας...

\photo{64.jpg}

...αλλα και μερικους ακομα πιο περιεργους συνταξιδιωτες!
Grimselpass, 2012

\photo{65.jpg}

Εκανα σταση να ξεμουδιασω και περπατησα τριγυρω. Το μερος ειχε πολυ πλακα. Οι παραξενες σιδερενιες κατασκευες ηταν διασπαρτες τριγυρω. Ποιος τις ειχε φτιαξει αραγε; Παντως οποιος και να ηταν ειχε πολυ ταλεντο και μερακι...

Οπως περπατουσα χαζευοντας το επιβλητικο ορεινο τοπιο, μεσα στην πυκνη ομιχλη ξαφνικα ειδα.... νερο;! Απιστευτο! Μια μεγαλη λιμνη στα 2.100 μετρα! 

\photo{66.jpg}

Και αυτο δεν ηταν τιποτα...
Η πορεια συνεχιστηκε προς το βορρα μεσα στη καταχνια, ωσπου μεσα απο την ομιχλη εμφανιστηκε μια εικονα που δεν θα ξεχασω ποτε: το φως λες και επαιζε με τα συννεφα και επεφτε πανω μια μεγαλη λιμνη που απλωνοταν τωρα μπροστα μου γεμιζοντας την με χρωματα! 
Η ``μοναδικη στιγμη'' -εκεινο το σημειο στο χρονο που ολα συμπιπτουν για να δημιουργησουν κατι πανεμορφο- και εγω ημουν τοσο τυχερος να ειμαι εδω τωρα!

\photo{67.jpg}

Ο δρομος μπροστα ανοιγοταν ατελειωτος ξανα και εγω παρακαλουσα αυτο το ταξιδι να μην τελειωσει ποτε...

\photo{68.jpg}

Δρομοι και εικονες βγαλμενες θαρρεις απο ονειρα... Η Ελβετικη φυση ηταν πραγματικα συγκλονιστικη!

\photo{69.jpg}

Ο δρομος απλωνοταν μπροστα μου ανοιχτος μεχρι εκει που εβλεπε το ματι και τα τοπια ηταν σαν πινακας ζωγραφικης...

\photo{70.jpg}
\photo{71.jpg}
\photo{72.jpg}

Στο παρελθον ειχα κανει πολλες φορες χιλιομετρα σε πανεμορφους επαρχιακους δρομους. Ομως αυτη τη φορα πραγματικα δεν ηξερα που να πρωτοκοιταξω απο την ομορφια -η διαδρομη απο το Grimsel μεχρι το Lungern ηταν ανετα μια απο τις πιο ομορφες που ειχα δει ποτε! Και επειδη καποιες φορες ακομα και οι εικονες δεν μπορουν να δειξουν τη μαγεια... απολαυστε! \href{http://www.youtube.com/watch?v=UuPvab4E5ZI}{Swiss Alps}

Συντομα ειχα βγει στο τελευταιο κομματι της απιθανης αυτης διαδρομης πριν την εθνικη οδο.
Ο δρομος περνουσε μεσα απο ενα πυκνο δασος, φτιαχνοντας ενα ατελειωτο καταπρασινο τουνελ που με προκαλουσε για ατελειωτο παιχνιδι... 

\photo{73.jpg}

Και καπου εδω ηταν που κρυμμενη αναμεσα στις φυλλωσιες των δεντρων βρισκοταν ισως η πιο ομορφη εικονα της ημερας...
.
.
.
.
.
.
.
Τα λογια ειναι απλα περιττα νομιζω...

\photo{74.jpg}

Δεν θα μπορουσα να ειχα ζητησει καλυτερο κλεισιμο για το κομματι των ορεινων και επαρχιακων δρομων της Ελβετιας.
Το ρολοι της μηχανης εδειχνε ηδη περασμενες 5 και ηταν ωρα να πιασω την εθνικη οδο για να γραψω τα τελευταια χιλιομετρα μεχρι το προορισμο μου. Οπως ολα τα αλλα σε αυτη τη χωρα, η εθνικη οδος τους ηταν απολυτως τελεια σε ...εκνευριστικο βαθμο!

\photo{75.jpg}

Συντομα ειχα βρεθει για μια ακομα φορα σε μια ακομη συνοριακη γραμμη: Vive la France!

\photo{76.jpg}

Μολις 40 χιλιομετρα εμεναν για το προορισμο μου και ενω σκεφτομουν το ποσο καλο καιρο ειχε κανει σημερα ο Μερφυ εκανε το θαυμα του: ξαφνικα και απο το πουθενα επιασε μια απιστευτα δυνατη μπορα ενω απο πανω ειχε ηλιο! Δεν ειχε ουτε ενα συννεφο απο πανω αλλα εγω ειχα γινει μουσκεμα. Μα απο ΠΟΥ με βρεχει ρε παιδια;
Συνεχισα μεσα στη βροχη απτοητος -ετσι και αλλιως ειχα σχεδον φτασει και η μπορα δεν θα κρατουσε για πολυ.
Οντως δεκα λεπτα αργοτερα η βροχη σταματησε και εγω εμπαινα στη μικρη κωμοπολη του Guebwiller.

\photo{77.jpg}
\photo{78.jpg}

Ολα πεντακαθαρα, μικρα δρομακια, χρωμα και πανεμορφες λεπτομερειες στα σπιτια παντου. 
Πολυ ομορφο μερος!

\photo{79.jpg}

Καπου εδω ηταν που με περιμενε η Christie -μια θεοπαλαβη, πολυ καλοστεκουμενη και χαμογελαστη 55αρα, με μια απιστευτη θετικη διαθεση και ορεξη για ζωη! Εμενε σε ενα πανεμορφο παλιο σπιτι Αλσατικου τυπου μαζι με το συντροφο της Francois και τη φοβερη γατα τους τη Lolotte.
Τακτοποιησα τη μηχανη και ανεβηκαμε πανω. Ηταν τοσο ομορφο συναισθημα να καταληγεις σε ενα ζεστο σπιτικο μετα απο μια ολοκληρη μερα στο δρομο!

Στη τραπεζαρια το τραπεζι ηταν στρωμενο και το φαγητο περιμενε. Οι ανθρωποι ειχαν ετοιμασει δειπνο για λογαριασμο μου! Ενοιωσα μεγαλη τιμη και μονο στη σκεψη οτι καποιος εμπαινε σε τετοιο κοπο για εμενα...
Η βραδια κυλισε υπεροχα με καλη παρεα, κουβεντουλα, εξαιρετικο γαλλικο κρασι και πολλα χαμογελα, ειδικα οταν εγω και Francois πιαναμε συζητηση: μπορει τα αγγλικα του να ηταν οσο καλα οσο και τα Γαλλικα μου (κοινως χαλια) αλλα αυτο δεν μας εμποδιζε να καταλαβαινομαστε μια χαρα! 
Οι πρωτες πρωινες ωρες μας βρηκαν σε φιλοσοφικες συζητησεις παρεα με μερικα ποτηρακια του εθνικου ποτου των Γαλλων: Pastis (κατι σαν το δικο μας ουζο).

Ο πιο ομορφος τροπος να κλεισεις μια τοσο τελεια ημερα...

\chapter{Day 3 -- Guebwiller - Colmar - Leverkuzen - 504km}

Ο ηλιος που περναγε μεσα απο τις κουρτινες στο παραθυρο επεσε στα ματια μου και με ξυπνησε. Χαμογελασα. Οταν βρισκεσαι στο δρομο η παρουσια του ηλιου ειναι το καλυτερο δωρο που μπορεις να ζητησεις.
Ανοιξα τα ματια και κοιταξα γυρω το σαλονι καλυτερα. Απεναντι μου ενα τεραστιο καλλιτεχνικο κολαζ απο Νεα Υορκη -κατι αναμεσα σε πινακα ζωγραφικης και φωτογραφιες- πλαισιωνοταν απο μικες κορνιζες και κουκλακια κρεμασμενα στο τοιχο.
Στο τραπεζακι του σαλονιου και το περβαζι του παραθυρου υπηρχαν γλαστρακια και διαφορα παλια αντικειμενα στολισμενα. Φοβερη διακοσμηση! Παντου τριγυρω στο σπιτι υπηρχαν μικρες εξαισιες λεπτομερειες: παλια αντικειμενα, πινακες ζωγραφικης, κολαζ, χειροτεχνιες, κουκλακια, κατασκευες και χρωματα. Πολλα χρωματα.

\photo{80.jpg}

Ηταν νωρις ακομα και η Christie κοιμοταν. Σηκωθηκα και χαζεψα λιγο τριγυρω.
Το σπιτι ηταν πολυ ομορφο. Πολυ παλιο (πανω απο 200 ετων οπως εμαθα αργοτερα) με μεγαλα ξυλινα δοκαρια να προεξεχουν εκτεθειμενα στην οροφη και τους τοιχους και πολλα δωματια με μικρα περασματα χωρις πορτες. Η αρχιτεκτονικη ηταν στο στυλ των Γερμανικων χωριατοσπιτων, λογικο αν σκεφτει κανεις οτι η Αλσατια ηταν επι πολλους αιωνες Γερμανικη επαρχια.
Η Lolotte ειχε ηδη ξυπνησει και με κοιτουσε με περιεργεια απο απεναντι: ``τι γυρευεις στο σπιτι του αφεντικου μου;''

\photo{81.jpg}

Συντομα ξυπνησε και η Christie και αρχισε να ετοιμαζει πρωινο ενω εγω εκατσα να δω το προγραμμα της ημερας.
Σημερα θα πηγαιναμε μαζι με τη Christie να δουμε το περιφημο Colmar και μετα θα αναχωρουσα για το Leverkusen της Γερμανιας οπου με περιμενε ο πολυ αγαπημενος μου φιλος Κωστας -ο διεθνως διασημος συνταξιδευτης Sieche- με μια μικρη σταση πρωτα στη Χαιδελβεργη για να δω τον παντα χαμογελαστο και εξαιρετικα ευγενικο δεν-με-λενε-Στεφανο-αλλα-Σταυρο akaSteven ER.
Διεθνης συναντηση mybike λεμε -η μερα ηταν αφιερωμενη στους φιλους!

\photo{82.jpg}

Εξω η λιακαδα ηταν απλα υπεροχη! Η Christie με περιμενε ηδη εξω και καθισαμε για πρωινο στο μακραν πιο ομορφο μπαλκονι που εχω δει ποτε!

\photo{83.jpg}

Πραγματικα ηταν απεριγραπτο. H φοβερη διακοσμηση του σπιτιου εδω εδινε ρεστα!
Παντου κουκλακια, γλαστρες, λουλουδια και φανταστικες χειροτεχνιες και τοσο μα τοσο χρωμα...

\photo{84.jpg}
\photo{85.jpg}

Καθως ο πρωινος ηλιος επεφτε ελουζε το μικρο μπαλκονι εκανε τα παντα να μοιαζουν ακομα πιο εντονα και ζωντανα: Μια απιστευτη εκρηξη χρωματων!

\photo{86.jpg}

Καθως τρωγαμε πρωινο και συζητουσαμε για τη καθημερινοτητα εκει μια επιμονη σκεψη μου ηρθε στο νου: κοιτα που ειμαι τωρα! Αν δεν ηταν το Couchsurfing τωρα θα βρισκομουν σε ενα τυπικο camping, hostel η χειροτερα σε ενα απροσωπο ξενοδοχειο, αγνωστος μεταξυ αγνωστων.
Αλλα αντι για αυτο καθομουν εδω και απολαμβανα το καλυτερο πρωινο που θα μπορουσα να σκεφτω σε ενα πανεμορφο Γαλλικο σπιτι με εξαιρετικη παρεα.
Τα λογια ειναι φτωχεια... Κοιταχτε απλα που και πως ζουν αυτοι οι ανθρωποι! (και να σκεφτει κανεις οτι για εμενα η ιδεα του πρωινου αρχιζε και τελειωνε σε ενα νες καφε στο ποδι)

\photo{87.jpg}

Ποσο φοβερο ειναι αυτο πραγματικα... Ανθρωποι χωρις να με γνωριζουν μου ανοιγουν τα σπιτια τους, μου προσφερουν απλοχερα την παρεα και με βαζουν στις ζωες τους ετσι απλα. Με καλη διαθεση και πιστη ακομα στη καλοσυνη των ανθρωπων.
Φιλια, καλες προθεσεις, Εμπιστοσυνη... Ποσο πιο ομορφος θα ηταν ο κοσμος μας αν θυμομασταν ξανα τετοιες εννοιες που εχουμε χασει τωρα πια...

Καποτε κοιμομασταν με το κλειδι στη πορτα του σπιτιου για να μπορει να μπαινει ο γειτονας μας, ο φιλος, ο συγγενης. Τωρα ``καγκελα ασφαλειας μεταλουμιν'', συναγερμοι και διπλοκλειδωμα στην εξωπορτα της πολυκατοικιας...
Πας να μιλησεις σε εναν ανθρωπο στο δρομο και η καχυποψια κυριαρχει: τι θελει αυτος ο περιεργος απο εμενα; Γιατι μου μιλαει; Θελει να με κλεψει/βιασει/σκοτωσει;
Η απολυτη ειρωνια! Ο Ελλην ο ``φιλοξενος'': Εξω η φιλοξενια τυπου couchsurfing ειναι πολυ διαδεδομενη και ο κοσμος σε δεχεται στο σπιτι τους σαν να εισαι μερος της οικογενειας τους. Εδω οι περισσοτεροι το να φιλοξενησουν καποιον αγνωστο το θεωρουν αδιανοητο! ``Και που ξερω εγω οτι δεν θα με κλεψει; Που ξερω εγω αν θα με σκοτωσει το βραδυ;''

Ζουμε ζωες μιζερες, βαρετες και ασχημες γαμωτο -ποτισμενες απο το δηλητηριο του φοβου και της ελλειψης πιστης στο διπλανο μας- και για παρηγορια κανουμε shopping therapy: νεο κινητο, νεα τσαντα, νεες μποτες, νεο κρανος, νεο GSXR... Γιατι;
Αυτο εγω δεν το λεω ζωη: Καταντια ειναι. Λιγη θεληση χρειαζεται και μια δοση τρελας: να εμπιστευτουμε ξανα τον κοσμο γυρω μας. Να δωσουμε με χαμογελο σε οσους ζητανε ωστε οταν ζητησουμε και εμεις να εισπραξουμε το ιδιο... Τοσο πολλα ζηταω ρε παιδια; Εγω ειμαι ο τρελος η οι αλλοι χαζοι τελικα που δεν βλεπουν αυτα που βλεπω εγω;

``Παμε να δουμε τη πολη;'' η Christie με ξυπνησε απο τις σκεψεις μου. ``Φυγαμε''.
Η Αυρα ειχε και αυτη ...σπεσιαλ θεση για παρκινγκ κατω απο τις σκαλες του κουκλοσπιτου και ηταν κριμα να την ενοχλησω.

\photo{88.jpg}

Πηραμε λοιπον το -εννοειται Γαλλικο- αμαξακι της Christie για να παμε παρεα στη πολη.
Το Colmar μας περιμενε! (θεα με special guests τι αλλο; κουκλακια!)  

\photo{89.jpg}

Συντομα φτασαμε στη πολη... Το Colmar εδειχνε να στεκεται στο υψος της φημης του!

\photo{90.jpg}

Παντου μεγαλα παρκα με πολυ πρασινο, αψογη αισθητικη, καθαριοτητα και ατελειωτα παρτερια με λουλουδια...

\photo{91.jpg}

Οι ανθρωποι εδω ειναι πολυ ζωντανοι: φωνακλαδες, χαμογελαστοι, εγκαρδιοι. Καπως ετσι τους ειχα φανταστει τους Γαλατες...
(Μη ξεχνιομαστε! Ποδηλατο Πεζο ετσι; )

\photo{92.jpg}

Το Κολμάρ είναι πόλη της Αλσατίας στη βορειοανατολική Γαλλία. Η πόλη βρίσκεται στο Route de Vin (δρομο του κρασιου) της Γαλλιας αποτελώντας την πρωτεύουσα των περιφημων αλσατικών κρασιών.

Η πόλη ιδρύθηκε τον 9ο μ.Χ. αιώνα: Αναφέρεται για πρώτη φορά το 823, ως Columbarium -από όπου φαίνεται ότι προέρχεται και το σημερινό της όνομα- σε διάταγμα του Λουδοβίκου του Α' γιου του Καρλομαγνου, αλλά της δόθηκε το δικαίωμα να υφίσταται ως ελεύθερη αυτοκρατορική πόλη της Αγίας Ρωμαϊκής Αυτοκρατορίας το 1226 (civitatis). Το 1354 το Κολμάρ συμμετέχει στη δημιουργία της Δεκαπόλεως, μιας ομοσπονδίας δέκα αυτοκρατορικών πόλεων της Αλσατίας.

Το 1679 με τη συνθήκη του Νιμάγκεν το Κολμάρ αποδόθηκε στη Γαλλία και αποτέλεσε ``Βασιλική Γαλλική πόλη''. Στο καθεστώς αυτό παρέμεινε μέχρι το 1871, οπότε ολόκληρη η Αλσατία, με τη λήξη του Γαλλογερμανικού πολέμου, αποδόθηκε στη Γερμανία. Υπό γερμανική διοίκηση παρέμεινε μέχρι το τέλος του Α΄ Παγκοσμίου Πολέμου, οπότε με τη Συνθήκη των Βερσαλλιών η Αλσατία επεστράφη στη Γαλλία. Το Κολμάρ αναπτύχθηκε και το 1934 οι κάτοικοι φθάνουν σχεδόν τους 50.000.
Δυστυχώς όμως στον Β΄ Παγκόσμιο Πόλεμο οι Γερμανοί εισβάλλουν και καταλαμβάνουν εκ νέου την Αλσατία, την οποία προσαρτούν στο Γ' Ράιχ. Η πόλη υφίσταται καταστροφές μνημείων της και υφίσταται κατοχή με έντονα στοιχεία εκγερμανισμού και ναζιστικοποίησης. Η Γαλλία τελικά ανέκτησε εκ νέου τον έλεγχο της Αλσατίας ύστερα από τη μάχη του ``θύλακα του Κολμάρ'' το 1945.

\photo{93.jpg}

Η πόλη, βασιζόμενη στην αμπελοκαλλιέργεια αλλά και στη βιομηχανική της ανάπτυξη, ευημερεί από τον Πόλεμο ως σήμερα. Είναι μια πλούσια πόλη, με κύρια δύναμη της τον αναπτυσσόμενο τουρισμό, αλλά αποτελεί και έδρα γνωστών εταιριών όπως η Leitz και η Leibherr.
Βρίσκεται στο διαμέρισμα του Άνω Ρήνου σε απόσταση 68 χλμ. νοτιοδυτικά του Στρασβούργου. Ο ποταμός Ρήνος περνά 15 χλμ ανατολικά της, αλλά η πόλη συνδέεται με αυτόν μέσω ενός καναλιού, που συνδέει το Ρήνο με τον ποταμό Lauch, ο οποίος διασχίζει την πόλη, ενώ δυτικά της βρίσκεται η οροσειρά των Βοσγίων. Το θερμό και ξηρό μικροκλίμα της περιοχής του Κολμάρ είναι ιδιαίτερα ευνοϊκό για την καλλιέργεια αμπέλου, εξηγώντας γιατί η πόλη είναι το επίκεντρο παραγωγής και διακίνησης των Αλσατικών κρασιών.

\photo{94.jpg}

Ομως το κυριο αξιοθεατο εδω και ο λογος που η πολη ειχε αποκτησει τη φημη της πιο ομορφης στην Ευρωπη ηταν η ``la Petite Venise'' -η Μικρη Βενετια: μια ακρως γραφικη περιοχη στη καρδια της πολης που διασχιζεται απο μικρα καναλια του ποταμου Lauch, που στο παρελθον αποτελουσε το κεντρο κρεοπωλων, αλιεων και βυρσοδεψων.

\photo{95.jpg}

Η αρχιτεκτονικη εδω αντανακλα 8 αιωνες Γερμανικης και Γαλλικης σχεδιαστικης κουλτουρας η οποια εχει συνδιαστει περιφημα με τα τοπικα στυλιστικα εθιμα και υλικα κατασκευης: κοκκινος και κιτρινος ψαμμιτης και ξυλινα πλαισια με μεγαλα τμηματα τους να ειναι εκτεθειμενα ετσι ωστε να καταληγουν μερος του αρχιτεκτονικου σχεδιασμου.

\photo{96.jpg}
\photo{97.jpg}

Μπηκαμε μεσα στα μικρα σοκκακια και δεν ηξερα που να πρωτοκοιταξω!
Η ημερα ηταν γεματη φως και χαρουμενες φωνες απο τους επισκεπτες -το μερος εσφυζε απο ζωη.

\photo{98.jpg}

Πλακοστρωτα, μικρα καφε, bistrot...

\photo{99.jpg}
\photo{100.jpg}

...συντριβανια, καταπρασινα δεντρα, γεφυρακια και απειρα πολυχρωμα λουλουδια που αντικατοπτριζονταν στα νερα του μεγαλου καναλιου που κοβει την παλια πολη στα δυο.

\photo{101.jpg}
\photo{102.jpg}

Τα παλια παραδοσιακα σπιτια ηταν γεματα χρωματα βαμμενα ροζ, γαλαζια, κιτρινα, γαλαζια, καφε...
Μια σκετη πανδαισια!

\photo{103.jpg}
\photo{104.jpg}

Και η φοβερη λεπτομερεια: πολλα παραθυροφυλλα ειχαν στη μεση μια μικρη καρδια!

\photo{105.jpg}

Περπατησαμε για πολυ ωρα στα στενα, μη μπορωντας να χορτασω τις εικονες που εβλεπα. Σκεφτηκα το πως θα ηταν να ζουσα σε ενα τετοιο μερος...

\photo{106.jpg}

Θα κρατουσα αυτο τον ενθουσιασμο ανεπαφο; Ενας ντοπιος που μενει εδω αραγε αντιλαμβανεται οτι ζει σε ενα απο τα ομορφοτερα μερη του κοσμου η τα θεωρει ολα αυτα δεδομενα, αναξια λογου, βαρετα στην τελικη;

\photo{107.jpg}

Δεν ηξερα την απαντηση αλλα βλεποντας τη καθαριοτητα παντου και την πληρη ελλειψη γκραφιτι, αυτοκολλητων, αφισσων και λοιπων αισθητικων παρεμβασεων που συναντουσα συνηθως σε μεγαλες πολεις, μαλλον αν μη τι αλλο ηξεραν να σεβονται αυτο που εχουν...

\photo{108.jpg}

Ομως δεν ειχα δει ακομα αυτο για το οποιο ειχα ερθει...
Στο μυαλο μου απο τοτε που πρωτοδιαβασα ενα αρθρο για το Colmar υπηρχε μια συγκεκριμενη εικονα. Πολυχρωμα σπιτακια στις οχθες του ποταμου και απειρα λουλουδια να καθρεφτιζονται στα νερα.
Χαθηκαμε ξανα στα στενακια ψαχνοντας μια εικονα που μεχρι τοτε ειχα δει μονο στο αψυχο χαρτι.
Και ξαφνικα την ειδα εκει μπροστα μου. Ζωντανη. Και εχασα καθε μετρο συγκρισης... Απιστευτο!

\photo{109.jpg}

Τι να ελεγα για ΑΥΤΗ την εικονα που εβλεπα τωρα μπροστα μου; Πως να περιγραψω μια τοσο ομορφη εικονα που εμοιαζε να ειναι απο παραμυθι βγαλμενη;
Εμεινα απλα εκθαμβος να κοιταζω την ομορφια της φυσης και το μερακι των ανθρωπων...

\photo{110.jpg}
\photo{111.jpg}

Ειχε πλεον μεσημεριασει για τα καλα και η Christie με οδηγησε σε ενα ωραιο μαγαζακι που ηξερε για να δοκιμασω μια διασημη γαστρονομικη λιχουδια της Γαλλιας, το \href{http://en.wikipedia.org/wiki/Tarte_flamb%C3%A9e}{Tarte flambée}. Μια μεγαλη ανοιχτη πιτα με πολυ λεπτη ζυμη γεματη με φρεσκο λευκο τυρι, ζαμπον, ψιλοκομμενο κρεμυδι και κρεμα γαλακτος.

Ο θρυλος λεει οτι οι δημιουργοι αυτης της λιχουδιας ηταν γερμανοφωνοι αγροτες της Αλσατιας που συνηθιζαν να ψηνουν φουρνιστο ψωμι μια φορα τη βδομαδα. Η tart flambee χρησιμοποιουνταν για να δοκιμασουν τη θερμοκρασια του φουρνου. Στη καταλληλη θερμοκρασια ο φουρνος μπορουσε να ψησει τη ταρτα σε 1 με 2 λεπτα. Η κρουστα γυρω απο τη ταρτα κατεληγε να καει σχεδον απο τις φλογες (εξ' ου και flambee).

\photo{112.jpg}

Καθισα πισω και απολαμβανα τις στιγμες με ολες μου τις αισθησεις. Με τη πεντανοστιμη ταρτα, τη συνοδεια ενος εξαιρετικου λευκου κρασιου και τον υπεροχο ηλιο τωρα πραγματικα ημουν στο παραδεισο! Τι αλλο να ηθελα για να ειμαι ευτυχισμενος; Τα πιο ομορφα ειναι τα πιο μικρα παντα...

Η ωρα ομως πλεον ειχε περασει και οσο και να μην το ηθελα τωρα επρεπε να πηγαινω.
Ειχα 500 χιλιομετρα για το Leverkusen και ο καλος μου φιλος Κωστας με περιμενε. Δεν ελεγε να φτασω μεσανυχτα.
Φευγοντας αγορασα τα καθιερωμενα αυτοκολλητα μου και πηραμε το δρομο της επιστοφης για να ετοιμαστω. Στο βαθος οι καταπρασινοι αμπελωνες απλωνονταν αγερωχοι...

\photo{113.jpg}

Αρχισα να μαζευω τις βαλιτσες και να βαζω τη στολη αλλα ενας κομπος στο λαιμο δεν με αφηνε ησυχο. Η Christie ηταν ενας γλυκυτατος ανθρωπος -εξαιρετικα φιλοξενη, απιστευτα ζωντανη, χαρουμενη και αψογη οικοδεσποινα- και στεναχωριομουν πολυ που επρεπε να φυγω. Ομως δεν γινοταν αλλιως και το ηξερα. Η μοιρα του ταξιδιωτη. Τα παντα ρει και ουδεν μενει. 
Ως ειθισται, ανταλλαξαμε καποια μικρα δωρακια της χωρας μας ο καθενας και δωσαμε ραντεβου καπου καπως καποτε για το μελλον: 
A bientot Christie! Au revoir!

\photo{114.jpg}

Καβαλησα τη μηχανη. Τη σηκωσα απο το stand και τη ζυγισα πανω στα ποδια μου. Γνωριμη και οικεια η αισθηση, σαν μια υποσχεση οτι οσο ημασταν μαζι δεν μπορουσε να παει τιποτα στραβα. Εσυ και εγω. Εμεις.
Καπως ετσι θα ενοιωθε ενας καβαλαρης που ανεβαινει πανω στο αλογο του μετα απο καιρο. 
Κατεβασα τη ζελατινα και γυρισα το κλειδι στη μιζα. Ο ηχος του μοτερ γεμισε τη μικρη αυλη και ενοιωσα τη χαρα του ταξιδιου να διωχνει τη θλιψη της αποχωρησης. Η ωρα ηταν 3 το μεσημερι. Μπορει να ειχα αργησει λιγο αλλα ποτε δεν αντεχα να με κυνηγαει ο χρονος! Αυτες σημερα ηταν στιγμες πολυ σημαντικες για να τις ειχα χασει.

Ηλιος και 28 βαθμοι -ιδανικες συνθηκες. Επομενη σταση Χαιδελβεργη.

\photo{115.jpg}

Κατευθυνθηκα βορεια, παραλληλα με το ποταμο Ρηνο -το φυσικο συνορο μεταξυ Γαλλιας και Γερμανιας- απολαμβανοντας ενα μικρο κομματι της Γαλλικης επαρχιας. 
Σιγουρα ενα επομενο ταξιδι θα με βρει να εξερευνω καλυτερα αυτη την υπεροχη χωρα. 

\photo{116.jpg}

Λιγο εξω απο το Στρασβουργο η διαδρομη με οδηγησε πανω απο το πασιγνωστο ποταμο και ακριβως στη μεση του αλλο ενα συνορο: \textbf{Willkommen in Deutschland!}

\photo{117.jpg}

Αφησα πισω μου τους μικρους επαρχιακους δρομους καθως ηταν καιρος να γραψω χιλιομετρα. Η autobahn ανοιγοταν μπροστα μου και η Αυρα ηταν πανετοιμη να κανει αυτο για το οποιο φτιαχτηκε: ``παμε να κυνηγησουμε τον οριζοντα μωρο μου...''

\photo{118.jpg}

Μεχρι τη Χαιδελβεργη τα χιλιομετρα εφυγαν πολυ γρηγορα. 
Συναντησα το Σταυρο εξω απο τη δουλεια του και τα ειπαμε για λιγη ωρα. Ειχε ερθει στη Γερμανια πριν λιγο καιρο και θα καθοταν για ενα μικρο διαστημα δουλευοντας εκει. Πιασαμε κουβεντα περι ταξιδιων, νοοτροπιας των λαων και βεβαια -τι αλλο;- για μηχανες... Ηταν ενα γλυκυτατο παλικαρι, χαμογελαστος και πολυ ευγενικος. 
Πολυ ωραιος τυπος!

\photo{119.jpg}

Η πενταμορφη και το τερας η ``πως να χαλασετε μια ωραιοτατη φωτογραφιας μοτοσυκλετας''!

\photo{120.jpg}

Κοιταζε με λαχταρα τη μηχανη και με ρωτουσε για το ταξιδι που ειχα μπροστα μου, ομως εγω ηθελα να του πω οτι το ``ταξιδι'' που εκεινος ειχε ξεκινησει ηταν πολυ καλυτερο απο το δικο μου. Xαμογελασα και δεν ειπα τιποτα. Καποια πραγματα απλα δεν λεγονται με λογια.

Θα ηθελα να καθομουν περισσοτερο να τα λεγαμε αλλα ο χρονος οσο και να μην ηθελα πιεζε. 
``Κριμα ρε φιλε!'' -το ειδα οτι στεναχωρηθηκε αλλα ηξερα οτι καταλαβαινε. Βαρυ πραγμα η ξενιτια, ακομα και για λιγο καιρο. Το ειχα νοιωσει στο πετσι μου καλα τα χρονια που ζουσα στο βροχονησο...

Απο πανω μας ο ουρανος ειχε μαζεψει συννεφα αλλα κρατουσε ακομα. Λες να ημασταν τοσο τυχεροι μεχρι το Leverkusen; Ο Κωστης στο τηλεφωνο μου εκοψε τα ποδια: ``Βαλε αδιαβροχα. Εδω βρεχει ασταματητα.''

\photo{121.jpg}

Ο Σταυρος με συνοδεψε μεχρι την εξοδο της πολης και αποχαιρετηθηκαμε. Βγηκα στην εθνικη και ανοιξα το γκαζι. Τα απειλητικα γκριζα συννεφα μπροστα εδειχναν τις διαθεσεις του καιρου...

\photo{122.jpg}

Το φως χανοταν πλεον και η βροχη αρχισε να πεφτει ανελεητη. Κατεβασα κεφαλι και εσφιξα το δεξι γκριπ -ηταν ωρα να σοβαρεψουν τα πραγματα. 
Περνουσα μεσα απο τη καταιγιδα με 160+. Η Αυρα δεν καταλαβαινε τιποτα: παρατεταμενες στροφες με 140 και ευθειες με κλειδωμενο γκαζι -απολυτη σταθεροτητα σαν να ηταν το πιο φυσικο πραγμα του κοσμου.

Δεν μπορουσα ομως να πω το ιδιο και για τον αναβατη της.... Η διαδρομη ηταν δυσκολη. Πισω απο τη βρεγμενη και θολη ζελατινα προσπαθουσα να δω μεσα στο σκοταδι, ενω το κρυο πλεον να ειχε αρχισει να γινεται πολυ ενοχλητικο παρα τα αδιαβροχα και τη στολη. Ειχα ηδη κανει 350 χιλιομετρα απο το μεσημερι και αρχισα να νοιωθω τη κουραση της ημερας βαρια πανω μου. 
Σκυφτος πισω απο τα φερινγκ, εβλεπα τα χιλιομετρα στο οδομετρο να περνανε ενα ενα και δυσκολα και το γκαζι ανοιγε ακομα πιο πολυ για να φτασω συντομα. Ευτυχως αν υπηρχε ενα μερος στο κοσμο να πηγαινω ετσι μεσα στη βροχη ηταν εδω στην autobahn.

\photo{123.jpg}

Η πινακιδα που εγραφε Leverkusen ηρθε σαν σανιδα σωτηριας. Ειχα φτασει. 

\photo{124.jpg}

Στο σπιτι ο Κωστας και η γλυκυτατη Σαρα με υποδεχτηκαν με μεγαλη χαρα. 
Δεν θα μπορουσα να ζητησω κατι καλυτερο για το παγωμενο και βρεγμενο τομαρι μου: δυο ζεστα χαμογελα και μια στεγη πανω απο το κεφαλι μου αυτη τη κρυα νυχτα.
Κατσαμε να φαμε και το βραδυ περασε με κουβεντες για τα παντα: για τα πραγματα στην Ελλαδα, στη Γερμανια, για ταξιδια περασμενα και επομενα, για ονειρα και στοχους... 

Εγω ομως ενοιωθα πραγματικα πολυ χαρουμενος και γιατι επιτελους συναντουσα το Κωστα απο κοντα! Ηταν κατι που ηθελα να γινει εδω και χρονια γιατι πραγματικα με ειχε σκλαβωσει με την καλοσυνη του, την ευγενεια και το χαρακτηρα του. 

Βλεπετε στα τελη του 2008 εψαχνα να αγορασω VFR. Τα ελαχιστα που ειχα βρει στην Ελλαδα ηταν για πεταμα οποτε πλεον κοιτουσα απο Γερμανια. Ειχα βρει μια σε αριστη κατασταση σε μια αντιπροσωπεια Honda στη κεντρικη Γερμανια περιπου μια ωρα ανατολικα του Leverkusen, ομως δεν μπορουσα να παω να τη δω απο κοντα. 
Ετσι ενας φιλος προσφερθηκε να παει αυτος 200 χιλιομετρα πηγαινελα να τη δει και να τη δοκιμασει για παρτη μου, οπως και εγινε τελικα. 
Αυτος ο φιλος ηταν ο Κωστας και αυτη η VFR ηταν η Αυρα μου...

Και να που τωρα ημασταν εδω σε ενα ομορφο σπιτι συνοδεια μπυριτσας και τσιμπολογηματος να τα λεμε ενω εξω η βροχα επιπτε ραιτ θρου...
Ομορφα...

\chapter{Day 4 -- Leverkusen (D) - Aarhus (DK) - 740km}

Ολο το βραδυ δεν σταματησε να βρεχει. Το πρωτο φως της ημερας, χλωμο και ξεθωριασμενο μπηκε απο το παραθυρο απεναντι μου και ανοιξα τα ματια. Κοιταξα εξω. Βρεγμενοι, ερημοι δρομοι και ο ουρανος ειχε αυτο το βαρυ γκριζο της βροχης που εχει ξεχασει ποτε αρχισε και ποτε θα τελειωσει. Τιποτα δεν θυμιζε καλοκαιρι εδω.

Διπλα μου ομως ειχα την πιο ομορφη παρεα... Ο Φιοκο, ο υπεροχος καταλευκος σκυλαρος του Κωστα, με κοιτουσε αγουροξυπνημενος ολο απορια: ``Ποιος εισαι εσυ παλι;''

\photo{125.jpg}

Το σπιτι ηταν πολυ ομορφο. Αν το σπιτι της Christie ηταν η αποθεωση της διακοσμησης και των χρωματων, το σπιτι των παιδιων ηταν η χαρα της απλοτητας, της ζεστασιας, της χυμα ελευθεριας και του cool. 
Επιπλα παλια καθε ειδους και στυλ παντου, ενας μεγαλος ανετος καναπες, βιβλια και παλια περιοδικα μηχανης στα ραφια, ανταλλακτικα, βαλιτσες και αξεσουαρ της μοτοσυκλετας, computer, καλωδια, το μεγαλο στερεοφωνικο για τις μουσικες, το ηλεκτρικο μπασο και ο ενισχυτης στη γωνια, πολυχρωμοι πινακες στους τοιχους... 
Ο χωρος απεπνεε μια φοβερη ελευθερια, ενα cool ``χυμα'', σαν να εβγαζε με αυθαδεια τη γλωσσα στο δηθεν, στα επωνυμα design φροντισμενων, αγρατζουνιστων ιλουστρασιον καταλογων χωρις ουτε μια ατελεια, ετη φωτος απο τα προκατ ΙΚΕΑ αυτου του κοσμου. Εδω ζουσαν και ανεπνεαν ροκ ανθρωποι φιλε μου!

Ξεχασα το κωλοκαιρο εξω και χαμογελουσα σαν χαζος! Ποσο χαιρομουν που ημουν εδω. Τι υπεροχο συναισθημα να ξυπνας σε ενα τοσο ζεστο περιβαλλον και κοντα σε τοσο καλους φιλους οσο ηταν ο Κωστας και η Σαρα! 

Κατα φωνη. Tα παιδια ειχαν μολις ξυπνησει και με το ζεστο πρωινο καφεδακι στα χερια πιασαμε τη κουβεντα για τη ζωη εδω πανω... 
Βλεποντας τωρα τα πραγματα απο κοντα και παιρνοντας μια ελαχιστη γευση απο τη καθημερινοτητα εδω, καταλαβα οτι οι απλοι ανθρωποι ζουσαν στις ιδιες αγωνιες που ζουμε και εμεις στην Ελλαδα. Οι στημενες ειδησεις στο χαζοκουτι δειχνουν μονο τις εικονες που θελουν να προβαλλουν: ο Γερμανος ο πλουσιος κυριλε θειος, ο Ελληνας ο φτωχος συγγενης. 

Τελικα η ζωη στην Γερμανια δεν ηταν οπως την ειχα πριν στο μυαλο μου. 
Μειωσεις μισθων, ακριβεια, αυξανομενη ανεργια, ενα αυριο με ενα τεραστιο ερωτηματικο να κρεμεται απο πανω και μελλον χωρις εξασφαλιση... Μοιαζει γνωριμο το σκηνικο; 
Και ολα αυτα σε μια χωρα που δεν ειναι η δικη σου -διπλο το βαρος. Η επιστροφη ειναι παντα μια φλογα που καιει μεσα σου και σε τρωει -το ηξερα καλα. Ο Κωστας μου ειπε οτι σκεφτηκαν πολυ να γυρισουν πισω ομως αυτα τοτε. Πριν τη κριση. Οχι τωρα. 
Τωρα υπομονη, ανασυγκροτηση, αναμονη για κατι καλυτερο. Οπως ολοι μας φιλε -κι εκει και εδω. 
Εξαλλου αυτα τα παιδια δεν ειχαν να φοβουνται τιποτα -τα πιο σημαντικα συστατικα ηταν εκει: υγεια, αγαπη, φροντιδα, ελπιδα, παθος. Οσο εχεις αυτα δεν εχεις να φοβασαι τιποτα και κανεναν...

Η κουβεντα συντομα γυρισε σε πιο ευχαριστα θεματα καθως αρχισαμε να μιλαμε για το ταξιδι μου. 
Ο Κωστας ηταν παλιος γνωριμος στα Νορβηγικα λημερια, εχοντας κανει αρκετα ταξιδια στη χωρα των Βικινγκ στο παρελθον, οποτε οι συμβουλες του ηταν πολυτιμες! Με τους χαρτες ανοιχτους λοιπον αρχισαμε να χαραζουμε διαδρομες καθως μου προτεινε ποια μερη αξιζει να δω και ποια να αποφυγω -αν και σε μια τοσο εντυπωσιακη χωρα δεν νομιζω οτι θα μπορουσε να υπαρχει μερος βαρετο η αναξιο αναφορας...

Ο Κωστας μιλουσε με τοσο ενθουσιασμο για οσα ειχα να δω μπροστα μου: ``Α ρε Νικολα, να μην ειχα τη δουλεια στο μαγαζι και θα ερχομουν μαζι σου ρε φιλε.''
Αυτο θα ηταν πραγματικα οτι καλυτερο αλλα η λογικη υπαγορευε οτι δεν μπορουσε να γινει. Αυτη τη φορα το ταξιδι θα γινοταν σολο. 
Καθως κοιτουσα τις γραμμες στο χαρτη αρχισα να νοιωθω πιεση για αυτο που με περιμενε μπροστα μου. Τι παω να κανω....; Θα το καταφερνα ενα τοσο μακρυνο ταξιδι μονος μου; 

Απομακρυνα αυτες τις σκεψεις απο το μυαλο μου και επικεντρωθηκα στη σημερινη διαδρομη. Το προγραμμα ηταν απλο: θα καναμε μια αλλαγη ελαστικων στην Αυρα (καθως αυτα που ειχα επανω ειχαν ηδη 15.000 και δεν θα αντεχαν αλλα 11.000) και μετα αναχωρηση για το Aarhus της Δανιας περιπου στα μισα της χωρας. 
740 χιλιομετρα. 
Πολλα. 
Ομως το καραβι για Νορβηγια εφευγε αυριο το πρωι απο το Hirtshals στο βορειο ακρο της Δανιας και αν ηθελα να ειμαι εκει επρεπε σημερα να φαω οσο μεγαλυτερο μερος της αποστασης γινοταν.

\photo{126.jpg}

Το ηξερα οτι αυτη η μερα οτι θα ηταν ζορικη. Ομως αν ηξερα τι θα με περιμενε θα κλειδωνομουν στη κοντινοτερη μπυραρια και δεν θα εβγαινα απο εκει μεχρι τη Δευτερα Παρουσια... 

Ο καφες ειχε τελειωσει απο ωρα και ηταν καιρος να ξεκινησουμε τη μερα.
Τα λαστιχα της μηχανης τα ειχα παραγγειλει ηδη προ ημερων σε Γερμανικο eshop και ηταν ηδη στο σπιτι του Κωστα. Πηραμε τη μηχανη και πηγαμε στο μικρο γκαραζακι που φιλοξενουσε τη δικια του κουκλα και ενα πανεμορφο FJR του πεθερου του και ξεκινησαμε να ....βγαλουμε τις ροδες; 

- Ρε Κωστη δεν χρειαζοταν να μπεις σε τετοια φασαρια ρε φιλε!"
- Δεν ειναι φασαρια ρε συ Νικο, να! {τσακ, τσακ} Οριστε! Βγηκαν οι ροδες!

\photo{127.jpg}

Μεσα σε ελαχιστο χρονο και με το πιο μεγαλο χαμογελο ο Κωστας ειχε λυσει εξατμιση, δαγκανες, αξονες και ειχαμε τους τροχους ανα χειρας. Τα φορτωσαμε ολα μεσα στο αυτοκινητο του και τα αφησαμε σε ενα γνωστο του λαστιχα. Aν και βλοσσυρος και παρα το κουσουρι του να ασχολειται κυριως με αγροτικα μηχανηματα (aka BMW moto) εδειχνε να ξερει τι κανει: ``Σε μερικες ωρες θα ειναι ετοιμα, ελατε να τα παρετε.''

Στο μεταξυ ομως εμεις δεν θα καθομασταν με σταυρωμενα χερια. 
Ειχαμε μια ακομα συναντηση που περιμενα απο καιρο: θα πηγαιναμε στο κοντινο Solingen να γνωρισω απο κοντα το μουρλοκομειο εκ Θεσσαλονικης Γιαννη Παναγιωτιδη, παγκοσμιως γνωστο Μπεμβεδοφαγο με αδυναμια στις πιτσες, στις στροφες και μια περιεργη ταση να περναει απο πανω οτι supersport μηχανη πεσει στο δρομο του! 
Ο Γιαννης ηταν μεγαλη μορφη! Μας περιμενε στο μαγαζι του με μεγαλη χαρα και με εκανε να νοιωσω σαν στο σπιτι μου! Απιστευτο χιουμορ, τρελαρας, πολυ large - σαν τον Elvis ηταν!  

Το μικρο μαγαζι του ηταν πολυ ομορφο και τακτικο, με προσεγμενο φαγητο και μερακι. Πολλα χρονια εκει, δουλευε σκληρα μαζι με την γλυκυτατη συζυγο του αλλα οι κοποι τους ειχαν αποδοσει καρπους. Ειχανε φτιαξει μια πολυ ομορφη οικογενεια, ειχαν μια δικια τους στεγη, μια δικια τους δουλεια. 

\photo{128.jpg}

Ελληνες της διασπορας. Δυσκολιες αλλα και ονειρα, στο internet radio μονιμα Ελληνικος σταθμος, στο παγκο το φορητο με φωτογραφιες απο νησια της Ελλαδας: ο νοστος για τη πατριδα, η πικρα και το παθος... Ολα εκει...

Ενοιωθα τον πονο τους που ηταν μακρυα απο οσα θεωρουσαν οικεια και ας ελειπαν τοσα χρονια, και ας ζουσαν σε πολυ καλυτερες συνθηκες απο οτι εμεις εδω σημερα. 
Ο Κωστας γυρισε και μου ειπε με παραπονο: ``Εδω δεν ειναι η χωρα της επαγγελιας φιλε. Η Γερμανια ειναι φυλακη δουλειας.''
Τον καταλαβαινα αλλα και ο Ελληνας που βρισκεται στη χωρα του χωρις δουλεια ειναι σε εξισου ασχημη ``φυλακη'' ομως... 
Για αλλη μια φορα σκεφτηκα το πως καταληξαμε να πληρωνουμε σπασμενα προηγουμενων γενεων και ενος αδηφαγου συστηματος ακρατης καταναλωσης πορων, χρηματων, ανθρωπων, αξιων... Σαν να ειχε γινει το μεγαλυτερο παρτυ ολων των εποχων και εμεις που δεν μας καλεσαν να επρεπε να πληρωσουμε τωρα το λογαριασμο.

Δεν ηταν ζωη αυτο που μας ειχαν σπρωξει να ζουμε τωρα -κλεισμενοι στα κουτια μας, φοβισμενοι, χωρις θεληση και οικονομικα μεσα να ξεφυγουμε και να παμε λιγο παραπερα. Ο ενας εδω ο αλλος εκει και ολοι καθηλωμενοι. Ημασταν οι μπαταριες του ματριξ και αυτο εμενα τουλαχιστον μου καθοταν πολυ στραβα! Αν θα μπορουσα να παρακινησω εστω και εναν ανθρωπο να ανοιξει τα ματια και να παρει ενα ρισκο για κατι καλυτερο θα αξιζε τον κοπο.
Τουλαχιστον αυτα τα παιδια εδω το ρισκο τους το ειχαν παρει και οποιο και να ηταν το αποτελεσμα, σημασια εχει παντα η προσπαθεια.

Χαμενος οπως παντα στο κοσμο μου, ο Γιαννης με συνεφερε με χαμογελο. 
- Εχεις φαει πιτσα γυρο;
- Πιτσα ..τι;
- Καλα ρε, πλακα με κανεις, δεν εχεις φαει πιτσα με γυρο; Κατσε να δεις. λεει και φερνει 2 μεγαλες λαχταριστες πιτσες με ...γυρο επανω! Μουρλια! Ο Γιαννης με ειχε κατασκλαβωσει με την ευγενεια και την περιποιηση του!

Η ωρα περνουσε και ηταν καιρος να πηγαινουμε, οχι ομως πριν τις απαραιτητες αναμνηστικες φωτογραφιες. Γεια σας ρε παιδια! Να ειστε παντα καλα, χαμογελαστοι και να εχετε οτι καλυτερο στο διαβα σας... Τετοιοι φιλοι ειναι ο αληθινος θησαυρος της ζωης!

\photo{129.jpg}

Στο μαγαζι ο Γερμανος ειχε αλλαξει τα λαστιχα οχι ομως αναιμακτα: στο μεσα μερος της πισω ζαντας ειχε σπασει ενα μικρο κομματακι χρωματος μεχρι το μεταλλο, ισως χτυποντας το με καποιο εργαλειο οταν πηγε να βαλει τα βαρακια της ζυγοσταθμισης. Και να σκεφτει κανεις οτι η ζαντα ειχε γινει βαφη πριν ενα χρονο! Χαλαστηκα λιγο αλλα δεν εδωσα συνεχεια -μικρο το κακο εξαλλου.

Εξω η βροχη δεν ελεγε να κοψει. ``Ριχνει ετσι εναμισι μηνα τωρα. Ειναι απο τα πιο βροχερα καλοκαιρια της χωρας εδω και χρονια.''
Γ@μω τη τυχη μου. Κανω το πιο μεγαλο ταξιδι μου τη χρονια που ο Θεος αποφασισε να ξεπλυνει τις αμαρτιες των ανθρωπων. Κυριολεκτικα.

Ο Γιαννης στο μαγαζι με ειχε προσγειωσει πολυ αποτομα σε μια ζορικη πραγματικοτητα:

\dialogue{Ποσα χιλιομετρα εχεις κανει μεχρι εδω φιλε;}
\dialogue{Ε περιπου 2 με 2μισι χιλιαδες.}
\dialogue{Και ποσα εχεις συνολο; 12.000; Πωπωω ρε φιλε, δηλαδη εχεις αλλα 10.000 χιλιομετρα μπροστα σου;;; Χαρα στο κουραγιο σου!}
\dialogue{........}

Σκατα. Ενοιωσα την πραγματικοτητα του ταξιδιου που υψωνοταν τωρα μπροστα μου θεορατο και λιγοψυχησα. Δεν ηθελα να συνεχισω. Η σκεψη να εμενα εκει για μερικες μερες και τελικα να γυριζα πισω δεν με αφηνε σε ησυχια. Ευτυχως ο Κωστας ηταν εκει να με στηριξει με το καλοσυνατο και χαμογελαστο τροπο του: ``Νικολα, μην σκας φιλε! Κανεις ενα υπεροχο ταξιδι. Πηγαινε και ζηστο. Ολα θα πανε καλα.''

Η ωρα ειχε παει 5 το απογευμα και ο Κωστας επρεπε να παει στη δουλεια. Πηρα τηλεφωνο στο hostel που ειχα κλεισει στο Aarhus οτι θα αργουσα: ``Κανενα προβλημα. Θα αφησουμε τη καρτα για το δωματιο σε μια θυριδα στην εισοδο. Πατηστε τον ταδε κωδικο για να το παρετε.''

Ξεκινησα να μαζευω τα πραγματα δεν ενοιωθα ομως ομορφα. Ειχα καθυστερησει ηδη πολυ να ξεκινησω και ειχα 740 ολοκληρα χιλιομετρα μπροστα μου με αυτο το κωλοκαιρο; 
Εχοντας αποφασισει οτι μου εφτανε η πιεση του ταξιδιου και μιας που δεν ηθελα να εχω και το ζορι να σφηνωνω καθε μερα τα πραγματα στις βαλιτσες, δανειστηκα ενα τεραστιο αδιαβροχο σακο απο τον Κωστα για να βαζω μεσα ευκολα τη σκηνη, τον υπνοσακο και λοιπα πραγματα πρωτης αναγκης. 
Για αλλη μια φορα με σκλαβωνε αυτο το παλικαρι! Ομως ενοιωθα ασχημα να του παρω το σακκο στην Ελλαδα παροτι θα τον εστελνα πισω με ταχυδρομειο.

Αποχαιρετησα τον Κωστα και τη Σαρα και καβαλησα, αλλα αυτη τη φορα κανοντας μεγαλη προσπαθεια να πεισω τον εαυτο μου να φυγει... To ταξιδι ηταν μπροστα μου αλλα για αλλη μια φορα επρεπε να αφησω πισω αγαπημενα ατομα...

\photo{130.jpg}

Βγηκα στην εθνικη οδο αλλα κατι με ετρωγε. Η βροχη ειχε σταματησει τωρα και ενας απογευματινος ηλιος εφτιαχνε το ιδανικο σκηνικο για να ταξιδεψω, ομως δεν ενοιωθα καλα. Ηταν και αυτος ο τεραστιος σακκος πισω μου...

Και τοτε μου ηρθε η ιδεα! Το πρωι ειχα δει ενα φυλλαδιο απο ενα μεγαλο μαγαζι με ειδη μοτο και ειχε ενα πολυ ωραιο μακροστενο αδιαβροχο σακκιδιο με ανοιγμα απο πανω και το ειχα βαλει στο ματι αλλα λογω χρονου με τον Κωστα δεν ειχαμε παει να το παρουμε.
Ε θα πηγαινα να το παρω! Θα εχανα τουλαχιστον αλλη μια ωρα και ηδη ειχα αργησει τραγικα να ξεκινησω! Αυτο θα ηταν εντελως τρελο! Ω ναι!

Εκανα δεξια στο πρωτο βενζιναδικο που βρηκα για να βρω τη διευθυνση του μαγαζιου. Εψαχνα στα μενου του GPS οταν ακουσα διπλα μου ενα μεγαλο τετρακυλιδρο να πλησιαζει. Γυρισα και ειδα ενα Suzuki RF 900 τιγκα φορτωμενο με ενα νεαρο Γερμανο, τον Ion. Μου επιασε κουβεντα για το απο που ειμαι και που παω. 
Αυτος γυριζε απο tour 2 βδομαδων μονος σε Ισπανια - Γαλλια. Ωραιος...
Του ειπα για το ταξιδι μου μεχρι εκει, τις αναποδιες με το πεσιμο στο πλοιο, το ασχημα πρησμενο δαχτυλο στο δεξι χερι που δεν ελεγε να περασει, τη ζημια στη ζαντα, την ατελειωτη βροχη και τα ασχημα προγνωστικα...

Ο Ion γυρισε και μου εδειξε προς την μηχανη του: ενας απο τους ιμαντες που εδεναν τα πραγματα περνουσε μεσα απο τα πλαστικα της ουρας που ηταν κομμενα καθετα περα ως περα!
Μου ειπε: "Το βλεπεις αυτο; Αυτο εγινε τη πρωτη μερα του ταξιδιου μου. Ο ιμαντας πιαστηκε στη πισω ροδα και εκοψε τελειως την ουρα της μηχανης στα δυο. Με χαλασε παρα πολυ αλλα μετα σκεφτηκα οτι αφου δεν σκοτωθηκα ολα τα αλλα περισσευαν. Ετσι πιεσα τον εαυτο μου και πηγα στο ταξιδι και μαλιστα περασα υπεροχα! \textbf{``So remember. Keep riding and the sun will shine again!''}

Ειχε δικιο! Μπροστα μου ειχα μια μεγαλη περιπετεια και θα την χαιρομουν ΟΤΙ και να συνεβαινε! Αποχαιρετησα το νεο μου φιλο και πηγα στο μαγαζι για το σακκο. Το κοστος του ηταν ελαχιστο και σιγουρα λιγοτερο απο το ταχυδρομειο του αλλου σακκου απο Ελλαδα στη Γερμανια. 
Λιγη ωρα μετα βρισκομουν παλι στο μαγαζι του φιλου. Ο Γιαννης γουρλωσε με εκπληξη τα ματια του σαν να εβλεπε εξωγηινο. 

\dialogue{Επ! Νικο;; Δεν εφυγες ακομα ρε φιλε;}
\dialogue{Οχι, αργησαμε λιγο με τα λαστιχα και το μαζεμα, συν οτι αποφασισα να παρω και ενα σακκο για να μην παρω του Κωστα μαζι, οποτε ηρθα να στον αφησω να του τον δωσεις.}
\dialogue{Και θα φυγεις; Τωρα;}
\dialogue{Ναι, πρεπει να ειμαι αυριο στο Hirtshals στη Δανια.}\\

\noindent Με κοιταξε σαν εξωγηινο.\\

\dialogue{Εισαι τελειως τρελος ρε αδερφε! Πας να κανεις τετοια αποσταση οπως κανουμε εμεις μια βολτιτσα παραδιπλα;!; Πως αντεχεις; Ελα παμε λιγο εξω να χαζεψω το εργαλειο.}\\

Ο Γιαννης εκατσε πανω στην Αυρα ποζαροντας με καμαρι και η συζυγος πανταχου παρουσα εβγαλε μερικες ακομα αναμνηστικες φωτογραφιες προτου αναχωρησω και παλι, αυτη τη φορα ομως τα πραγματα ηταν πολυ δυσκολα. Η οθονη εγραφε 748 χλμ και η ωρα ηταν πλεον 7.30 το βραδυ! Αποψε θα χρειαζοταν να ξεπερασω ολα μου τα ορια... Το να χασω το πλοιο για Νορβηγια την επομενη μερα απλα δεν ηταν επιλογη. 
Στα ακουστικα εβαλα το \href{http://www.youtube.com/watch?v=g5vYEAUIcLM}{Highway of light} των Madrugada τερμα...

Αποψε θα ημουν στη Δανια η πουθενα...

\photo{131.jpg}

Αν ολα αυτα δεν ηταν αρκετα, στην autobahn με περιμενε ενα νεο προβλημα: εργα. Πολλα εργα. Οι τρεις λωριδες της εθνικης γινονταν μια και κατι και τα αυτοκινητα εκαναν ουρες χιλιομετρων! Στριμωχνα την μηχανη στο περισσευμα της λωριδας που ηταν διαθεσιμη, αναμεσα στο στηθαιο του δρομου και τα σχεδον σταματημενα αυτοκινητα και παρακαλουσα τους θεους του ταξιδιου να μην ανοιγε καποιος τη πορτα του... Οι ταχυτητες ηταν αναγκαστικα μικρες τουλαχιστον μεχρι να βγουμε απο τα εργα και να ανοιξουμε παλι γκαζι μεχρι το επομενο σημειο μποτιλιαρισματος.

Περασαν πανω απο 350 δυσκολα χιλιομετρα ετσι. ``Κανενα προβλημα, εχεις αλλα 400 να τα απολαυσεις μεγαλε'' σκεφτηκα και γελαγα με τον εαυτο μου!
Οταν οι διαδρομες ειναι τοσο μεγαλες και εισαι ηδη απο ωρα στο δρομο τα χιλιομετρα φευγουν βασανιστικα...
Η κουραση ειχε κανει εντονη τη παρουσια της απο ωρα και οι δυναμεις μου με εγκατελλειπαν. Κρατουσα σταθερη την Αυρα στην ευθεια του δρομου και προσπαθουσα με καθε τροπο να κραταω τον εαυτο μου σε ενγρηγορση ενω το μυαλο ειχε κατεβασει τις ασφαλειες προ πολλου.

Τα μεσανυχτα με βρηκαν εξω απο το λιμανι του Αμβουργου, παγωμενο, εξαντλημενο, σε κατασταση οριακη. Το κρυο της νυχτας ηταν πλεον αφορητο. Πονουσα παντου, χερια, ποδια, αυχενας, κρυωνα παρα τα αδιαβροχα και το μονο που ηθελα να κανω ηταν να κοιμηθω. Ομως δεν γινοταν ακομα -ειχα 300 χιλιομετρα μπροστα μου και επρεπε να βγουν.

Οι σκεψεις μου παραξενες αυτη την ωρα, μπλεγμενες σε μια ομιχλη σαν αυτη που τυλιγε το εφιαλτικο, παγωμενο βιομηχανικο σκηνικο γυρω μου.

Οι τεραστιοι γερανοι του λιμανιου μου εφεραν στο μυαλο την διασημη εικονα των \href{http://upload.wikimedia.org/wikipedia/en/0/03/Pinkfloydhammers.jpg}{παρελαυνοντων σφυριων των Pink Floyd} και στο mp3 ξεκινησε το \href{http://www.youtube.com/watch?v=jySUpMqmzd4}{Comfortably Numb} -Tοσο ταιριαστο υπο τις συνθηκες...

\begin{verse}
Hello,
Is there anybody in there?
Just nod if you can hear me
Is there anyone home?
\end{verse}

Η συνεχεια ηταν θαμπη και συγκεκχυμενη μεσα στο μυαλο μου -εικονες φευγαλεες περνουσαν και χανονταν μεσα σε μια ομιχλη ζαλης, κουρασης και υπνηλιας. Η εξοδος απο την Autobahn... Τα συνορα με Δανια που ολο και περιμενα και δεν ερχονταν ποτε... Ενας παραξενος σκουρος μπλε ουρανος... Μα τετοια ωρα; Ποτε νυχτωνει εντελως; 

Πλησιαζοντας τα συνορα Γερμανιας-Δανιας, 10.42μμ...

\photo{134.jpg}

Ατελειωτες ευθειες χιλιομετρων αναμεσα σε δαση... Ανεφοδιασμοι για βενζινη σε πρατηρια ...καπου... Σκοταδι πια και ταξιδι στο πουθενα...

\photo{135.jpg}

Η θερμοκρασια στο οργανο φλερταρε πλεον με μονοψηφια νουμερα αλλα τωρα καλοδεχομουν τον παγωμενο αερα γιατι με κρατουσε ξυπνιο. Καταφερα να κραταω ενα σταθερο ρυθμο και η Αυρα καταπιε τα τελευταια χιλιομετρα γρηγορα... 
Το Aarhus ηταν μπροστα μου -τα ειχα καταφερει! Δρομοι ερημοι, λουσμενοι σε ενα κιτρινο φως απο τις νυχτερινες λαμπες, φαναρια αναβοσβηναν πορτοκαλι περιμενοντας το ξημερωμα για να ζωντανεψει παλι η πολη... 
Το GPS με οδηγησε γρηγορα στο hostel στο κεντρο. 
Παρκαρα τη μηχανη στο πεζοδρομο κοντα στα τελευταια κλαμπακια που ακομα ειχαν ζωη και ανεβηκα στο πολυ μοντερνο και προσεγμενο ξενωνα. 
Επεσα στο κρεββατι με ενα μεγαλο χαμογελο. Το ρολοι εγραφε 2.30 το πρωι. 
Μπορει να ειχα να σηκωθω σε 4 ωρες αλλα δεν με ενοιαζε γιατι σε λιγο ξημερωνε η μεγαλη μερα: η Νορβηγια με περιμενε μυστηριωδης, αγνωστη! 
Το Ονειρο ξεκινουσε... Και ουτε στα πιο τρελα μου ονειρα θα μπορουσα να ειχα φανταστει τι θα εβλεπα στη συνεχεια...

\chapter{Day 5 -- Aarhus - Hirtshals (DK) - Lysebotn (NO) - 540 km}

Το φως που εμπαινε απο το μισοκλειστο παραθυρο δεν αφηνε περιθωρια διαφωνιας: σηκω.
Δεν θα διαφωνουσα. Το σωμα μου ειχε πλεον αρχισει να συνηθιζει στις καταπονησεις του ταξιδιου και οι 4 ωρες υπνου μου ηταν αρκετες.
Σημερα εφτανα στο ακρωτηρι Hirtshals της Δανιας και μετα... ολα τα ενδεχομενα ανοιχτα. Ολα αγνωστα, terra incognita.
Πως ειναι να πλαθεις μια εικονα στο μυαλο σου για εκεινη, τη μια Μοναδικη Στιγμη που ονειρευεσαι; Για την ωρα που περνας τη πυλη με τη ροζαλια, για τη στιγμη που γυρνας το κλειδι στη μιζα της πρωτης σου Μηχανης, για τη στιγμη που Εκεινη ερχεται κοντα και την παιρνεις στην αγκαλια σου, για την στιγμη που αντικρυζεις τα ματια του πρωτου σου παιδιου, για τη στιγμη που κανεις το ονειρο επιτελους πραξη;

Και να... Να που τωρα ημουν στο ηλιολουστο μικρο δωματιακι ενος ξενωνα στη Δανια, σε αποσταση αναπνοης απο αυτο που τοσα χρονια στριφογυριζα στο μυαλο μου. Ενα τετοιο ταξιδι. Απο τα Μεγαλα. Τα καλα.

\photo{136.jpg}

``Εχουμε μεγαλη μερα σημερα'' ειπα και χαμογελασα. Μαζεψα τα πραγματα και κατεβηκα στο σαλονι για το πρωινο. 
Ο ξενωνας \href{http://www.citysleep-in.dk/}{City Sleep-in} ηταν πραγματικα εξαιρετικος. Πολυ νεανικος, πολυχρωμος, καθαρος, με αυτη την αψογη απερριτη βορεια αισθητικη που συνδιαζε το συγχρονο με το παραδοσιακο, πετρα, ξυλο και με μια πολυ ομορφη αυλη για τους ενοικους. 
Ακομα πιο εντυπωσιακο ομως ηταν το ποσο οικονομικος ηταν για τα μετρα της Δανιας: μολις 25 ευρω τη βραδια! 
Αν τυχει και βρεθειτε στα περιξ σας τον συστηνω ανεπιφυλακτα!

\photo{137.jpg}
\photo{138.jpg}

Ο μπουφες ηταν ολα τα λεφτα! Πρωινο με βιολογικα προιοντα, ανοιχτα χρωματα, φυσικα υλικα και μοντερνες πινελιες απο ενα ...κανονικοτατο φαναρι της τροχαιας! Λατρευω τους ξενωνες! Τρελα μερη με εξισου τρελους ανθρωπους.

\photo{139.jpg}

Λατρευω ομως και ενα καλο πρωινο, οποτε οταν βρισκομαι σε τετοια μερη δεν χανω ευκαιρια να το απολαυσω! 
Η ωρα ηταν 7 παρα και το πλοιο εφευγε σε 3 ωρες αλλα δεν αγχωνομουν. Εμεναν 130 χιλιομετρα μεχρι το Hirtshals και με ευθειες της εθνικης οδου θα ημουν εκει πολυ ευκολα. 
Ποσο μου αρεσε αυτη η στιγμη! Εξω ο ηλιος ελαμπε ηδη πανω σε ενα καταγαλανο ουρανο, βρισκομουν σε ενα κουκλιστικο περιβαλλον που μεχρι προτινος εβλεπα μονο σε καταλογους του ΙΚΕΑ και στο τραπεζι ειχα οτι θα μπορουσα να θελω για ενα βασιλικο πρωινο -γιαουρτι φρουτοχυμο (πολυ διαδεδομενο ποτο στη Δανια), φρεσκα φρουτα, καφε, ομελετα, σαντουιτς...

Οσο καθομουν εκει και χαζευα γυρω τις παρεες των ενοικων να τιτιβιζουν αγουροξυπνημενοι, σκεφτομουν για αλλη μια φορα τις διαφορες στη ζωη των ανθρωπων σε αυτες τις χωρες με τις ζωες μας στην Ελλαδα -η τουλαχιστον με τη δικη μου τη ζωη. 
Αν ηταν μια λεξη που μου ερχοταν στο μυαλο ηταν Κουλτουρα. Λεμε οτι οι Ελληνες εχουμε πολιτισμο, κουλτουρα κλπ, αλλα που ηταν στην καθημερινοτητα τελικα; 
Εδω η κουλτουρα φαινοταν και στα πιο απλα πραγματα: οριστε ενας τυπικος Δανεζικος ξενωνας που προσεφερε μονο φρεσκα βιολογικα προιοντα για πρωινο, σε ενα χωρο που θα ζηλευε και το πιο κυριλε εστιατοριο της Αθηνας, με προσοχη ακομα και στη παρουσιαση και μερακι στη διακοσμηση. 
Οι ανθρωποι γυρω μπορει να ηταν διερχομενοι ταξιδιωτες αλλα δεν ακουγες φωνες, φασαρια και κακο χαμο: ηξεραν να σεβονται ο ενας το διπλανο του και να χαιρονται χωρις να ενοχλουν.
Οταν δε τελειωναν το πρωινο τους, μαζευαν τους δισκους τους, καθαριζαν το τραπεζι τους και πηγαιναν τα υπολοιπα χαρτια, πλαστικα κλπ στους αντιστοιχους καδους ανακυκλωσης που βρισκονταν στο χωρο. Πολιτισμος. 
Εμεις εδω στην Ελλαδα ομως σε αυτα κολλαμε μια ταμπελα ``ελα μωρε οι ξενερωτοι/δεν ξερουν να χαιρονται τη ζωη τους'' και ο μηνας εχει 9... Και ας αφηνουμε στη παραλια φραπεδες, τσιγαρα, αντηλιακα, χαρτια και κουτακια μπυρας... Και ας γινεται πανικος σε καθε μερος που παμε χωρις να μας ενδιαφερει τι κανει ο διπλανος μας. Ειμαστε οι καλυτεροι ετσι και αλλιως...

Δεν θα ψυχοπλακωνομουν ομως! Στο χερι του καθενα μας ειναι να κανει οτι μπορει ωστε να δωσει στις νεες γενιες που ερχονται τα εφοδια να γινουν εφαμιλλοι και καλυτεροι απο αυτους τους λαους εδω πανω. Γιατι τα υλικα υπαρχουν. Απλα πρεπει να τα φροντισουμε και να τα φτιαξουμε με σωστο τροπο. 

Τελειωσα το πρωινο και βγηκα εξω να φορτωσω τη μηχανη με ενα μεγαλο χαμογελο. 
Το Århus με αποχαιρετουσε πανεμορφο. Η γραφικη παλια πολη, το μεγαλο λιμανι, η πολυ ομορφη αρχιτεκτονικη των σπιτιων -νεων και παλιων- κατα μηκος της προκυμαιας...

\photo{140.jpg}
\photo{141.jpg}

Τι ομορφη μερα! Στο ταξιδι μου ειχα καποια σημεια ``κομβικα'' οπου ευχομουν να βρω καλες συνθηκες. Ε λοιπον, αυτο ηταν ενα τετοιο σημειο και δεν θα μπορουσα να ζητησω καλυτερο καιρο ουτε στα ονειρα μου...
Εθνικη οδος-ποιημα!

\photo{142.jpg}

Παρα τη μικρη καταιγιδα που βρηκα στη πορεια (εδω δεν ηταν νοτια Ευρωπη -ο καιρος σε αυτα τα μερη ηταν εντελως απροβλεπτος και παρανοικος!) τιποτα δεν μπορουσε να μετριασει τη χαρα μου! Η λιακαδα εμφανιστηκε ξανα και με παρεα μια ακομα μηχανη συντομα εφτασα στο λιμανι του Hirtshals με τη Βορεια Θαλασσα να ανοιγεται μπροστα μου... Η πυλη για εναν αλλο εντελως εξωπραγματικο κοσμο...

\photo{143.jpg}

Το Fjord Cat, ενα υπερσυγχρονο catamaran τεραστιων διαστασεων, ηταν εκει και με περιμενε για να με περασει στα μερη που τοσα χρονια σκεφτομασταν σαν ενα μακρυνο ονειρο... Δεν μπορουσα να πιστεψω ακομα οτι τωρα ημουν εκει πλεον!

\photo{144.jpg}

Μπηκα στο αμπαρι και εδεσα τη μηχανη γερα με τους θηριωδεις ιμαντες του πλοιου, ενω γυρω μου περνουσε πολυς κοσμος που πηγαινε προς το σαλονι. Ανεβηκα κι εγω και το θεαμα δεν θυμιζε καθολου τα πλοια που ξερουμε: ενας πολυ μεγαλος hi-tech χωρος σε δυο οροφους (!) με δερματινες πολυθρονες και τραπεζια παντου και στο κεντρο στο ισογειο μαγαζια που σερβιραν καφε και φαγητο.

\photo{145.jpg}

Επιασα μια θεση και αραδιασα τα πραγματα να στεγνωσουν απο τη μπορα που ειχα βρει πριν λιγο στη διαδρομη, ενω το catamaran ξεκινησε για το Kristiansand της Νορβηγιας! 

Ανοιξα τους χαρτες και αρχισα να σχεδιαζω τη διαδρομη της ημερας μου. Ο αρχικος σχεδιασμος μου ηταν να ξεκινησω απο το λιμανακι του Kristiansand και να ανεβω περιπου 300 χιλιομετρα βορειοδυτικα με προορισμο το βουνο Forsand, οπου θα εμενα σε ενα ορεινο καταφυγιο για να επισκεφτω την επομενη μερα ενα απιστευτο γεωλογικο θαυμα της φυσης, το διασημο Preikestolen. 

Στη Γερμανια ο Κωστης οταν εμαθε για τα σχεδια μου ειχε να μου προτεινει διαφορες μικρες αλλαγες, και εδω ηταν μια απο αυτες: ``Νικο, αν μπορεις απεφυγε την εθνικη. Ο Ε39 δεν εχει τιποτα σπουδαιο να δεις. Παρε καλυτερα τον μικροτερο αλλα απειρως πιο γραφικο δρομο ``9'' προς το Evje και θα σε βγαλει στο πανεμορφο λιμανακι Lysebotn που πραγματικα δεν πρεπει να το χασεις. Απο εκει μπορεις να παρεις το φερυ και να σε περασει κοντα στο Forsand και να ανεβεις στο βουνο.''

Εχοντας διαβασει ενα προσφατο ταξιδιωτικο του Κωστα στη νοτια Νορβηγια και εχοντας δει φωτογραφιες απο τα μερη που μου προτεινε ημουν σιγουρος οτι η δικη του προταση ηταν απειρως καλυτερη απο τη δικη μου!

\photo{146.jpg}

Συντομα ομως τα σχεδια και οι σκεψεις για τις διαδρομες κοπηκαν αποτομα. Το πλοιο βγηκε στα ανοιχτα και εκει ξεκινησε και το μαρτυριο μου. Γενικα με τα πλοια δεν τα παω καλα και ζαλιζομαι ποσο μαλλον τωρα που μια απιστευτη θαλασσοταραχη τιναζε το θηριωδες σκαρι σαν καρυδοτσουφλο απο τη μια μερια στην αλλη. 
Και δεν μιλαμε για ενα τυχαιο σκαφος. Το Fjord Cat μπορει να μεταφερει 900 επιβατες και 240 αυτοκινητα, ενω ειναι ενα απο τα πιο γρηγορα επιβαταγωγα στο κοσμο και κατεχει το παγκοσμιο ρεκορ της πιο γρηγορης διασχισης του Ατλαντικου ωκεανου!
Σε διαφορα σημεια στο σαλονι ειχε οθονες οπου εδειχνε το στιγμα του πλοιου μεσω GPS καθως και την ταχυτητα του: ακομα και στη θαλασσοταραχη τωρα κρατουσε σταθερη ταχυτητα 76 χλμ/ωρα, με μεγιστη ταχυτητα τα 89 χιλιομετρα/ωρα!

Ισως λογω ομως και της ταχυτητας του, το πλοιο εσκαγε με βια στα κυματα και εγω υπεφερα. Εκατσα στη πολυθρονα και εβαλα το κεφαλι πανω στο τραπεζι προσπαθωντας να ηρεμησω το στομαχι που ειχε σκαρφαλωσει στη πλατη μου! Η Αυρα τι να εκανε αραγε τωρα; Θα την εβρισκα στη θεση της η ξαπλωμενη φαρδια πλατια πανω σε κανενα αυτοκινητο; 

Μετα απο δυο μαρτυρικες ωρες το μαρτυριο εδειχνε να ειχε τελειωσει... Η θαλασσα ειχε καλμαρει και το στιγμα εδειχνε οτι μπαινουμε στο λιμανι του Kristiansand. 
Βγηκα στο μικροσκοπικο καταστρωμα οπου με περιμενε το πιο ομορφο θεαμα! Η Νορβηγια με καλοσωριζε με τον πιο ομορφο τροπο! 

\photo{147.jpg}

Κατεβηκα στο αμπαρι με το αγχος του πως εβρισκα τη μηχανη. Τελικα ηταν στη θεση της και με περιμενε να εξερευνησουμε αυτη τη αγνωστη γη. 
Σε λιγο οι ροδες της Αυρας πατουσαν Ν ο ρ β η γ ι α ! 
Ο καιρος ηταν καταπληκτικος! 25 βαθμοι και λιακαδα! Τι αλλο θα μπορουσα να ζητησω; 

\photo{148.jpg}

Εξω απο το λιμανι εκανα μια σταση για ανεφοδιασμο που ηξερα οτι θα ...πονεσει. Εδω η απλη βενζινη ειχε 2 ευρω το λιτρο! Εβγαλα απο το μυαλο της σκεψη οτι η βενζινη στη πιο πλουσια χωρα της Ευρωπης ειχε μολις 10-15 λεπτα διαφορα με τη τιμη στη Μπανανια και χαιρετησα τα 2 παιδια με τις Ducati που ηρθαν στο πρατηριο για να γεμισουν. Δανοι που γυριζαν απο μια ολιγοημερη βολτα στα περιξ, μου προτειναν και αυτοι την Οδο 9 και χαμογελασα: ``Ο γκουρου ειχε δικιο!''

Πηρα το δρομο προς την περιφημη επαρχιακη οδο και μπορει να ηταν η ιδεα μου αλλα ολα εδω πανω μου εμοιαζαν εντελως διαφορετικα. Τα σπιτια, οι δρομοι, η φυση, η καθαριοτητα... 

\photo{149.jpg}

H διαδρομη ανοιγοταν μπροστα μεσα σε μια Φυση απεριγραπτη! Σαν απο αλλο πλανητη! Δρομοι που χανονταν στο βαθος του οριζοντα, κινηση μηδενικη, ασφαλτος αψεγαδιαστη σαν ψευτικη, καταφυτα δαση, λοφοι, νερα...
Και ακομα δεν ειχα δει τιποτα.... 

\photo{150.jpg}

Οκ, εδω χρειαζεται λιγο προλογο....
Σε αυτη τη χωρα υπαρχει νερο. ΠΟΛΥ νερο. Λιμνες, ποταμια, καταρρακτες, θαλασσες... 
Ενας καλος φιλος καποτε ειπε οτι ``η Νορβηγια σου δινει την αισθηση οτι ειναι μικρες νησιδες γης που επιπλεουν πανω στο νερο'' και ειχε απολυτο δικιο.

Και εκεινη η πρωτη φορα που θα αντικρυσεις μια απο τις αμετρητες λιμνες στη Νορβηγια ειναι πολυ ιδιαιτερη... 
Καθως περνουσα αναμεσα στο δασος, τα δεντρα ανοιξαν για να φανερωσουν ενα α π ε ρ ι γ ρ α π τ ο θεαμα.
.
.
.
.
.
.
.
.
.
.
.
.
.
.
.
.
.
.
.
.
.
.
.
.

\photo{152.jpg}

Αφησα την Αυρα στην ακρη του δρομου.... 

\photo{153.jpg}

...και καθισα να κοιταζω αφωνος το μαγευτικο σκηνικο! Αρχισα να γελαω αμηχανα και η μονη σκεψη μου ηταν ``ΤΙ βλεπω θεε μου....''

\photo{154.jpg}

Δεν ξερω ποση ωρα εκατσα εκει και τραβουσα φωτογραφιες... Δεν χορταινα τις εικονες που εβλεπα μπροστα μου!
Συνεχισα και οσα υπερθετικα και να χρησιμοποιουσα δεν εφταναν για να περιγραψουν την ομορφια αυτο του τοπου...
Περικυκλωμενος απο ατελειωτα δαση....

\photo{155.jpg}

Λιμνες μεσα σε τοπια σαν απο καρτ-ποσταλ ...

\photo{156.jpg}

Εικονες τοσο ομορφες που ελεγα οτι δεν μπορει να ειναι αληθινες...!

\photo{157.jpg}

Αυτο ομως ηταν η Νορβηγια. Εκει που νομιζες οτι τα ειχες δει ολα στην επομενη στροφη ερχοταν μια νεα εικονα να σε αφησει αναυδο!

\photo{158.jpg}
\photo{159.jpg}

Μεσα σε αυτο το απεριγραπτο σκηνικο σταματησα στην ακρη του δρομου και εσβησα τη μηχανη. 

\photo{160.jpg}

Εκανα να βγαλω το κρανος και ενοιωσα τα ματια μου υγρα... Το μυαλο μου παλευε να χωρεσει τα οσα μετεφεραν οι αισθησεις... Τα τοπια, οι μυρωδιες, οι ηχοι... Ημουν σε ενα κοσμο αληθινα παραμυθενιο και αυτο δεν το ειχα ζησει ποτε πριν. Θελω να πιστευω οτι εχω ταξιδεψει λιγο στη ζωη μου, ομως ΠΟΤΕ και πουθενα δεν ειχα δει κατι σαν ολα αυτα που εβλεπα τωρα εδω μπροστα μου....
Πως γινοταν να υπαρχουν τετοια μερη στο πλανητη μας και να μην εχουμε ιδεα καν;
Κοιταξα την καλη μου κατακοκκινη Αυρα και χαμογελασα περηφανα. Κοιτα μεχρι που με εχεις φερει καλη μου...

Καπου εκει η ησυχια του τοπου εσπασε απο τον χαρακτηριστικο μπασο ηχο δικυλινδρων μηχανων. Γυρισα και ειδα δυο μηχανες που ηρθαν και αραξαν διπλα μου: μια λευκη BMW GS 800 και ενα πορτοκαλι Kawasaki ER6. Ηταν ενα ζευγαρι σαρανταρηδων Σουηδων, ο Christian και η Asa, που ειχαν ξεκινησει τη προηγουμενη μερα απο το Örebro της Σουηδιας με τις μηχανες τους και θα κανανε ενα ταξιδακι μερικων ημερων στη Νοτια Νορβηγια. Πολυ ευχαριστα παιδια! Αφου ανταλλαξαμε τις κλασσικες αβροφροσυνες για τις μηχανες μας, με ρωτησαν με μεγαλο ενδιαφερον απο που ερχομουν και που πηγαινα και συντομα πιασαμε τη κουβεντα περι ταξιδιων και των χωρων που ειχαμε δει... 
Ο Christian ηταν σχετικα νεος μηχανοβιος μιας που ειχε τη μηχανη του μολις 3 χρονια αλλα ειχε καταφερει να γραψει πολλα χιλιομετρα σε ταξιδια εντος και εκτος δρομου σε ολες τις Σκανδιναβικες χωρες. Δικοι μου ανθρωποι! 
Ποσο χαιρομουν τετοιες μοτοσυναντησεις απο το πουθενα! Αυτοι οι ανθρωποι ειχαν σταματησει και ειχαμε πιασει κουβεντα απλα και μονο γιατι καβαλουσαμε μηχανη. Δεν χρειαζονταν συστασεις και περιττα λογια. Ειχαμε τη κοινη αγαπη της Μηχανης και του Ταξιδιου να μας ενωνει.

Συντομα τα παιδια καβαλησαν τις μηχανες τους και συνεχισαν. Οπως μου ειχαν πει πηγαιναν στην ιδια διαδρομη με εμενα, οποτε ισως τους εβρισκα παρακατω στο δρομο! 

Η διαδρομη τωρα επιασε να περναει μεσα απο μικρα χωρια οπου ειχα την ευκαιρια να κανω μερικες πρωτες διαπιστωσεις για τους Νορβηγους. 
Καταρχας οι Νορβηγοι αγαπουν τη φυση.
Η αρχιτεκτονικη των σπιτιων και μαγαζιων τους ηταν σε απολυτη αρμονια με το περιβαλλον γυρω τους. 
Πετρα και ξυλο παντου με ομορφες λεπτομερειες και στους πιο μικρους οικισμους.

\photo{161.jpg}

Επισης, οπως ολοι οι Σκανδιναβοι, λατρευουν το camping. Η Νορβηγια εχει κυριολεκτικα εκατονταδες χωρους κατασκηνωσης. Αριστερα και δεξια στο δρομο εβλεπα συνεχως αναμεσα στις συσταδες των δεντρων campings, αλλα πιο οργανωμενα και αλλα πιο ...ελευθερα.

\photo{162.jpg}

Αυτο ομως δεν ηταν τυχαιο. 
Βλεπετε στη Νορβηγια (οπως και στη Σουηδια και τη Φινλανδια) η διαβιωση στη φυση ηταν τροπος ζωης απο αρχαιων χρονων. 
Οι παντες εδω εχουν το δικαιωμα να μπαινουν, να διασχιζουν, ακομα και να διαμενουν σε ακαλλιεργητη γη οπουδηποτε στη χωρα -απο την ακρη του δρομου, μεχρι οποιαδηποτε ακτη, και οποιοδηποτε βουνο μεχρι ακομα και σε προστατευμενες περιοχες οπως Εθνικους Δρυμους!

Αυτο το δικαιωμα γνωστο στη χωρα ως \textbf{allemannsrett} (All man's right / Το δικαιωμα καθε ανθρωπου) υπηρχε ως καθιερωμενο εθιμο απο αρχαιων χρονοων και το 1957 περασε και ως νομοθεσια στο συνταγμα της Νορβηγιας. 
Βασιζεται στη φροντιδα για τη φυση και ολοι οι επισκεπτες αναμενονται να δειχνουν τον απαραιτητο σεβασμο προς τους αγροτες, ιδιοκτητες γης, αλλους επισκεπτες καθως και για το περιβαλλον. 
Η φιλοσοφια του σεβασμου του περιβαλλοντος φαινεται και στις πιο μικρες λεπτομερειες. Στη καλλιεργημενη γη δεν ειναι επιτρεπτο για ευνοητους λογους να διασχιζεται, ομως επιτρεπεται να τη διασχιζει κανεις οταν ειναι παγωμενη και σκεπασμενη με χιονι! 

Στην ουσια ο περιηγητης σε αυτες τις χωρες μπορει -ελευθερα και δια νομου- να διασχισει οποια περιοχη θελει χωρις κανενα περιορισμο και να διανυκτερευσει απολυτως οπουδηποτε μεχρι δυο 24ωρα, ενω ειναι ελευθερος να φαει τους καρπους των δεντρων και λοιπων εδωδιμων και φρουτων που μπορει να βρει, αρκει να σεβαστει το περιβαλλον και τυχον αλλους περιηγητες και να αφησει το μερος που εμεινε καλυτερο και απο οτι το βρηκε.

Χαρακτηριστικο του ποσο σοβαρα παιρνουν την ελευθερια του καθε ανθρωπου να περιηγειται ελευθερα στη φυση ειναι το οτι απαγορευεται αυστηρα η κατασκευη φρακτων και αλλων εμποδιων σε κοινοχρηστη γη, και αν αυτο συμβει πεφτουν πολυ βαρια προστιμα.

Τι να λεμε... Τι συγκρισεις να εκανα τωρα... Απιστευτοι λαοι, ετη φωτος μπροστα!

Ακομα και στις πιο μικρες λεπτομερειες οι ανθρωποι εδειχναν το σεβασμο τους στη φυση και το περιβαλλον.
Ναι, αυτο ηταν σταση λεωφορειου και οχι αυτη η ζουγκλα στην σκεπη ΔΕΝ ειχε γινει τυχαια, ουτε ειχε λογο υπαρξης -ηταν εκει απλα για διακοσμηση! Η πρακτικη του να βαλεις ...χωμα(!) στη οροφη του σπιτιου και να σπειρεις λουλουδια και πρασιναδα θεωρειται συνηθισμενη και ομορφη για τους Σκανδιναβους... 

\photo{162.jpg}

Αφησα πισω μου τα μικρα χωριουδακια και ο δρομος τωρα μπροστα ανοιχτηκε, ατελειωτος, αδειος μεχρι εκει που εβλεπε το ματι...

\photo{163.jpg}

Εβαλα στα ακουστικα το καλυτερο soundtrack \href{http://goo.gl/xUsd7}{Sivert Hoyem - Give it a whirl} για αυτο εδω τον απιστευτο τοπο και η μπαλα χαθηκε... 
Δυναμωστε και απολαυστε!

\photo{164.jpg}

Ο δρομος περνουσε συνεχως τωρα διπλα σε ατελειωτες λιμνες και δεν ηξερα που να πρωτοκοιταξω! 

\photo{165.jpg}

Απο τη μια παντου νερα και απο την αλλη θεορατοι βραχοι και βουνα που υψωνονταν σαν γιγαντες διπλα μου...

\photo{166.jpg}

Τα δεκαδες τουνελ στη διαδρομη τρυπουσαν τους βραχους και εδιναν την αισθηση οτι με μεταφερουν σε ενα παραλληλο συμπαν...
Ωσπου μετα απο ενα ακομα τουνελ....

\photo{167.jpg}

...εμφανιστηκε μπροστα μια ακομα εικονα απο αυτες που παιρνεις μαζι σου μεχρι να κλεισεις τα ματια σου απο αυτο τον κοσμο...!
.
.
.
.
.
.
.
.
.
.
.
.
.
.
.
.
.
.
.
.
.
.
.
.

\photo{168.jpg}

Οh yeah baby, now we're talkin'....
Now ΜΥ waiting was done...

\photo{169.jpg}

Συντομα αφησα πισω τον οντως καταπληκτικο Δρομο 9 και μπηκα σε ενα μικρο δρομακι μεσα στο δασος που μου εδειχνε το GPS. 
Ο δρομος αρχισε να ανηφοριζει προς τα πανω και ειχε στενεψει απιστευτα! Η Αυρα επιανε ολο το πλατος του δρομου και δεν ηθελα καν να σκεφτω τι θα γινοταν αν 2 αυτοκινητα συναντιοντουσαν εδω περα -ναι, αυτο που βλεπετε ηταν δρομος διπλης κατευθυνσης!

\photo{170.jpg}

Συνεχισα τη πορεια προς το βουνο και στην οθονη το GPS ελεγε ``Αγνωστη Οδος''. Γαμω! Τη πρωτη μερα μου στην Νορβηγια και ηδη διεσχιζα αγνωστους δρομους που δεν υπαρχουν καν στο χαρτη! Αυτα ειναι! 

\photo{171.jpg}

Ο δρομος ολο και ανηφοριζε και η θεα απο εκει πανω ηταν μοναδικη...

\photo{172.jpg}

Εδω ομως αρχιζαν τα πραγματικα απιστευτα!
Συντομα η διαδρομη με εβγαλε σε ενα υψιπεδο, που ομως δεν εμοιαζε με ΟΤΙΔΗΠΟΤΕ αλλο ειχα δει στο παρελθον!
Αγρια πετρα, πληρης ελλειψη οποιασδηποτε βλαστησης, κανενα δεντρο, ομως παντου νερο και χιονι! 

\photo{173.jpg}

Το τοπιο ηταν εντελως εξωπραγματικο! Εμοιαζε σαν κατι τοπια που ειχα δει σε φωτογραφιες στα 4000+ μετρα σε βουνα οπως το Εβερεστ, ομως εδω το GPS διαφωνουσε: βρισκομουν μολις στα 1000 μετρα υψομετρο! 
Τοτε πως; Που ημουν; Τι ηταν ολο αυτο εδω το πραγμα;

\photo{174.jpg}

Συνεχισα να προχωραω μεσα σε αυτο το εξωγηινο τοπιο μεχρι που ειδα ενα γνωριμο θεαμα!

\photo{175.jpg}

Ο Christian και η Asa ηταν και αυτοι εδω πανω στην ερημια, φωτογραφιζοντας τα ...αξιοθεατα!
Τι φοβερη συμπτωση! Χαιρομουν παρα πολυ που εβλεπα δυο γνωριμα προσωπα σε αυτο το τοσο αλλοκοτο περιβαλλον. 

\photo{176.jpg}

Αρχισα να συζηταω με τον Christian για το μερος. Τι ηταν ολο αυτο; 
Ο Christian γυρισε και μου ειπε: ``Στη Νορβηγια οπως υπαρχουν τα Fjord υπαρχουν και τα Fjell.''
Κυριολεκτικα Fjell σημαινει ``βουνο'' στα Νορβηγικα και στην ουσια ηταν τεραστιες επιπεδες εκτασεις στις κορυφες των οροσειρων της χωρας που ειχαν ισοπεδωθει πριν εκατομμυρια χρονια απο τους ιδιους τεραστιους παγετωνες που ειχαν δημιουργησει και τα fjord. Μαλιστα τα βουνα αυτα αρχικα ειχαν ως και ΠΕΝΤΕ φορες μεγαλυτερο υψομετρο απο οτι τωρα φτανοντας σε υψος και τα 10.000 μετρα.
Κατα το μεγαλυτερο μερος του χρονου τα fjell ηταν σκεπασμενα απο χιονι, ενω οταν το χιονι υποχωρουσε αφηνε παντου λιμνες και ελαχιστη βλαστηση.

Ωραια! Πριν καν να εβλεπα το πρωτο μου fjord θα εβλεπα τα fjell! Μα ποιος ημουν επιτελους;; 
Και παλι ομως αυτο δεν εξηγουσε το γιατι το τοπιο εδειχνε τοσο εξωπραγματικο. Και στην Ελλαδα ειχαμε βουνα που σκεπαζονται για μεγαλα διαστηματα με χιονι αλλα δεν ηταν ετσι ακομα και σε υψομετρα πανω απο 2000 μετρα. 

Καπου εκει θυμηθηκα κατι που ειχα διαβασει στον οδηγο του Lonely Planet. Εδω πανω δεν επαιζε ρολο το υψομετρο των βουνων. Μπορει το GPS να εδειχνε σχετικα χαμηλο υψομετρο αλλα εδω πανω ημασταν ψηλα πλεον στο πλανητη και τα βουνα ειχαν χαρακτηριστικα που σε μικροτερα πλατη εβρισκες σε πολυ μεγαλυτερα υψομετρα. Γεωγραφικο πλατος και οχι υψος...

Αν μου ελεγε καποιος οτι βρισκομουν σε αλλο πλανητη αυτη τη στιγμη θα τον πιστευα...

\photo{177.jpg}

Ρωτησα τα παιδια τι εκαναν εδω πανω. Μου ειπαν οτι πηγαιναν στο Lysebotn οπου και θα εμεναν! Τι συμπτωση! Απο ολα τα μερη που θα μπορουσαν να διαλεξουν, ειχαν επιλεξει να κανουν ακριβως την ιδια διαδρομη με μενα!\\

\dialogue{Τι λες; Παμε παρεα μεχρι εκει;}
\dialogue{Φυγαμε!}\\

Ετσι ξεκινησαμε πορεια ολοι μαζι μεσα απο αυτο το εκπληκτικο τοπιο και σκεφτομουν ολα τα απιθανα πραγματα που ειχαν γινει μεχρι στιγμης! Μερη που δεν μπορουσα να φανταστω οτι υπαρχουν στο κοσμο και τωρα ημουν παρεα με δυο Σουηδους που ειχα γνωρισει πριν μερικες ωρες ετσι απλα στο δρομο!
Η ασφαλτος ξετυλιγοταν μπροστα μου ατελειωτη, σε μια διαδρομη που ομοια της δεν ειχα ξανακανει ποτε στη ζωη μου...

\photo{178.jpg}

Τι τοπιο....! Εκει που νομιζα οτι δεν θα μπορουσα να δω κατι ακομα καλυτερο απο οσα ειχα δει μεχρι τωρα, ερχοταν ΑΥΤΗ η εικονα και με εκανε να κοιταζω χωρις να ξερω τι να πω... Τα συννεφα απλωνοντας απο πανω μεχρι εκει που εβλεπε το ματι και εφτιαχναν μια αληθινα Βιβλικη σκηνη... 

\photo{179.jpg}
\photo{180.jpg}

Δεν ηξερα τι να πω για ολο αυτο που εβλεπα... Πολλες φορες εχω θαυμασει την ομορφια της φυσης σε μια διαδρομη αλλα ηταν ελαχιστες οι φορες που πραγματικα δεν μπορουσα να πιστεψω στα ιδια μου τα ματια! 

Ο δρομος ανεβαινε και κατεβαινε συνεχως ακολουθωντας το αναγλυφο του παραξενου τοπιου και μου χαριζε εικονες μοναδικης ομορφιας μεχρι το ακρο του οριζοντα! 
Ηθελα να καθομαι να κοιταω το τοπιο με τις ωρες...

\photo{181.jpg}
\photo{182.jpg}

Δεν ειχα ξαναδει ποτε κατι παρομοιο σαν τις εικονες αυτες! 

\photo{183.jpg}
\photo{184.jpg}

Ενοιωθα απειροελαχιστος μπροστα στο Μεγαλειο μιας συγκλονιστικης Φυσης...
Τι να ελεγα αλλο για τετοιες εικονες...;

\photo{185.jpg}

Ο δρομος οσο πηγαινε στενευε και αλλο και ανεβοκατεβαινε το βουνο στριφογυριζοντας, με μια νεα εικονα πισω απο καθε στροφη...

\photo{186.jpg}

...ομως δεν τιποτα δεν μπορουσε να με προιδεασει για αυτο που θα εβλεπα τωρα μπροστα μου....!
.
.
.
.
.
.
.
.
.
.
.
.
.
.
.

\photo{187.jpg}

Καποια στιγμη φτασαμε στο ψηλοτερο σημειο του βουνου και αντικρυσα το πιο αλλοκοτο θεαμα: παντου γυρω μου απειρες μικρες και μεγαλες στιβες απο πετρες! Και οταν λεμε παντου εννοουμε παντου!

\photo{188.jpg}

Εσβησα τη μηχανη και κατεβηκα. Περιπλανηθηκα στο χωρο χαζευοντας το περιεργο αυτο θεαμα... Τι ηταν ολα αυτα; Τοτε εφερα στο μυαλο μου ενα παλιο ταξιδιωτικο στην Ανατολια που ανεφερε οτι οι ταξιδιωτες που περνουσαν απο μερη καποιας σημασιας ηθελαν να αφησουν το σημαδι τους για καλη τυχη και ετσι εστηναν μικρους η μεγαλους σωρους απο πετρες για να δειξουν οτι περασαν απο εδω.

Διαβασα στον οδηγο: "Αυτες οι στιβες με πετρες που βρισκονται διασπαρτες σε ολη τη χωρα ειναι γνωστες ως Cairns. Η λεξη προερχεται απο τη Σκοτσεζικη διαλεκτο càrn και τετοιες στιβες βρισκονται παντου στο κοσμο, οπου υπαρχει καποιος μεγαλος λοφος, κορυφη βουνου, καταρρακτες η ακρωτηρια, αλλα και σε ανυδρα μερη οπως ερημοι και τουνδρες. Διαφερουν σε μεγεθος, απο μικρες στιβες μεχρι μεγαλοι τεχνητοι λοφοι, αλλα και σε περιπλοκοτητα, απο χαλαρες πετρες στιβαγμενες η μια πανω στην αλλη ως και επιτευγματα ογκολιθικης τεχνικης. Τα Cairns μπορει να ειναι βαμμενα η διακοσμημενα για μεγαλυτερη ορατοτητα η ακομα και για θρησκευτικους λογους.

Αυτη η πρακτικη ακολουθειται απο την προιστορια, οπου τα cairns χρησιμοποιουνταν για κυνηγι, αμυντικους, θρησκευτικους η ακομα και αστρολογικους σκοπους. Σημερα χρησιμοποιουνται απλα ως μνημεια σηματοδοτοντας τη παρουσια καποιου στη περιοχη."
Δεν θα μπορουσα λοιπον να παραλειψω να φτιαξω και εγω τη δικη μου μικρη στιβα απο πετρες αναμεσα στις αλλες. Τη σημασια της την ηξερα εγω και αρκουσε...

\photo{189.jpg}

Βεβαια ακομα πιο κλασσικη ηταν η πρακτικη των ταξιδιωτων να βαζουν ενα αυτοκολλητο η ενα σημειωμα στις πινακιδες του δρομου και εδω μαλλον το ειχαν παει ενα βημα παραπερα...

\photo{190.jpg}

Η καλη μου Αυρα παντα διπλα μου συντροφος πιστος, ετοιμη να με ταξιδεψει ακομα παραπερα...

\photo{191.jpg}

...μεσα σε τοπια απιστευτης αγριας ομορφιας...

\photo{192.jpg}

...και εκει που νομιζα οτι δεν υπηρχε κατι αλλο πλεον να με εντυπωσιασει ο δρομος επιασε να κατηφοριζει προς το Lysebotn και για αλλη μια φορα εμεινα να κοιταω σαν χαζος...
Το αναγλυφο του τοπιου και ο δρομος που χανοταν και στριφογυριζε σαν φιδι μεσα σε ολο αυτο ηταν ενα θεαμα απλα μοναδικο... 
Τοσα ταξιδια και τετοια ομορφια δεν ειχα ποτε μου και πουθενα αλλου...!

\photo{193.jpg}
\photo{194.jpg}

Εδω και ωρα ταξιδευα μεσα στο απιστευτο αυτο τοπιο παρεα με το ζευγαρι των Σουηδων, ομως συντομα στη παρεα προστεθηκαν και καποια αλλα παιδια απο τη Δανια... Δεν ηθελα να τελειωσει ποτε αυτη η διαδρομη!

\photo{195.jpg}

...Καπως ετσι αρχισαμε να κατηφοριζουμε ολοι μαζι το πασο προς το λιμανακι του Lysebotn, στη μυτη του διασημου Lysefjord (ελληνιστι το Φιορδ Lyse).

\photo{196.jpg}

Οι εκπληξεις ομως δεν ελεγαν να τελειωσουν! Η Νορβηγια εκτος ολων των αλλων ηταν διατρητη απο αμετρητα τουνελ ολων των μεγεθων, διατομων και σχεδιων. Αλλα λιγων εκατονταδων μετρων και αλλα πολλων χιλιομετρων, αλλα μεσα απο βραχο σε μεγαλο υψομετρο και αλλα υποθαλασσια... Λιγα ομως τουνελ μπορουσαν να συγκριθουν με το τουνελ που βρισκοταν τωρα εδω... 
Μπηκα στο στενο τουνελ. 
Σταγονες νερου επεφταν απο το γυμνο βραχο και ο χαμηλος κιτρινος φωτισμος εδινε στο μερος μια σχεδον τρομακτικη οψη.
Ξαφνικα ο δρομος αρχισε να κατεβαινε αποτομα προς τα κατω και ο ηχος του V4 της Αυρας αντηχουσε πανω στα τοιχωματα του τουνελ σαν ουρλιαχτο. Και τοτε το ειδα: μια στροφη 180 μοιρων ΜΕΣΑ στο τουνελ! Εκοψα οσο μπορουσα και πηρα τη στροφη προσεκτικα... Απιστευτο και ομως αληθινο...

Και εκει που τα λογια ειναι πολυ φτωχα για να περιγραψουν μια κατασταση, ισως η εικονα τα καταφερει καλυτερα...
Δειτε, ακουστε, απολαυστε και ...μην τρομαξετε! \href{http://www.youtube.com/watch?v=Gt4kouWnKh0}{Lysebotn Tunnel}

Και τελικα... φως! Πηρα μια βαθεια ανασσα και το μαγευτικο τοπιο με καλοσωριζε στη γη των Φιορδ....

\photo{197.jpg}

Εδω το λογο ειχαν οι τιτανιοι ογκοι βραχου που κρεμονταν πανω απο τα κεφαλια μας αριστερα και δεξια του δρομου.... 
Νερα ετρεχαν απο παντου γυρω και επεφταν με δυναμη απο υψος εκατονταδων μετρων!

\photo{198.jpg}

Τα μεγεθη σε αυτη τη χωρα ειχαν πλεον χασει καθε νοημα!
Μπορειτε να βρειτε τη μηχανη σε αυτη την εικονα;

\photo{199.jpg}

Ο Κωστης στο δικο του ταξιδιωτικο ειχε περιγραψει με τα πιο ομορφα λογια το Lysebotn. Εφτασα στο μικροσκοπικο λιμανακι και αφησα τη κατακοκκινη κουκλα μου να ξεκουραστει και να γινει και αυτη μερος αυτου του μαγικου τοπου...

\photo{200.jpg}

Και εκει ηταν που Το Αντικρυσα. 
Το Ονειρο. 
Το πρωτο μου φιορδ. 
Αυτο που τοσα χρονια ειχα σαν αμυδρη εικονα στο μυαλο μου ανοιγοταν τωρα μπροστα μου και ειχα παραλυσει.
Εμεινα αναυδος και το μυαλο μηδενισε. 
.......
... Ημουν εδω...!
.
.
.
.
.
.
.
.
.
.
.
.
.
.
.
.
.
.
.
.
.


\photo{201.jpg}

Ο Christian και η Asa στο μεταξυ ειχαν παει παραδιπλα στο τοπικο camping, οπου ειχαν αποφασισει να διανυκτερευσουν σημερα. 
Εγω ειχα αρχικο σκοπο να παρω το καραβακι για την αλλη ακρη του φιορδ και το Forsand, αλλα τωρα αληθινα δεν ηθελα καν να σκεφτω να φυγω απο εδω! 
Και να ηθελα ομως δεν θα μπορουσα τελικα: βλεπετε το τελευταιο φερυ της μερας ειχε μολις φυγει και το επομενο ηταν αυριο το πρωι.
Το αρχικο προγραμμα μου πηγαινε περιπατο αλλα ποιος θα παραπονιοταν σε ενα τετοιο μερος και με τετοιο καιρο; Εξαλλου εδω θα ειχα και τους νεους μου φιλους παρεα! 

Παμε να στησουμε και να βολευτουμε λοιπον!
Το camping ηταν ακριβως διπλα στη θαλασσα και πολυ φροντισμενο αν και η βλαστηση περιοριζοταν σε μερικα λιανα δεντρακια στην εισοδο του... Σκια στο χωρο των σκηνων; you've got to be kidding!

\photo{202.jpg}
\photo{203.jpg}

Εστησα τη σκηνη σε ενα επικο τοπιο... Στο βαθος πισω ενας τεραστιος καταρρακτης επεφτε απο τα καθετα βραχια και στα αριστερα μου η θαλασσα του φιορδ με τον ηλιο να πεφτει σιγα σιγα...

\photo{204.jpg}

Δεν υπηρχε περιπτωση να αφησω την ευκαιρια να εξερευνησω αυτο το μερος! Τα παιδια ειχαν πιασει ενα ξυλινο σπιτακι και κανονισαμε να βρεθουμε σε καμια ωριτσα για φαγητο, οποτε ειχα χρονο μπροστα μου να απολαυσω το απιστευτο σκηνικο. 

Κατευθυνθηκα προς τον τεραστιο καταρρακτη και ενοιωθα μια αγρια χαρα σαν το μικρο παιδι με το καλυτερο παιχνιδι του κοσμου!
Δεν ειχα ξαναδει τετοιο θεαμα απο τοσο κοντα... Ο θεορατος βραχος πρεπει να ειχε υψος τουλαχιστον 200 μετρα και το νερο ετρεχε απο τη κορυφη φτιαχνοντας ενα σωρο μικροτερους καταρρακτες που εσκαγαν κατω με την εκκωφαντικη βουη του νερου που χτυπαει στα βραχια...

\photo{205.jpg}

Το μερος ηταν απεριγραπτο...! 
Περπατουσα στη σκια του καταρρακτη στη μικρη παραλια και τριγυρω μια παρεουλα τουριστων απολαμβαναν και αυτοι το μερος με μπαρμπεκιου, μπυριτσες και μουσικη απο το αμαξι τους...

Μπροστα μου ο ηλιος ειχε παρει να δυει και εβαφε τα καθετα τεραστια βραχια με χρυσα χρωματα, ενω το φως που επεφτε απο το πλαι στο βαθος εμοιαζε σαν ασπρογαλανες κουρτινες ριγμενες στον οριζοντα... Μια εικονα μοναδικη...

\photo{206.jpg}

Ποσα απεριγραπτα πραγματα ειχα δει μεσα σε μια μονο μερα; 
Ποσες φορες ειδα εικονες που ελεγα οτι δεν υπαρχει περιπτωση να δω κατι καλυτερο; 
Και ομως... Καπου εκει η Φυση βαλθηκε να με διαψευσει για αλλη μια φορα χαριζοντας μου μια απο τις καλυτερες εικονες της ζωης μου...


Lysefjord Norway, 9.50 μμ

\photo{207.jpg}

Με αυτη την εικονα στα ματια γυρισα στο camping και η μερα εκλεισε με τον καλυτερο τροπο: με τη παρεα φιλων!
Μαζι με τους γλυκυτατους Σουηδους ζευγαρι πηγαμε σε ενα γραφικοτατο τοπικο εστιατοριο και απολαυσαμε τις τοπικες λιχουδιες, με μπυριτσα και κουβεντουλα για τα παντα. Μηχανες, φιλια, σχεσεις, ταξιδια.
Ποσο μου αρεσει αυτη η Σκανδιναβικη αρχιτεκτονικη!

\photo{208.jpg}

Εβλεπα τους νεους φιλους μου να μιλανε, να γελανε και να αλληλοπειραζονται και χαμογελασα... Σκεφτομουν οτι τελικα ειδικα εμεις οι μηχανοβιοι ειμαστε παντου απο την ιδια ``παστα''! Θα μπορουσα καλλιστα να ειμαι αναμεσα σε παλια φιλαρακια σε καποια εκδρομη στα περιξ των Αθηνων και να γελαμε κουβεντιαζοντας ακριβως με τον ιδιο τροπο...
Στερεοτυπα χρονων για κρυους και απομακρους Σκανδιναβους που δεν ανοιγονται ευκολα σε αγνωστους ανθρωπους τωρα χανονταν σαν καπνος. Αυτο τελικα ειναι και το μεγαλυτερο δωρο ενος ταξιδιου: σου ανοιγει τα ματια στην πραγματικη ζωη που περιμενει εκει εξω...

Γυρισα στη σκηνη και επεσα να κοιμηθω με ενα χαμογελο μεχρι τα αυτια.
Η ωρα ηταν 11.30 το βραδυ αλλα το φως εξω ηταν οπως το σουρουπο στην Ελλαδα! 
Καλωσηρθες στη γη που δεν νυχτωνει ποτε φιλε μου....


\end{document}
