\documentclass[]{book}
\usepackage{fontspec}
\usepackage{hyperref}
\usepackage{graphicx}

\hypersetup{pdfborder=0 0 0, colorlinks=true, linkcolor=red}
\setmainfont[Ligatures=TeX]{Times New Roman}

%\setlength\parindent{0pt}
%\setlength\parskip{1ex plus 2pt minus 1pt}
\newcommand\dialogue[1]{\par\noindent--~\textit{#1}}

\newcommand\photo[1]{\noindent\includegraphics[width=\textwidth]{photos/#1}}

\title{72ος Παράλληλος: Η μοναξιά του αναβάτη}
\author{Ταξιδευτής}

\begin{document}
\maketitle

\chapter*{Εισαγωγη}

Αυτο το ταξιδι το χρωσταω σε ολους τους υπεροχους ανθρωπους που συναντησα καθ οδον, σε φιλους παροντες και εκεινους που δεν ειναι πια εδω... Christie, Κωστα, Γιαννη, Ion, Christian, Asa, Οle, Marko, Oili, Annika, Marek, Αlex, Dragan, Πετρο, Θαλεια, Κωστη, Μαθο, Πανο, Μανωλη, Γωγω... Χωρις εσας πολυ απλα δεν θα υπηρχε το ταξιδι τουτο και γι' αυτο και σας ανηκει.

Στο ραφι απεναντι μου, εκει, αναμεσα σε βιβλια παρτιτουρες και καθε λογης ματζαλα, εχω ενα σημειωματαριο με τους χαρτες απο το ταξιδι, γεματο με σημειωσεις για σκεψεις, εικονες, συναισθηματα. Ειναι καιρος τωρα που προσπαθω να το αποφυγω το ατιμο. Το εχω αφησει εκει και κανω οτι δεν το βλεπω ομως νοιωθω να με κοιταει απο απεναντι επικριτικα σαν να μου λεει ``Αντε! Ξεκινα! Γραψε!''

Ειναι αληθεια οτι καθε φορα που καθομαι να γραψω για ενα ταξιδι δυσκολευομαι. Αυτη τη φορα ομως ειναι διαφορετικα. Δεν ξερω....
Δεν ξερω αν μπορω να γραψω αρκετα καλα ωστε να μεταφερω το τι εζησα. Νοιωθω λιγος αυτη τη φορα γαμωτο!
Γραφω, σβηνω και ξαναγραφω, ξανασβηνω και τα παραταω. Γιατι ειναι τοσο δυσκολο να παρει η ευχη;

Αυτη η φωνη μεσα στο μυαλο δεν λεει να σωπασει.

\dialogue{Ξερεις το λογο Νικο.}
\dialogue{Σκασε.}
\dialogue{Ξερεις τι πρεπει να κανεις για να γραψεις.}
\dialogue{Ασε με!}
\dialogue{Για να γραψεις για αυτο το ταξιδι πρεπει να μπορεσεις να τα πεις ολα. Να παραδεχτεις αληθειες δυσκολες, να αντιμετωπισεις φαντασματα, να μηδενισεις κοντερ. Αντεχεις ρε φιλε; Εχεις τα αντερα να μην κρυφτεις απο τον ιδιο σου τον εαυτο; Εδω σε θελω....}

Ναι λοιπον ειναι καιρος να γραψω γιατι ολα αυτα μεσα δεν χωρουν πια...
Δεν γινεται να το αποφευγω πια... Στο ποτηρι εχει απο ωρα μπει ενα καλο single malt και απ' τα ηχεια η μοναδικη φωνη του Sivert Hoyem ερχεται να βαλει φωτια στα συναισθηματα, να πυρωσει μνημες και εικονες. Το soundtrack του ταξιδιου αυτου δεν θα μπορουσε να ξεκιναει αλλιως.... \href{http://goo.gl/xUsd7}{Sivert Hoyem - Give it a whirl}

\begin{verse}
\textit{It's my time to be waiting\\
Locked down, chained in the hole\\
But when my waiting is done I'm gonna give it a whirl...}\\
\end{verse}

Αυτο το ταξιδι ηταν οτι πιο δυσκολο εχω ζησει. Γιατι ομως; Ουτε ο πρωτος ουτε ο τελευταιος που πηγε στο Βορειο Ακρωτηρι ημουν.
Αλλοι πηγανε και περασαν ζαχαρη και ολα καλα και ωραια και χαλαρα και σουπερ γουαου.
Οχι εγω. Μπορει να ηταν Ονειρο ζωης, μια μοναδικη εμπειρια που θα κουβαλαω μεχρι να χτυπησω καρτα απο το ματαιο τουτο κοσμο, να ηταν κατι που χρωστουσα σε εμενα και σε αδερφια που δεν μπορεσαν να πανε ποτε, αλλα ευκολο ΔΕΝ ηταν...
Σωματικα, νοητικα, ψυχικα. Το ταξιδι αυτο ζητησε τα παντα και ακομα περισσοτερα. Καθε οριο, καθε κολλημα, καθε φοβια, καθε νοοτροπια. Οσο το αγαπησα τοσο το μισησα και οσο το θελησα αλλο τοσο το μετανοιωσα. Ψαχνετε να βρειτε λογικη; Διαβαστε κατι αλλο καλυτερα γιατι εδω δεν παιζει να τη βρειτε...

Αγκαλιασα το αγνωστο γιατι πολυ απλα δεν ειχα αλλες επιλογες και ειχα το χαμογελο του παλαβου καθως ο Μερφυ εκανε οτι μπορουσε για να μου χαλασει τη διαθεση.
Ειδα μερη αποκοσμα που δεν μπορουσα να φανταστω οτι υπαρχουν σε αυτο το πλανητη.
Εζησα καταστασεις τρελες και παλεψα με καθε ειδους συνθηκες. Απο τον καυσωνα των 42 βαθμων στο πολικο ψυχος του αρκτικου κυκλου, με ηλιο, με βροχη, αερα, χιονι και οτι αλλο μπορει να ριξει η φυση πανω σε ενα ταλαιπωρο μηχανοβιο ταξιδιωτη.
Ατελειωτα χιλιομετρα, στο ελεος και τις διαθεσεις του καιρου, κατω απο ενα βαρυ ουρανο, με συντροφο τη μηχανη μου... Εκει ειναι που κανεις φιλη τη μοναξια δικε μου γιατι αλλιως εισαι χαμενος απο χερι.

Ποσο ειρωνικο ομως ειναι οτι αναμετρηθηκα με οτι μπορουσα να φανταστω, για να ανακαλυψω οτι ο μεγαλυτερος εχθρος μου απο την αρχη ηταν ενας: Εγω.
Στο ταξιδι αυτο ομως τον εκανα μαγκα: τον γκρεμισα κομματι κομματι για να ανακαλυψω αυτο που κρυβοταν πισω απο τη σκουρια και τη μπιχλα της καθημερινοτητας και του βολεματος... Εγινα κατι αλλο απο αυτο που ημουν. Αυτο το ταξιδι σε αλλαζει βαθεια. Ειναι βασανιστικο, σκληρο αλλα τελικα λυτρωτικο...
Στο καθενα δινει την απαντηση σε οσα αναζητα. Σε εμενα εδωσε την απαντηση που γυρευα να βρω, και ας ηταν δυσκολη να την παραδεχτω...

Ταξιδευοντας εμαθα να εμπιστευομαι τους ανθρωπους που συνανταω, να ακολουθω το ενστικτο μου και πανω απο ολα να μη φοβαμαι κανεναν και τιποτα στο κοσμο. Παντα στα ταξιδια γυριζα λιγο διαφορετικος απο οτι εφευγα. 
Ομως αυτη τη φορα κατι αλλαξε οριστικα μεσα μου...
Μετα απο αυτο το ταξιδι.... δεν ξερω... νοιωθω να μπορω να κανω τα παντα και τιποτα πια...

Αν ψαχνετε ενα ταξιδιωτικο με λυρικες περιγραφες για τα φοβερα μερη που ειδαμε, για το ποσο τελεια ηταν ολα και το ποσο ωραια περασαμε, καλυτερα να σταματησετε να διαβαζετε εδω. Αν μη τι αλλο, υπαρχουν ενα σωρο αλλα ταξιδιωτικα εκει εξω απο παιδια που εχουν κανει αυτο το ταξιδι και τα μπορουν να τα πουνε πολυ καλυτερα απο εμενα ολα αυτα.

Αυτη εδω η ιστορια ειναι απλα οι σκεψεις ενος μονου αναβατη που προσπαθησε να κανει πραγματικοτητα ενα μεγαλο ονειρο -εικονες, συναισθηματα, νοηματα, ολα ατακτως ερριμενα.
Μη με παρεξηγησετε... Ισως βαζοντας τα στο χαρτι να βρω μια ακρη και να ξετυλιξω το κουβαρι...
Τι; Νομιζατε οτι θα τη γλυτωσετε ετσι απλα; Γελαστηκατε! Θα παρει χρονο το ταξιδιωτικο αυτο και δεν θα ειναι ευκολο.
Θα βγει με κοπο αλλα ισως και να αξιζε στο τελος -οπως εγινε και με το ιδιο το ταξιδι.

Σχολια, ευχες, παρατηρησεις, μπινελικια, ευχες και καταρες, ριψατε εντος. Ολα καλοδεχουμενα, σημαντικα, απολυτως απαραιτητα τελικα.

``There was nowhere to go but everywhere, so just keep on rolling under the stars.''

\photo{1.jpg}

\chapter*{Day 0 -- Αθήνα (GR) - Ancona (IT)}

Το ταξιδι αυτο ηταν κατι που υπηρχε παντα μεσα στο μυαλο μου. Απο τοτε που θυμαμαι τον εαυτο μου σε δυο ροδες, παντα αναφερομουν στο ταξιδι στο Βορειο Ακρωτηρι σαν Το Ταξιδι. Eνα απο αυτα τα ονειρα που πρεπει να κανει καποιος πραγματικοτητα πριν κλεισει τα ματια.

Η φωτια σιγοκαιγε χρονια μεσα στο μυαλο αλλα καθε φορα που πηγαινε να φουντωσει την εσβηνα λεγοντας ``αστο μωρε, εχουμε χρονο, τωρα εξαλλου εχουμε το Χ, το Ψ, το Ω''. Καθε φορα μια δικαιολογια, καθε φορα μια αναβολη. Ωσπου ηρθε εκεινο το τηλεφωνημα εκεινη την ρημαδα Κυριακη του Σεπτεμβριου απο τον κολλητο.

\dialogue{Ελα ρε Νικο, που εισαι ρε και σε ψαχνω;}
\dialogue{Ελα φιλε, ειμαι σε κατι συγγενικες υποχρεωσεις. Τι εγινε;}
\dialogue{Δεν τα εμαθες; Ο Πανος ρε... Εφυγε...! Συνορα Ουκρανιας με ενα διερχομενο φορτηγο.}
\dialogue{........}

Ο καλος μου φιλος ο Πανος... Αδερφος ταξιδευτης και αυτος ειχε φαει τα Βαλκανια και την Ευρωπη με το κουταλι, και πλεον εγραφε ροτες για νοτια. Με εκεινον ειχα ξεκινησει το πρωτο μεγαλο μου ταξιδι στην Ευρωπη το 2007, με εκεινος ειχα κανει ενα σωρο βολτες και ταξιδακια οσο βρισκοταν Ελλαδα πριν μετακομισει Βουλγαρια. Μια μερα πριν μιλουσαμε γαμωτο! Και τωρα....

Εκει κατι αλλαξε μεσα μου. Ο χρονος που πριν ηταν απλετος τωρα ειχε τελειωσει. Οι δικαιολογιες, οι αναβολες εγιναν χιλια κομματια με τον πιο βιαιο τροπο. Τωρα. ΤΩΡΑ.
Μην αναβαλλεις δικε μου το Ονειρο γιατι μια μερα θα ειναι πολυ αργα. Παντα υπαρχει τροπος, αλλα ειναι πιο ευκολο να βρισκεις δικαιολογιες για να λες ``δεν μπορω''... Τα ονειρα ζητανε βλεπεις πολλα: θυσιες, κοπο και αιμα που βραζει. Πως να το κανεις οταν εισαι βολεμενος στο τακτοποιημενο, καθαρο, αποστειρωμενο κουτακι της ζωης σου; Το κουτι μου ομως εγω το ειχα σπασει και δεν κοιταζα πισω πια.

Γυρισα σπιτι αμιλητος, μουδιασμενος. Ανοιξα τους παλιους χαρτες γιατι ξερω οτι αυτο λατρευε και ξεκινησα να σχεδιαζω ροτες... Το ταξιδι ειχε ξεκινησει ηδη. Ενα ταξιδι που δεν προλαβε γαμωτο αλλα θα το καναμε μαζι.

Δυσκολοι μηνες περασαν απο τοτε, ομως το ονειρο δεν ξεχαστηκε. Δεν μιλησα σε κανεναν. Θα το εκανα μονο οταν ολα θα ηταν πλεον ετοιμα. Ηταν τοσα που θα μπορουσαν να πανε στραβα. Δουλεια, αδεια, χρονος, χρημα, διαθεση, μηχανη και αλλα τοσα απροβλεπτα και ξαφνικα.

Σιγα σιγα ξεκινησαν ετοιμασιες. Ο εξοπλισμος εκστρατειας ηταν απο πριν απολυτα ικανος να αντεξει ενα τετοιο ταξιδι, οποτε απλα αρχισα να μαζευω τα λιγα που μου ελειπαν: ενα καλο αδιαβροχο, ολοσωμο ισοθερμικο, νεα δερματινα, ενα κουζινακι για το μαγειρεμα... Λιγο πριν το καλοκαιρι η αδεια κανονιστηκε, οι διαδρομες αρχισαν να βγαινουν, το προγραμμα κανονιστηκε, τα πλοια εκλεισαν. Οταν ελαβα το πρωτο confirmation απο το πλοιο που θα με πηγαινε απο Δανια - Νορβηγια ενοιωσα μαγικα: το ονειρο επαιρνε πλεον σαρκα και οστα! Θα το εκανα!

Οι ημερομηνιες που επελεξα συγκεκριμενες: απο τα μεσα Ιουλιου μεχρι αρχες Αυγουστου. Ο λογος απλος: εκει πανω ο καιρος απο τον Αυγουστο και μετα αγριευει και θελει προσοχη. Η καλυτερη περιοδος για να επισκευτει καποιος αυτες τις χωρες ειναι απο τελη Ιουνιου μεχρι μεσα Αυγουστου. Χα! Ετσι ελεγαν.... Καθως οι μερες περνουσαν και εφτανε ο καιρος, το αγχος για το τι πηγαινα να κανω ολοενα και μεγαλωνε. Το να λες ``παω Βορειο Ακρωτηρι'' ειναι τρεις απλες λεξουλες, αλλα οταν το κανεις πραγματικα, εκει δικε μου ειναι ΕΝΤΕΛΩΣ αλλη ιστορια...

Η τελευταια βδομαδα πριν το ταξιδι περασε μεσα σε ενα πανικο δουλειας, πιεσης να τα προλαβω ολα και αγχους για να μην ξεχασω τιποτα.

Πεμπτη μεσημερι στο γραφειο και κατι περισσοτερο απο 24 ωρες για την αναχωρηση! Παρασκευη βραδυ θα επαιρνα το πλοιο απο Ηγουμενιτσα για Ανκονα. Ολα ηταν ετοιμα. Μεχρι που χτυπησε το τηλεφωνο:

\dialogue{Ναι γεια σας, τηλεφωνω απο την Greek Ferries. Eχετε κλεισει ενα εισιτηριο απο Ηγουμενιτσα για Ανκονα για αυριο το βραδυ;}
\dialogue{Ναι.}
\dialogue{Θελουμε να σας ενημερωσουμε οτι το πλοιο ακυρωνεται!}

Αφου συνηλθα απο το μινι εγκεφαλικο εγινε χαμος:

\dialogue{Μα τι μου λετε;;;;! Αυριο φευγω για 23 μερες ταξιδι και ολα ειναι κλεισμενα ηδη! Και μου λετε μια μερα πριν οτι ακυρωνεται το πλοιο;}
\dialogue{Ε ναι, μας ενημερωσαν απο την Superfast οτι το πλοιο αυτο πλεον δεν θα κανει αυτο το δρομολογιο. Μπορειτε να παρετε ενα αλλο πλοιο απο Ηγουμενιτσα στις 5 }το απογευμα.
\dialogue{Αυτο ΔΕΝ γινεται κυρια μου! Μεχρι το μεσημερι αυριο εργαζομαι και δεν μπορω να το φυγω νωριτερα. Αυτος ηταν και ο λογος που επελεξα το βραδυνο δρομολογιο.}
\dialogue{Τι να σας πω, υπαρχει και αλλο ενα που φευγει 12 το μεσημερι απο Πατρα.}
\dialogue{.....}

Εκλεισα γιατι αν μιλουσα με ενα ντουβαρι θα ειχα καλυτερη συνεννοηση και αρχισα πανικοβλητος να κοιταω εναλλακτικες. Πως θα προλαβαινα αυτα τα δρομολογια; Θα επρεπε να φυγω απο πρωι και να καταφερω να μην παω καν στο γραφειο -κατι παρα πολυ δυσκολο! Ακομα δεν ξεκινησαμε και αρχισε ο Μερφυ να κανει παιχνιδι; Τη βαψαμε!

Εκει ως απο (βαυαρικης) μηχανης θεος εμφανιστηκε ο αδερφος Κωστης. Του εξηγησα το προβλημα και ανελαβε δραση. Με τα κοννε του στις ακτοπλοικες της Πατρας εμαθε οτι το ιδιο το πλοιο που θα επαιρνα απο Ηγουμενιτσα το βραδυ εφευγε 5 το απογευμα απο την Πατρα! Απλα δεν εκανε σταση πλεον Ηγουμενιτσα αλλα πηγαινε Ανκονα απευθειας, κατι που η ηλιθια του πρακτορειου δεν γνωριζε καν!

Την καλω επι τοπου και τη βαζω να ψαξει με την Superfast. Αφου ενημερωνεται οτι οντως ετσι ειναι, κλεινουμε θεση στο νεο δρομολογιο και σημαινει ληξη συναγερμου! Επιστροφη στο σπιτι για ενα τελικο ελεγχο στα συστηματα της Αυρας: GPS, θερμαινομενα, MP3, ασφαλειες, φωτα...

Εχετε ακουσει για κανεναν που απλα εκανε μια αλλαγη λαδια και πηγε Βορειο Ακρωτηρι; Οχι; Ε τωρα θα ακουσετε! Το τελευταιο σερβις που της ειχα κανει ηταν στις 88.000 χλμ οποτε τωρα που ειχε αισιως μπει στις 100.000 το μονο που χρειαζοταν ουσιαστικα ηταν μια αλλαγη λαδιων που της χρωστουσα. Ολα τα αλλα ηταν μια χαρα! Τι αλλο να χρειαζοταν για το ταξιδι;

2 το πρωι, μεσα σε μια ζεστη, υγρη νυχτα του Ιουλιου, και οι βαλιτσες κουμπωναν πανω στην Αυρα.
Το μυαλο ακομα δεν μπορουσε να το χωρεσει αυτο που πηγαινα να κανω, αλλα τωρα δεν υπηρχε επιστροφη...
Η μερα της μεγαλης φυγης ξημερωνε συντομα.

\photo{2.jpg}

Το πρωι στο γραφειο τρελη πιεση να κλεισουν οι εκκρεμοτητες και πανικος να φυγω στην ωρα μου. 
Το αγχος του φευγιου. Πως θα γινει μια φορα να φυγω με το πασο μου; Τρεχουμε και δεν φτανουμε μια ζωη! 
Ο χρονος που νομιζεις οτι εχεις και τελικα ανακαλυπτεις οτι ποτε δεν ειναι αρκετος.

Ο Κωστης μου στελνει μηνυμα: ``Μην ξεχαστεις! Το πλοιο φευγει απο το νεο λιμανι στη Πατρα.'' Ιδεα δεν ειχα! Ποτε εφτιαξαν νεο λιμανι; Το 2010 ειχα φυγει απο το παλιο κλασσικο στην εισοδο της πολης. Το αλλαξαν;; 

Ευτυχως που με ενημερωσε γιατι ιδεα δεν ειχα! Μου δινει σχετικες οδηγιες και κανονιζουμε να συναντηθουμε στην εξοδο της περιφερειακης Πατρων για να με παει μεχρι το πλοιο. 

Κατεβαινω στη μηχανη σαν τον κυνηγημενο. Αγχος, στρες, πιεση, ουφ! Βαζω το κλειδι στη μιζα και... κοντοστεκομαι. Αυτο ειναι! Γυριζω μιζα και ξεκιναω να ζησω ενα ονειρο. Ενα κουμπι με χωριζει. Κλικ. Το μπασο γρυλισμα της Αυρας αντηχει στο υπογειο γκαραζ και εγω χαμογελαω σαν χαζος! Φυγαμε!

Η μηχανη κυλουσε σβελτα στην Εθνικη για Πατρα και το χαμογελο δεν ελεγε να φυγει απο τα χειλη μου. Η διαδρομη εκανε οτι μπορουσε να μου χαλασει τη διαθεση αλλα δεν εδινα σημασια. Μια βδομαδα τωρα οι θερμοκρασιες ηταν πανω απο 40 βαθμους και σε τετοιες συνθηκες το ταξιδι ειναι δραματικο. Η ζεστη χτυπουσε κοκκινα και ο αερας ηταν τοσο καυτος που εκλεισα ζελατινα για να δροσιστω.  Η υπνηλια απο την ανυποφορη ζεστη και τη βαρεμαρα καραδοκουσε και τα χιλιομετρα παντα φαινονται περισσοτερα σε τετοιες συνθηκες..

Μετα απο εναν αιωνα περνουσα πλεον εξω απο τη Πατρα. Η ωρα ηδη περασμενες 4! Ειχα κατι λιγοτερο απο μια ωρα για το πλοιο.  Στη περιφερειακη Πατρων το αγχος για το αν προσπερασα τη σωστη εξοδο η οχι αρχισε να ερχεται ξανα. Ημουν οριακα σε χρονο. Αν επρεπε να γυρισω πισω και να ψαχνω το πλοιο το ειχα χασει!

Πανω που αρχισα να σκεφτομαι οτι ισως να ειχα κανει βλακεια ηρθε η θεα της σωστης πινακιδας να με ανακουφισει!

Ο Κωστης με το Φωτη με περιμεναν απο ωρα στη γεφυρα. Χαμογελο. Υπεροχο συναισθημα κ τοσο τιμητικο να σε μετρανε τοσο οι φιλοι που να καθονται να περιμενουν κατω απο το λιοπυρι για παρτη σου! 

Φυγαμε για το λιμανι συνοδεια. Ο Κωστης με την Μπεμπα μπροστα, ο Φωτης με το αγριμι πισω και η Αυρα φορτωμενη στη μεση. Μα ποιος ειμαι επιτελους; 

Το λιμανη ηταν χωμενο κυριολεκτικα στου διαολου το κερατο. Εγω απο μονος μου δεν θα το εβρισκα εγκαιρως με τιποτα. Ο Κωστης εσωσε το ταξιδι, κυριολεκτικα. Τυπικοτητες στο τσεκ ιν και φυγαμε για το πλοιο που ηταν ετοιμο για αναχωρηση.

Η ωρα ειχε περασει. Ο Κωστης γυρισε στο μερος μου και μου ειπε σε τονο υπηρεσιακο ``Φυγε, απο εδω και μετα ειναι ελεγχομενος χωρος. Ζησε το ονειρο σε καθε στιγμη και καλη ανταμωση.'' Ο χρονος ετρεχε και επρεπε να φυγω ομως ξερω τον Κωστη: λακωνικος, ομως δυο κουβεντες του ισοδυναμουν με χιλιες συζητησεις. Αντιο παιδια και θα τα ξαναπουμε.

\photo{3.jpg}

Δεσαμε τη μηχανη στο αμπαρι, διπλα σε ενα Ducati. Που ηταν ολοι οι αλλοι ταξιδιωτες; Ανεβηκα επανω και εκανα την καθιερωμενη βολτα στους οροφους του πλοιου...  Ομορφο, καινουργιο και καθαρο, ομως δεν εβρισκα μερος που να μπορει να ξαπλωσει κανεις. Που ειναι οι καναπεδες ρε παιδια; Βγηκα στο καταστρωμα ενω το πλοιο ειχε πιασει να σαλπαρει. Βλεπω την ακτη να χανεται κ μαζι της νοιωθω να χανονται και οι σκεψεις, το αγχος, τα προβληματα.. Το μονο που εχει σημασια πλεον ειναι το ταξιδι. Καθε μερα και αλλου, αλητεια κ περιπλανηση, αγνωστα μερη και περιπετεια. Να μην ξερεις τι σε ξημερωνει και να εχεις για μπουσουλα τον οριζοντα.

\photo{4.jpg}

Tο πλοιο ηταν σχεδον αδειο παροτι ημασταν στη μεση του καλοκαιριου. Προφανως λιγοι ειναι οι τρελοι που θελουν (και μπορουν πλεον!) καταμεσης του καλοκαιριου να φυγουν απο Ελλαδα. Φανταζομοαι οτι στην επιστροφη για Πατρα οι τουριστες θα κρεμονταν απο τα ρελια σαν τσαμπια! Ετσι ομως τωρα ηταν καπως καταθλιπτικο το θεαμα -μονο ενα τσουρμο πιτσιρικια γερμανακια φτιαχνουν λιγο την ατμοσφαιρα και μου θυμιζουν οτι ο κοσμος φευγει σε διακοπες.

Ο Νορβηγος φιλος Ole μου εστειλε μηνυμα, φτιαχοντας μου τη διαθεση: ο καιρος εκει πολλα υποσχομενος: ηλιος και 21 βαθμοι σημερα, το μεγαλυτερο που ειχαν δει λεει μεχρι τωρα! Τωρα τι του λες;

Η νυχτα επεσε σιγα σιγα και εγω αρχισα να ψαχνω ενα μερος να την πεσω. Μπρος γκρεμος και πισω ρεμα. Στο σαλονι, τηρωντας τις ενδοξες θαλασσινες παραδοσεις, το αircondition δουλευε τερμα γκαζια: οι θερμοκρασιες μπορει να χαροποιουσαν εναν πιγκουινο αλλα οχι εμενα. Απο την αλλη το καταστρωμα ηταν ζεστο αλλα τα πλαστικα παγκακια δεν τα ελεγες ακριβως και ιδανικα για υπνο. Λαγοκοιμαμαι δυο ωρες με τα φωτα φθοριου στα ματια και σηκωνομαι. Επιστροφη στο σαλονι οπου βρισκω μια καπως ευρυχωρη καρεκλα και κοιμαμαι σαν τελικο σιγμα. Superfast και υπνος ...ελευθερας βοσκης δεν πανε μαζι. 

Το πρωινο ξημερωσε με εναν λαμπρο ηλιο να ανατελλει πανω απο τη θαλασσα. Μοναδικο θεαμα να σε ξυπναει κατι τετοιο...

Σηκωθηκα και βγηκα στο καταστρωμα. Πηρα τον κλασσικο βαπορισιο καφε και αραξα σε μια μερια να απολαυσω τη στιγμη. Στο MP3 οι \href{http://goo.gl/HcaG4}{Starsailor} και οι \href{http://goo.gl/6sdDG}{Oasis} (κλικ ντεεεε!) εφτιαχναν το ιδανικο soundtrack της μερας:

\begin{verse}
There's a fever
On the freeway
In the morning

And the lover
Smiling for me
Without warning

There's an outlaw
On the highway
And she's falling

Man I must have been blind
To carry a torch
For most of my life

These days I'm hanging around
You're out of my heart
And out of my town...
\end{verse}

Αραχτος στο καταστρωμα, με το Zen and the art of Motorcycle Maintenance ανα χειρας, υπεροχος ηλιος, ζεστη, καφεδακι, μουσικη και κατω στο γκαραζ η Αυρα πανετοιμη να περιμενει να φυγουμε! Η Τελεια Στιγμη...

Σε λιγο το πλοιο πιανει λιμανι στην Ancona. Τι μας περιμενει καλη μου;

\photo{5.jpg}

\chapter*{Day 1 -- Ancona (IT) - Como (IT) -- 492km}

Η ωρα πανω στα πλοια περναει αργα. Οσοι εχουν ταξιδεψει ετσι ξερουν. Κοιταξα το ρολοι μου. Με τη διαφορα ωρας θα επρεπε να κοντευουμε στο λιμανι. Ομως τριγυρω το μονο που βλεπω ειναι το απειρο μπλε της θαλασσας. Τι ωρα φτανουμε; Η απαντηση απο τον λογιστη του πλοιου με ξαφνιασε: 1 το μεσημερι;! Εγω πως νομιζα οτι φτανουμε 10.30 το πρωι; Η ιδεα μου να χαζεψω στους επαρχιακους δρομους της κεντρικης Ιταλιας τωρα πηγαινε απατη...

Η ωρα τελικα περασε χαζευοντας τους χαρτες για να δω τις διαδρομες που θα ακολουθησω για το προορισμο μου. Τι πιο ομορφο πραγμα απο αυτο; Να ξεκινας τη μερα σου χαζευοντας ολες αυτες τις μικρες πολυχρωμες γραμμουλες πανω στους χαρτες και να προγραμματιζεις αγνωστες διαδρομες για να νυχτωθεις οπου σε βγαλει ο δρομος...

Εγω ομως σημερα ειχα προορισμο: τη λιμνη Como, τη τριτη μεγαλυτερη λιμνη της Ιταλιας στα βορειοδυτικα της χωρας, ακριβως πανω στα Ιταλο-ελβετικα συνορα. Εκει ειναι και το ομωνυμο φημισμενο χωριο που αποτελει ενα (αρκετα κυριλε/ακριβο) τουριστικο θερετρο για τους Ευρωπαιους και οχι μονο, καθως ειναι και το μερος που ζει ο ...George εδω και χρονια. (Μη με ρωτησετε ποιος ειναι ο George ετσι; )
Ηθελα να επισκευτω την περιοχη εδω και χρονια και να που ειχε ερθει η ευκαιρια να το κανω. Βεβαια με το πλοιο να δενει τοσο αργα στην Ancona και εχοντας να διασχισω ουσιαστικα ολη την Ιταλια δεν θα εβλεπα και πολλα απο το Como αλλα δεν πειραζει. Ακομα και ετσι θα επαιρνα μια μικρη γευση!

\photo{6.jpg}

Το πλοιο ειχε πιασει να μπαινει στο λιμανι της Ancona και βρηκα την ευκαιρια να χαζεψω τριγυρω...
Δεν θα το ελεγα ακριβως ωραιο το μερος -ενα μικρο, μαλλον αδιαφορο λιμανακι ηταν αλλα ακομα και ετσι μερικα κτηρια διατηρουσαν την περιφημη Ιταλικη φινετσα και ομορφια...

\photo{7.jpg}

Κατεβηκα στο γκαραζ για να ετοιμαστω. Οσοι με ξερουν γνωριζουν οτι με το χρονο εχω μια πολυ περιεργη σχεση: μια ζωη αργω σε οτι και να κανω -και αυτη η φορα ΔΕΝ ηταν η εξαιρεση.

Τα παιδια με το Multistrada που ηταν παρκαρισμενο διπλα μου ειχαν ηδη κατεβει κατω και περιμεναν υπομονετικα να μαζεψω το τσαντιρι για να φυγουν. Οι παρκαδοροι του πλοιου θελοντας να διευκολυνουν την κατασταση επιασαν να λυνουν τους ιμαντες απο τις μηχανες για να φυγουμε μια ωρα αρχιτερα -ποσο μα ποσο θα το μετανοιωνα αυτο πολυ συντομα....

Για να βοηθησω με τη σειρα μου το ζευγαρι να φυγει μετακινησα την Αυρα 2 μετρα πιο πισω. Ομως εκει που την πηγα το πατωμα ειχε ενα μικρο εξογκωμα. 
Οχι και τοσο σημαντικο θα μου πειτε. Θα συμφωνουσα αν δεν υπηρχαν δυο μικρες λεπτομερειες:

1. Η μηχανη λογω βαρους απο τις βαλιτσες (και λογω εγκεφαλικης βλακειας δικης μου που ΔΕΝ εσφιξα την προφορτιση πριν το ταξιδι) εκανε την αναρτηση να βυθιζεται ελαφρα, ετσι ωστε οταν καθοταν στο stand πλαγιαζε απο λιγο εως ελαχιστα.
2. Οι ιμαντες δεσιματος ειχαν λυθει εντελως.

Ετσι εβαλα το stand αφου ειχα κανει πισω να φυγουν τα παιδια και πηγα πισω στη βαλιτσα να κλεισω απλα το καπακι. Καπου εκει εκανα την ενδιαφερουσα παρατηρηση οτι η μηχανη στηριζοταν σε οριακα ορθια θεση και καθως ακουμπαω το καπακι της βαλιτσας.... ο Τιτανικος βυθιζεται!
340+ κιλα πλαστικων και μεταλλων εφυγαν προς τα δεξια με χαρη που θα ζηλευε και ο ελεφαντας του Circo Medrano. Προφανως οι προσπαθειες να σταματησω το θηριο τραβωντας απο αριστερα ηταν εντελως αναξιες λογου και ετσι βρεθηκα να κοιταω τη μηχανη φαρδια πλατια στο καταστρωμα του πλοιου!

Τι πιο ωραια αρχη για το μεγαλυτερο ταξιδι που θα εκανα ποτε; 
Τελικα αν δεν πεσει η μηχανη μια φορα σε ενα ταξιδι μου δεν θα παει καλα. Το 2010 ειχε συμβει το ιδιο ακριβως 2 φορες αλλα το ταξιδι ειχε παει υπεροχα. Απο την αλλη να πω τωρα οτι χαιρομουν θα ημουν ψευτης...

Σηκωσαμε τη μηχανη πανω μαζι με ενα νταλικιερη που προθυμοποιηθηκε και ...ευτυχως σχεδον καμια ζημια! Η μηχανη ειχε κατσει πανω στη δεξια βαλιτσα σωζοντας τα χειροτερα και το μονο αλλο που ειχε βρει ηταν ενα πολυ μικρο πλαστικο στο δεξι φλας που γεμισε γρατζουνιες. Παλι καλα!

Αυτο που ειχε παθει ομως καλη ζημια ηταν το μεγαλο δαχτυλο μου στο δεξι χερι που πρηστηκε σχεδον αμεσως και δεν μπορουσε να κλεισει. Κακωση; Καταγμα; Το σιγουρο ηταν οτι ποναγε διαολεμενα. Πως θα ταξιδευα ετσι 12.000 χιλιομετρα; Αν ηξερα ομως τι θα ακολουθουσε μεχρι να τελειωσει αυτη η απιστευτη περιπετεια μαλλον θα το προσπερνουσα αυτο ως κατι αναξιο λογου βρε αδερφε...

Καθως ομως η Αυρα ρολαριζε τις ροδες της για μια ακομα φορα επι Ιταλικου εδαφους τα ειχα ξεχασει ολα και χαμογελουσα σαν χαζος...

\photo{8.jpg}

Σταματησα στο πρωτο βενζιναδικο που βρηκα για να ταισω τα αλογα και να αποφασισω για τη διαδρομη. Ηθελα πολυ να βγω στους επαρχιακους δρομους αλλα το GPS μου εκοψε τα ποδια: 10+ ωρες συνεχους οδηγησης και η ωρα ηταν ηδη 2 το μεσημερι. Στην καλυτερη των περιπτωσεων και χωρις καμια σταση (κατι πρακτικα αδυνατον) θα εφτανα τις 12 το βραδυ και βεβαια χωρις να εχω κανονισει καπου να μεινω δεν ηταν και η καλυτερη ιδεα να ψαχνω μεσα στη μαυρη νυχτα καπου να κοιμηθω...

Συνεπως autostrada και ξερο ψωμι για σημερα δυστυχως.
Και το δυστυχως οχι γιατι ειναι κανενας παλιοδρομος (αν και δεν μπορει φυσικα να συγκριθει με το ΥΠΕΡΤΑΤΟ μεγαλειο της Κορινθου - Πατρων) αλλα γιατι εγω βαριεμαι τρομερα τις εθνικες οδους και τις ευθειες και τις αποφευγω οπως ο Lemmy τη Britney Spears. Αναγκαιο ομως κακο σημερα η εθνικη αν ηθελα να φτασω σε καποια λογικη ωρα στο προορισμο μου.

Περα απο τη πλακα βεβαια η autostrada ειναι δρομαρα και αυτο δεν κρυβοταν με τιποτα...

\photo{9.jpg}

...ενω και οι εικονες της επαρχιακης ζωης τριγυρω εκαναν οτι μπορουσαν για να μην πληττω.

\photo{10.jpg}

Συντομα επιασα ρυθμο και η μηχανη αρχισε να τρωει τα χιλιομετρα γουργουριζοντας χαρουμενη, κατι που δεν μπορουσα να πω και για εμενα. Τρελη ζεστη, ηλιος και δερματινα δεν ειναι και ο καλυτερος συνδιασμος. 
Οι στασεις για νερο ηταν συνεχεις και παροτι ανεβαινα ολοενα και πιο βορεια η καψα του μεσογειακου καλοκαιριου δεν ελεγε να κοπασει παρα τα συννεφα που εβλεπα να μαζευονται πλεον στον οριζοντα. 
Τωρα ειχε και ζεστη και μουνταδα!

\photo{11.jpg}

Βγηκα απο την autostrada και προετοιμαστηκα για τις πραξεις ΧΧΧ που θα μου εκαναν στα διοδια: 27 ευρω για 400+ χιλιομετρα; Χμμμ! Περιμενα χειροτερα.
Το απογευμα ειχε πιασει να πεφτει απο ωρα και ανηφοριζα πλεον προς το Como. Ειχε βρεξει εδω πριν λιγο... Ο ουρανος γκριζος πλεον με βαρια συννεφα -θα προτιμουσα λιακαδα, αλλα οι μυρωδιες της φυσης και η ατμοσφαιρικη εικονα που εφτιαχνε ο ουρανος ηταν εξισου ομορφες...

Καθως κατεβαινα απο την εξοδο του δρομου προς τη λιμνη, το Como αρχισε να αποκαλυπτει το λογο γιατι θεωρειται ενας απο τους πιο ομορφους καλοκαιρινους προορισμους στην Ευρωπη.
Εκανα μια σταση στην ακρη του δρομου και τα ματια δεν μπορουσαν να χορτασουν τις εικονες...
Εικονες μιας ταινιας προσεχως...

\photo{12.jpg}
\photo{13.jpg}

To τοπιο ηταν απλα μαγευτικο! Με καποιο παραξενο τροπο η μουνταδα του ουρανου ταιριαζε απολυτα στη στιγμη εκεινη και εγω καθισα σιωπηλος χαμενος στη στιγμη...

\photo{14.jpg}

Θα ηθελα να κανω το γυρο της λιμνης αλλα η ωρα δεν το επετρεπε. Η νυχτα επεφτε πλεον και εγω ακομα δεν ειχα βρει που θα εμενα το βραδυ. Παρολαυτα, η διαδρομη που ξεκινουσε γυρω απο το Como εδειχνε πολλα υποσχομενη και στο μυαλο μου ανανεωσα το ραντεβου μου με την περιοχη για καποια αλλη φορα στο μελλον. Τελικα αν κατσω και βαλω ενα σημαιακι στα μερη που εχω αφησει στα ``προσεχως'' στο τελος δεν θα βλεπω χαρτη για χαρτη!

\photo{15.jpg}
\photo{16.jpg}

Το GPS εδωσε τη λυση για καποιο κοντινο camping: Camping Europa, ενα συμπαθητικο καμπινγκ ακριβως στην εξοδο του χωριου.

\photo{17.jpg}

Δυστυχως ομως η ρεσεψιον οσον αφορουσε την εξυπηρετηση και τη συννενοηση βρισκοταν στο απολυτο μηδεν: ενας βαριεστημενος πιτσιρικας (που ειμαι σιγουρος οτι θα προτιμουσε να ειναι οπουδηποτε αλλου εκεινη την ωρα) μετα βιας εδειχνε ενδιαφερον να εξυπηρετησει ενα ταλαιπωρημενο μηχανοβιο που στεκοταν μπροστα του.
Με τα ελαχιστα ιταλικα μου ευτυχως συνεννοηθηκαμε και μετα τις τυπικες διαδικασιες ξεκινησα να στησω τα συμπραγκ... Επ! Οχι τοσο γρηγορα φιλε μου! Ο καιρος δεν ειχε πει την τελευταια του λεξη ακομα...
Ισα που ειχα προλαβει να απλωσω το εσωτερικο της σκηνης, οταν αρχισαν να πεφτουν δυνατοι κεραυνοι πανω απο τη λιμνη και η ατμοσφαιρα γεμισε τη γνωριμη μυρωδια του οζοντος -σημαδι οτι εντος δευτερολεπτων θα ξεκινουσε ο κατακλυσμος του Νωε!
Τα πυκνα φυλλωματα των δεντρων απο πανω μου θα κρατουσαν μια μικρη βροχουλα αλλα κατι μου ελεγε οτι αυτο που ερχοταν μονο μικρο δεν θα το ελεγες... 

Μαζεψα αρον αρον τη σκηνη και ετρεξα πισω στη ρεσεψιον ενω η βροχη αρχισε να δυναμωνει. 
Καλυβακι; Λαστ γιαρ! Και τωρα τι κανουμε; Να στησω υπο βροχη ηταν κατι που θα προτιμουσα να αποφυγω...
Εν μεσω βροχης το πιτσιρικι μου προτεινε ενα τροχοσπιτο. Πως ειπατε; 
Το καμπινγκ ειχε μερικα παναρχαια, ταλαιπωρημενα τροχοσπιτα στημενα εκει μονιμα ως λυση αναγκης για οσους ηθελαν καλυβακι αλλα δεν εβρισκαν.

Μου εδωσε ενα κλειδι και μου ειπε το νουμερο που επρεπε να βρω. Που ειναι ρε παιδια το νουμερο 12; Δεν υπηρχε πουθενα! Μετα απο τρια τεσσερα πανω κατω στη ρεσεψιον φορωντας τα δερματινα και ιδρωνοντας σαν γουρουνι στο σακι ΠΑΡΑ τη βροχη που επεφτε τελικα καταφερα να εντοπισω το περιφημο τροχοσπιτο. 

Κατι που τελικα μακαρι να μην το ειχα κανει και ποτε! Ολη μου τη ζωη την εχω περασει σε καμπινγκ -ε λοιπον δεν εχω δει ΠΟΤΕ αλλοτε ενα τροχοσπιτο σε τοσο αθλια κατασταση. Τα παντα ηταν σκουριασμενα και σπασμενα. Πομολα, μεντεσεδες στις πορτες, μπρατσα στηριξης στα παραθυρα... Το μεγαλυτερο μερος του τροχοσπιτου εκτελουσε χρεη (σκουπιδ)αποθηκης με οτι σαβουρα μπορει να φανταστει κανεις πεταμενη μεσα ενω το κρεβατι ειχε επανω κατι τεραστιους λεκεδες που θα προτιμουσα να μην ξερω απο τι εγιναν και μια εκλεκτη κολεξιον απο ``Πεθαμενα ζωυφια αγνωστου ταυτοτητας'' του 2012. 
Σαν να μην εφταναν αυτα, το τροχοσπιτο εδειχνε να ...κατοικειται ηδη! Στο τοιχο υπηρχαν κατι ρωσικα εικονισματα και σε μια καρεκλα κατι παλια ρουχα και παντοφλες καποιου αντρα.

Μεσα σε χρονο μηδεν ημουν και παλι στη ρεσεψιον: αστο φιλε, θα το ρισκαρω με τη καταιγιδα χιλιες φορες παρα αυτο το πραγμα!
Εξαλλου με ολα αυτα η μπορα ειχε σταματησει πλεον οποτε χιλιες φορες η σκηνη μου και η φυση παρα να κλειστω σε ενα αθλιο κουτι.

Αραξα τη μηχανη σε ενα ωραιο σημειο που δεν ειχε βραχει πολυ και αρχισα να στηνω. 
Λιγα μετρα πιο περα δυο αλλες ταξιδιαρες μηχανες ηταν αραγμενες: ενα RT1100 και ενα GS1150, που αν εκρινα απο τα αυτοκολλητα και τη κατασταση τους μαλλον ειχαν να πουν πολλες ιστοριες για ταξιδια. 
Βρετανικες πινακιδες με Ουαλλικα διακριτικα! Συντοπιτες.

Καθως το βραδυ ειχε πεσει πλεον το τσαρδι ηταν ετοιμο και ειχα ηδη πιασει να μαγειρεψω κατι για δειπνο...
Καπου εκει ηρθαν και οι Ουαλλοι και κατσαμε να τα πουμε υπο τη συνοδεια παγωμενης μπυριτσας. 
Πολυ ωραιοι τυποι! Παλιοσειρες μηχανοβιοι ταξιδιωτες που ειχαν φαει την Ευρωπη με το κουταλι. Ειχαν ξεκινησει απο Αγγλια πριν μια βδομαδα και αυριο θα ξεκινουσαν να δουν τα πασα της Ελβετιας οπως και εγω! Οι ιστοριες και οι διαδρομες στους χαρτες εδιναν και επαιρναν μεχρι αργα το βραδυ...

Κρατηστε τα πιο κυριλε εστιατορια και τα πιο γκουρμε φαγητα σας. Εγω ετσι τωρα ημουν ο πιο ευτυχισμενος ανθρωπος του κοσμου. Μια σκηνη, η μηχανη διπλα, οι μποτες μου, αναπαντεχες συναντησεις και ενα ζεστο πιατο φαι στο τελος της μερας. Παραδεισος! 
Αυριο ξημερωνε μια νεα μερα...

\photo{18.jpg}

\end{document}
