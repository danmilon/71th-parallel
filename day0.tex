\chapter{Day 0 -- Αθήνα (GR) - Ancona (IT)}

Το ταξιδι αυτο ηταν κατι που υπηρχε παντα μεσα στο μυαλο μου. Απο τοτε που θυμαμαι τον εαυτο μου σε δυο ροδες, παντα αναφερομουν στο ταξιδι στο Βορειο Ακρωτηρι σαν Το Ταξιδι. Eνα απο αυτα τα ονειρα που πρεπει να κανει καποιος πραγματικοτητα πριν κλεισει τα ματια.

Η φωτια σιγοκαιγε χρονια μεσα στο μυαλο αλλα καθε φορα που πηγαινε να φουντωσει την εσβηνα λεγοντας \textit{``αστο μωρε, εχουμε χρονο, τωρα εξαλλου εχουμε το Χ, το Ψ, το Ω''}. Καθε φορα μια δικαιολογια, καθε φορα μια αναβολη. Ωσπου ηρθε εκεινο το τηλεφωνημα εκεινη την ρημαδα Κυριακη του Σεπτεμβριου απο τον κολλητο.

\bigskip
\dialogue{Ελα ρε Νικο, που εισαι ρε και σε ψαχνω;}
\dialogue{Ελα φιλε, ειμαι σε κατι συγγενικες υποχρεωσεις. Τι εγινε;}
\dialogue{Δεν τα εμαθες; Ο Πανος ρε... Εφυγε...! Συνορα Ουκρανιας με ενα διερχομενο φορτηγο.}
\dialogue{........}
\bigskip

Ο καλος μου φιλος ο Πανος... Αδερφος ταξιδευτης και αυτος ειχε φαει τα Βαλκανια και την Ευρωπη με το κουταλι, και πλεον εγραφε ροτες για νοτια. Με εκεινον ειχα ξεκινησει το πρωτο μεγαλο μου ταξιδι στην Ευρωπη το 2007, με εκεινος ειχα κανει ενα σωρο βολτες και ταξιδακια οσο βρισκοταν Ελλαδα πριν μετακομισει Βουλγαρια. Μια μερα πριν μιλουσαμε γαμωτο! Και τωρα....

Εκει κατι αλλαξε μεσα μου. Ο χρονος που πριν ηταν απλετος τωρα ειχε τελειωσει. Οι δικαιολογιες, οι αναβολες εγιναν χιλια κομματια με τον πιο βιαιο τροπο. Τωρα. ΤΩΡΑ.
Μην αναβαλλεις δικε μου το Ονειρο γιατι μια μερα θα ειναι πολυ αργα. Παντα υπαρχει τροπος, αλλα ειναι πιο ευκολο να βρισκεις δικαιολογιες για να λες ``δεν μπορω''... Τα ονειρα ζητανε βλεπεις πολλα: θυσιες, κοπο και αιμα που βραζει. Πως να το κανεις οταν εισαι βολεμενος στο τακτοποιημενο, καθαρο, αποστειρωμενο κουτακι της ζωης σου; Το κουτι μου ομως εγω το ειχα σπασει και δεν κοιταζα πισω πια.

Γυρισα σπιτι αμιλητος, μουδιασμενος. Ανοιξα τους παλιους χαρτες γιατι ξερω οτι αυτο λατρευε και ξεκινησα να σχεδιαζω ροτες... Το ταξιδι ειχε ξεκινησει ηδη. Ενα ταξιδι που δεν προλαβε γαμωτο αλλα θα το καναμε μαζι.

Δυσκολοι μηνες περασαν απο τοτε, ομως το ονειρο δεν ξεχαστηκε. Δεν μιλησα σε κανεναν. Θα το εκανα μονο οταν ολα θα ηταν πλεον ετοιμα. Ηταν τοσα που θα μπορουσαν να πανε στραβα. Δουλεια, αδεια, χρονος, χρημα, διαθεση, μηχανη και αλλα τοσα απροβλεπτα και ξαφνικα.

Σιγα σιγα ξεκινησαν ετοιμασιες. Ο εξοπλισμος εκστρατειας ηταν απο πριν απολυτα ικανος να αντεξει ενα τετοιο ταξιδι, οποτε απλα αρχισα να μαζευω τα λιγα που μου ελειπαν: ενα καλο αδιαβροχο, ολοσωμο ισοθερμικο, νεα δερματινα, ενα κουζινακι για το μαγειρεμα... Λιγο πριν το καλοκαιρι η αδεια κανονιστηκε, οι διαδρομες αρχισαν να βγαινουν, το προγραμμα κανονιστηκε, τα πλοια εκλεισαν. Οταν ελαβα το πρωτο confirmation απο το πλοιο που θα με πηγαινε απο Δανια - Νορβηγια ενοιωσα μαγικα: το ονειρο επαιρνε πλεον σαρκα και οστα! Θα το εκανα!

Οι ημερομηνιες που επελεξα συγκεκριμενες: απο τα μεσα Ιουλιου μεχρι αρχες Αυγουστου. Ο λογος απλος: εκει πανω ο καιρος απο τον Αυγουστο και μετα αγριευει και θελει προσοχη. Η καλυτερη περιοδος για να επισκευτει καποιος αυτες τις χωρες ειναι απο τελη Ιουνιου μεχρι μεσα Αυγουστου. Χα! Ετσι ελεγαν.... Καθως οι μερες περνουσαν και εφτανε ο καιρος, το αγχος για το τι πηγαινα να κανω ολοενα και μεγαλωνε. Το να λες ``παω Βορειο Ακρωτηρι'' ειναι τρεις απλες λεξουλες, αλλα οταν το κανεις πραγματικα, εκει δικε μου ειναι ΕΝΤΕΛΩΣ αλλη ιστορια...

Η τελευταια βδομαδα πριν το ταξιδι περασε μεσα σε ενα πανικο δουλειας, πιεσης να τα προλαβω ολα και αγχους για να μην ξεχασω τιποτα. Πεμπτη μεσημερι στο γραφειο και κατι περισσοτερο απο 24 ωρες για την αναχωρηση! Παρασκευη βραδυ θα επαιρνα το πλοιο απο Ηγουμενιτσα για Ανκονα. Ολα ηταν ετοιμα. Μεχρι που χτυπησε το τηλεφωνο:

\bigskip
\dialogue{Ναι γεια σας, τηλεφωνω απο την Greek Ferries. Eχετε κλεισει ενα εισιτηριο απο Ηγουμενιτσα για Ανκονα για αυριο το βραδυ;}
\dialogue{Ναι.}
\dialogue{Θελουμε να σας ενημερωσουμε οτι το πλοιο ακυρωνεται!}
\bigskip

\noindent Αφου συνηλθα απο το μινι εγκεφαλικο εγινε χαμος:

\bigskip
\dialogue{Μα τι μου λετε;;;;! Αυριο φευγω για 23 μερες ταξιδι και ολα ειναι κλεισμενα ηδη! Και μου λετε μια μερα πριν οτι ακυρωνεται το πλοιο;}
\dialogue{Ε ναι, μας ενημερωσαν απο την Superfast οτι το πλοιο αυτο πλεον δεν θα κανει αυτο το δρομολογιο. Μπορειτε να παρετε ενα αλλο πλοιο απο Ηγουμενιτσα στις 5 }το απογευμα.
\dialogue{Αυτο ΔΕΝ γινεται κυρια μου! Μεχρι το μεσημερι αυριο εργαζομαι και δεν μπορω να το φυγω νωριτερα. Αυτος ηταν και ο λογος που επελεξα το βραδυνο δρομολογιο.}
\dialogue{Τι να σας πω, υπαρχει και αλλο ενα που φευγει 12 το μεσημερι απο Πατρα.}
\dialogue{.....}
\bigskip

Εκλεισα γιατι αν μιλουσα με ενα ντουβαρι θα ειχα καλυτερη συνεννοηση και αρχισα πανικοβλητος να κοιταω εναλλακτικες. Πως θα προλαβαινα αυτα τα δρομολογια; Θα επρεπε να φυγω απο πρωι και να καταφερω να μην παω καν στο γραφειο, κατι παρα πολυ δυσκολο! Ακομα δεν ξεκινησαμε και αρχισε ο Μερφυ να κανει παιχνιδι; Τη βαψαμε! Εκει ως απο (βαυαρικης) μηχανης θεος εμφανιστηκε ο αδερφος Κωστης. Του εξηγησα το προβλημα και ανελαβε δραση.

Με τα κοννε του στις ακτοπλοικες της Πατρας εμαθε οτι το ιδιο το πλοιο που θα επαιρνα απο Ηγουμενιτσα το βραδυ εφευγε 5 το απογευμα απο την Πατρα! Απλα δεν εκανε σταση πλεον Ηγουμενιτσα αλλα πηγαινε Ανκονα απευθειας, κατι που η ηλιθια του πρακτορειου δεν γνωριζε καν! Την καλω επι τοπου και τη βαζω να ψαξει με την Superfast. Αφου ενημερωνεται οτι οντως ετσι ειναι, κλεινουμε θεση στο νεο δρομολογιο και σημαινει ληξη συναγερμου! Επιστροφη στο σπιτι για ενα τελικο ελεγχο στα συστηματα της Αυρας: GPS, θερμαινομενα, MP3, ασφαλειες, φωτα... Εχετε ακουσει για κανεναν που απλα εκανε μια αλλαγη λαδια και πηγε Βορειο Ακρωτηρι; Οχι; Ε τωρα θα ακουσετε!

Το τελευταιο σερβις που της ειχα κανει ηταν στις 88.000 χλμ οποτε τωρα που ειχε αισιως μπει στις 100.000 το μονο που χρειαζοταν ουσιαστικα ηταν μια αλλαγη λαδιων που της χρωστουσα. Ολα τα αλλα ηταν μια χαρα! Τι αλλο να χρειαζοταν για το ταξιδι; 2 το πρωι, μεσα σε μια ζεστη, υγρη νυχτα του Ιουλιου, και οι βαλιτσες κουμπωναν πανω στην Αυρα. Το μυαλο ακομα δεν μπορουσε να το χωρεσει αυτο που πηγαινα να κανω, αλλα τωρα δεν υπηρχε επιστροφη... Η μερα της μεγαλης φυγης ξημερωνε συντομα.

\photo{2.jpg}

Το πρωι στο γραφειο τρελη πιεση να κλεισουν οι εκκρεμοτητες και πανικος να φυγω στην ωρα μου. 
Το αγχος του φευγιου. Πως θα γινει μια φορα να φυγω με το πασο μου; Τρεχουμε και δεν φτανουμε μια ζωη! 
Ο χρονος που νομιζεις οτι εχεις και τελικα ανακαλυπτεις οτι ποτε δεν ειναι αρκετος.

Ο Κωστης μου στελνει μηνυμα: \textit{``Μην ξεχαστεις! Το πλοιο φευγει απο το νεο λιμανι στη Πατρα.''} Ιδεα δεν ειχα! Ποτε εφτιαξαν νεο λιμανι; Το 2010 ειχα φυγει απο το παλιο κλασσικο στην εισοδο της πολης. Το αλλαξαν;; Ευτυχως που με ενημερωσε γιατι ιδεα δεν ειχα! Μου δινει σχετικες οδηγιες και κανονιζουμε να συναντηθουμε στην εξοδο της περιφερειακης Πατρων για να με παει μεχρι το πλοιο. Κατεβαινω στη μηχανη σαν τον κυνηγημενο. Αγχος, στρες, πιεση, ουφ! Βαζω το κλειδι στη μιζα και... κοντοστεκομαι. Αυτο ειναι! Γυριζω μιζα και ξεκιναω να ζησω ενα ονειρο. Ενα κουμπι με χωριζει. Κλικ. Το μπασο γρυλισμα της Αυρας αντηχει στο υπογειο γκαραζ και εγω χαμογελαω σαν χαζος! Φυγαμε!

Η μηχανη κυλουσε σβελτα στην Εθνικη για Πατρα και το χαμογελο δεν ελεγε να φυγει απο τα χειλη μου. Η διαδρομη εκανε οτι μπορουσε να μου χαλασει τη διαθεση αλλα δεν εδινα σημασια. Μια βδομαδα τωρα οι θερμοκρασιες ηταν πανω απο 40 βαθμους και σε τετοιες συνθηκες το ταξιδι ειναι δραματικο. Η ζεστη χτυπουσε κοκκινα και ο αερας ηταν τοσο καυτος που εκλεισα ζελατινα για να δροσιστω.  Η υπνηλια απο την ανυποφορη ζεστη και τη βαρεμαρα καραδοκουσε και τα χιλιομετρα παντα φαινονται περισσοτερα σε τετοιες συνθηκες.. Μετα απο εναν αιωνα περνουσα πλεον εξω απο τη Πατρα. Η ωρα ηδη περασμενες 4! Ειχα κατι λιγοτερο απο μια ωρα για το πλοιο.  Στη περιφερειακη Πατρων το αγχος για το αν προσπερασα τη σωστη εξοδο η οχι αρχισε να ερχεται ξανα. Ημουν οριακα σε χρονο. Αν επρεπε να γυρισω πισω και να ψαχνω το πλοιο το ειχα χασει!

Πανω που αρχισα να σκεφτομαι οτι ισως να ειχα κανει βλακεια ηρθε η θεα της σωστης πινακιδας να με ανακουφισει! Ο Κωστης με το Φωτη με περιμεναν απο ωρα στη γεφυρα. Χαμογελο. Υπεροχο συναισθημα κ τοσο τιμητικο να σε μετρανε τοσο οι φιλοι που να καθονται να περιμενουν κατω απο το λιοπυρι για παρτη σου!  Φυγαμε για το λιμανι συνοδεια. Ο Κωστης με την Μπεμπα μπροστα, ο Φωτης με το αγριμι πισω και η Αυρα φορτωμενη στη μεση. Μα ποιος ειμαι επιτελους; 

Το λιμανη ηταν χωμενο κυριολεκτικα στου διαολου το κερατο. Εγω απο μονος μου δεν θα το εβρισκα εγκαιρως με τιποτα. Ο Κωστης εσωσε το ταξιδι, κυριολεκτικα. Τυπικοτητες στο τσεκ ιν και φυγαμε για το πλοιο που ηταν ετοιμο για αναχωρηση. Η ωρα ειχε περασει. Ο Κωστης γυρισε στο μερος μου και μου ειπε σε τονο υπηρεσιακο \textit{``Φυγε, απο εδω και μετα ειναι ελεγχομενος χωρος. Ζησε το ονειρο σε καθε στιγμη και καλη ανταμωση.''} Ο χρονος ετρεχε και επρεπε να φυγω ομως ξερω τον Κωστη: λακωνικος, ομως δυο κουβεντες του ισοδυναμουν με χιλιες συζητησεις. Αντιο παιδια και θα τα ξαναπουμε.

\photo{3.jpg}

Δεσαμε τη μηχανη στο αμπαρι, διπλα σε ενα Ducati. Που ηταν ολοι οι αλλοι ταξιδιωτες; Ανεβηκα επανω και εκανα την καθιερωμενη βολτα στους οροφους του πλοιου...  Ομορφο, καινουργιο και καθαρο, ομως δεν εβρισκα μερος που να μπορει να ξαπλωσει κανεις. Που ειναι οι καναπεδες ρε παιδια; Βγηκα στο καταστρωμα ενω το πλοιο ειχε πιασει να σαλπαρει. Βλεπω την ακτη να χανεται κ μαζι της νοιωθω να χανονται και οι σκεψεις, το αγχος, τα προβληματα.. Το μονο που εχει σημασια πλεον ειναι το ταξιδι. Καθε μερα και αλλου, αλητεια κ περιπλανηση, αγνωστα μερη και περιπετεια. Να μην ξερεις τι σε ξημερωνει και να εχεις για μπουσουλα τον οριζοντα.

\photo{4.jpg}

Tο πλοιο ηταν σχεδον αδειο παροτι ημασταν στη μεση του καλοκαιριου. Προφανως λιγοι ειναι οι τρελοι που θελουν (και μπορουν πλεον!) καταμεσης του καλοκαιριου να φυγουν απο Ελλαδα. Φανταζομοαι οτι στην επιστροφη για Πατρα οι τουριστες θα κρεμονταν απο τα ρελια σαν τσαμπια! Ετσι ομως τωρα ηταν καπως καταθλιπτικο το θεαμα -μονο ενα τσουρμο πιτσιρικια γερμανακια φτιαχνουν λιγο την ατμοσφαιρα και μου θυμιζουν οτι ο κοσμος φευγει σε διακοπες. Ο Νορβηγος φιλος Ole μου εστειλε μηνυμα, φτιαχοντας μου τη διαθεση.

\medskip
\textit{``Ο καιρος εκει πολλα υποσχομενος: ηλιος και 21 βαθμοι σημερα, το μεγαλυτερο που ειχαν δει λεει μεχρι τωρα!''}
\medskip
Τωρα τι του λες;
\medskip

Η νυχτα επεσε σιγα σιγα και εγω αρχισα να ψαχνω ενα μερος να την πεσω. Μπρος γκρεμος και πισω ρεμα. Στο σαλονι, τηρωντας τις ενδοξες θαλασσινες παραδοσεις, το αircondition δουλευε τερμα γκαζια: οι θερμοκρασιες μπορει να χαροποιουσαν εναν πιγκουινο αλλα οχι εμενα. Απο την αλλη το καταστρωμα ηταν ζεστο αλλα τα πλαστικα παγκακια δεν τα ελεγες ακριβως και ιδανικα για υπνο. Λαγοκοιμαμαι δυο ωρες με τα φωτα φθοριου στα ματια και σηκωνομαι. Επιστροφη στο σαλονι οπου βρισκω μια καπως ευρυχωρη καρεκλα και κοιμαμαι σαν τελικο σιγμα. Superfast και υπνος ...ελευθερας βοσκης δεν πανε μαζι.

Το πρωινο ξημερωσε με εναν λαμπρο ηλιο να ανατελλει πανω απο τη θαλασσα. Μοναδικο θεαμα να σε ξυπναει κατι τετοιο... Σηκωθηκα και βγηκα στο καταστρωμα. Πηρα τον κλασσικο βαπορισιο καφε και αραξα σε μια μερια να απολαυσω τη στιγμη. Στο MP3 οι \href{http://goo.gl/HcaG4}{Starsailor} και οι \href{http://goo.gl/6sdDG}{Oasis} (κλικ ντεεεε!) εφτιαχναν το ιδανικο soundtrack της μερας:


\begin{verse}
\begin{center}
\textit{
There's a fever\\
On the freeway\\
In the morning}

\textit{And the lover\\
Smiling for me\\
Without warning}

\textit{There's an outlaw\\
On the highway\\
And she's falling}

\textit{Man I must have been blind\\
To carry a torch\\
For most of my life}

\textit{These days I'm hanging around\\
You're out of my heart\\
And out of my town...}
\end{center}
\end{verse}


Αραχτος στο καταστρωμα, με το ``Zen and the art of Motorcycle'' Maintenance ανα χειρας, υπεροχος ηλιος, ζεστη, καφεδακι, μουσικη και κατω στο γκαραζ η Αυρα πανετοιμη να περιμενει να φυγουμε! Η Τελεια Στιγμη... Σε λιγο το πλοιο πιανει λιμανι στην Ancona. Τι μας περιμενει καλη μου;

\photo{5.jpg}
