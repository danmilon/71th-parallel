\chapter{Day 4 -- Leverkusen (D) - Aarhus (DK) - 740km}

Ολο το βραδυ δεν σταματησε να βρεχει. Το πρωτο φως της ημερας, χλωμο και ξεθωριασμενο μπηκε απο το παραθυρο απεναντι μου και ανοιξα τα ματια. Κοιταξα εξω. Βρεγμενοι, ερημοι δρομοι και ο ουρανος ειχε αυτο το βαρυ γκριζο της βροχης που εχει ξεχασει ποτε αρχισε και ποτε θα τελειωσει. Τιποτα δεν θυμιζε καλοκαιρι εδω.

Διπλα μου ομως ειχα την πιο ομορφη παρεα... Ο Φιοκο, ο υπεροχος καταλευκος σκυλαρος του Κωστα, με κοιτουσε αγουροξυπνημενος ολο απορια: ``Ποιος εισαι εσυ παλι;''

\photo{125.jpg}

Το σπιτι ηταν πολυ ομορφο. Αν το σπιτι της Christie ηταν η αποθεωση της διακοσμησης και των χρωματων, το σπιτι των παιδιων ηταν η χαρα της απλοτητας, της ζεστασιας, της χυμα ελευθεριας και του cool. 
Επιπλα παλια καθε ειδους και στυλ παντου, ενας μεγαλος ανετος καναπες, βιβλια και παλια περιοδικα μηχανης στα ραφια, ανταλλακτικα, βαλιτσες και αξεσουαρ της μοτοσυκλετας, computer, καλωδια, το μεγαλο στερεοφωνικο για τις μουσικες, το ηλεκτρικο μπασο και ο ενισχυτης στη γωνια, πολυχρωμοι πινακες στους τοιχους... 
Ο χωρος απεπνεε μια φοβερη ελευθερια, ενα cool ``χυμα'', σαν να εβγαζε με αυθαδεια τη γλωσσα στο δηθεν, στα επωνυμα design φροντισμενων, αγρατζουνιστων ιλουστρασιον καταλογων χωρις ουτε μια ατελεια, ετη φωτος απο τα προκατ ΙΚΕΑ αυτου του κοσμου. Εδω ζουσαν και ανεπνεαν ροκ ανθρωποι φιλε μου!

Ξεχασα το κωλοκαιρο εξω και χαμογελουσα σαν χαζος! Ποσο χαιρομουν που ημουν εδω. Τι υπεροχο συναισθημα να ξυπνας σε ενα τοσο ζεστο περιβαλλον και κοντα σε τοσο καλους φιλους οσο ηταν ο Κωστας και η Σαρα! 

Κατα φωνη. Tα παιδια ειχαν μολις ξυπνησει και με το ζεστο πρωινο καφεδακι στα χερια πιασαμε τη κουβεντα για τη ζωη εδω πανω... 
Βλεποντας τωρα τα πραγματα απο κοντα και παιρνοντας μια ελαχιστη γευση απο τη καθημερινοτητα εδω, καταλαβα οτι οι απλοι ανθρωποι ζουσαν στις ιδιες αγωνιες που ζουμε και εμεις στην Ελλαδα. Οι στημενες ειδησεις στο χαζοκουτι δειχνουν μονο τις εικονες που θελουν να προβαλλουν: ο Γερμανος ο πλουσιος κυριλε θειος, ο Ελληνας ο φτωχος συγγενης. 

Τελικα η ζωη στην Γερμανια δεν ηταν οπως την ειχα πριν στο μυαλο μου. 
Μειωσεις μισθων, ακριβεια, αυξανομενη ανεργια, ενα αυριο με ενα τεραστιο ερωτηματικο να κρεμεται απο πανω και μελλον χωρις εξασφαλιση... Μοιαζει γνωριμο το σκηνικο; 
Και ολα αυτα σε μια χωρα που δεν ειναι η δικη σου -διπλο το βαρος. Η επιστροφη ειναι παντα μια φλογα που καιει μεσα σου και σε τρωει -το ηξερα καλα. Ο Κωστας μου ειπε οτι σκεφτηκαν πολυ να γυρισουν πισω ομως αυτα τοτε. Πριν τη κριση. Οχι τωρα. 
Τωρα υπομονη, ανασυγκροτηση, αναμονη για κατι καλυτερο. Οπως ολοι μας φιλε -κι εκει και εδω. 
Εξαλλου αυτα τα παιδια δεν ειχαν να φοβουνται τιποτα -τα πιο σημαντικα συστατικα ηταν εκει: υγεια, αγαπη, φροντιδα, ελπιδα, παθος. Οσο εχεις αυτα δεν εχεις να φοβασαι τιποτα και κανεναν...

Η κουβεντα συντομα γυρισε σε πιο ευχαριστα θεματα καθως αρχισαμε να μιλαμε για το ταξιδι μου. 
Ο Κωστας ηταν παλιος γνωριμος στα Νορβηγικα λημερια, εχοντας κανει αρκετα ταξιδια στη χωρα των Βικινγκ στο παρελθον, οποτε οι συμβουλες του ηταν πολυτιμες! Με τους χαρτες ανοιχτους λοιπον αρχισαμε να χαραζουμε διαδρομες καθως μου προτεινε ποια μερη αξιζει να δω και ποια να αποφυγω -αν και σε μια τοσο εντυπωσιακη χωρα δεν νομιζω οτι θα μπορουσε να υπαρχει μερος βαρετο η αναξιο αναφορας...

Ο Κωστας μιλουσε με τοσο ενθουσιασμο για οσα ειχα να δω μπροστα μου: ``Α ρε Νικολα, να μην ειχα τη δουλεια στο μαγαζι και θα ερχομουν μαζι σου ρε φιλε.''
Αυτο θα ηταν πραγματικα οτι καλυτερο αλλα η λογικη υπαγορευε οτι δεν μπορουσε να γινει. Αυτη τη φορα το ταξιδι θα γινοταν σολο. 
Καθως κοιτουσα τις γραμμες στο χαρτη αρχισα να νοιωθω πιεση για αυτο που με περιμενε μπροστα μου. Τι παω να κανω....; Θα το καταφερνα ενα τοσο μακρυνο ταξιδι μονος μου; 

Απομακρυνα αυτες τις σκεψεις απο το μυαλο μου και επικεντρωθηκα στη σημερινη διαδρομη. Το προγραμμα ηταν απλο: θα καναμε μια αλλαγη ελαστικων στην Αυρα (καθως αυτα που ειχα επανω ειχαν ηδη 15.000 και δεν θα αντεχαν αλλα 11.000) και μετα αναχωρηση για το Aarhus της Δανιας περιπου στα μισα της χωρας. 
740 χιλιομετρα. 
Πολλα. 
Ομως το καραβι για Νορβηγια εφευγε αυριο το πρωι απο το Hirtshals στο βορειο ακρο της Δανιας και αν ηθελα να ειμαι εκει επρεπε σημερα να φαω οσο μεγαλυτερο μερος της αποστασης γινοταν.

\photo{126.jpg}

Το ηξερα οτι αυτη η μερα οτι θα ηταν ζορικη. Ομως αν ηξερα τι θα με περιμενε θα κλειδωνομουν στη κοντινοτερη μπυραρια και δεν θα εβγαινα απο εκει μεχρι τη Δευτερα Παρουσια... 

Ο καφες ειχε τελειωσει απο ωρα και ηταν καιρος να ξεκινησουμε τη μερα.
Τα λαστιχα της μηχανης τα ειχα παραγγειλει ηδη προ ημερων σε Γερμανικο eshop και ηταν ηδη στο σπιτι του Κωστα. Πηραμε τη μηχανη και πηγαμε στο μικρο γκαραζακι που φιλοξενουσε τη δικια του κουκλα και ενα πανεμορφο FJR του πεθερου του και ξεκινησαμε να ....βγαλουμε τις ροδες; 

- Ρε Κωστη δεν χρειαζοταν να μπεις σε τετοια φασαρια ρε φιλε!"
- Δεν ειναι φασαρια ρε συ Νικο, να! {τσακ, τσακ} Οριστε! Βγηκαν οι ροδες!

\photo{127.jpg}

Μεσα σε ελαχιστο χρονο και με το πιο μεγαλο χαμογελο ο Κωστας ειχε λυσει εξατμιση, δαγκανες, αξονες και ειχαμε τους τροχους ανα χειρας. Τα φορτωσαμε ολα μεσα στο αυτοκινητο του και τα αφησαμε σε ενα γνωστο του λαστιχα. Aν και βλοσσυρος και παρα το κουσουρι του να ασχολειται κυριως με αγροτικα μηχανηματα (aka BMW moto) εδειχνε να ξερει τι κανει: ``Σε μερικες ωρες θα ειναι ετοιμα, ελατε να τα παρετε.''

Στο μεταξυ ομως εμεις δεν θα καθομασταν με σταυρωμενα χερια. 
Ειχαμε μια ακομα συναντηση που περιμενα απο καιρο: θα πηγαιναμε στο κοντινο Solingen να γνωρισω απο κοντα το μουρλοκομειο εκ Θεσσαλονικης Γιαννη Παναγιωτιδη, παγκοσμιως γνωστο Μπεμβεδοφαγο με αδυναμια στις πιτσες, στις στροφες και μια περιεργη ταση να περναει απο πανω οτι supersport μηχανη πεσει στο δρομο του! 
Ο Γιαννης ηταν μεγαλη μορφη! Μας περιμενε στο μαγαζι του με μεγαλη χαρα και με εκανε να νοιωσω σαν στο σπιτι μου! Απιστευτο χιουμορ, τρελαρας, πολυ large - σαν τον Elvis ηταν!  

Το μικρο μαγαζι του ηταν πολυ ομορφο και τακτικο, με προσεγμενο φαγητο και μερακι. Πολλα χρονια εκει, δουλευε σκληρα μαζι με την γλυκυτατη συζυγο του αλλα οι κοποι τους ειχαν αποδοσει καρπους. Ειχανε φτιαξει μια πολυ ομορφη οικογενεια, ειχαν μια δικια τους στεγη, μια δικια τους δουλεια. 

\photo{128.jpg}

Ελληνες της διασπορας. Δυσκολιες αλλα και ονειρα, στο internet radio μονιμα Ελληνικος σταθμος, στο παγκο το φορητο με φωτογραφιες απο νησια της Ελλαδας: ο νοστος για τη πατριδα, η πικρα και το παθος... Ολα εκει...

Ενοιωθα τον πονο τους που ηταν μακρυα απο οσα θεωρουσαν οικεια και ας ελειπαν τοσα χρονια, και ας ζουσαν σε πολυ καλυτερες συνθηκες απο οτι εμεις εδω σημερα. 
Ο Κωστας γυρισε και μου ειπε με παραπονο: ``Εδω δεν ειναι η χωρα της επαγγελιας φιλε. Η Γερμανια ειναι φυλακη δουλειας.''
Τον καταλαβαινα αλλα και ο Ελληνας που βρισκεται στη χωρα του χωρις δουλεια ειναι σε εξισου ασχημη ``φυλακη'' ομως... 
Για αλλη μια φορα σκεφτηκα το πως καταληξαμε να πληρωνουμε σπασμενα προηγουμενων γενεων και ενος αδηφαγου συστηματος ακρατης καταναλωσης πορων, χρηματων, ανθρωπων, αξιων... Σαν να ειχε γινει το μεγαλυτερο παρτυ ολων των εποχων και εμεις που δεν μας καλεσαν να επρεπε να πληρωσουμε τωρα το λογαριασμο.

Δεν ηταν ζωη αυτο που μας ειχαν σπρωξει να ζουμε τωρα -κλεισμενοι στα κουτια μας, φοβισμενοι, χωρις θεληση και οικονομικα μεσα να ξεφυγουμε και να παμε λιγο παραπερα. Ο ενας εδω ο αλλος εκει και ολοι καθηλωμενοι. Ημασταν οι μπαταριες του ματριξ και αυτο εμενα τουλαχιστον μου καθοταν πολυ στραβα! Αν θα μπορουσα να παρακινησω εστω και εναν ανθρωπο να ανοιξει τα ματια και να παρει ενα ρισκο για κατι καλυτερο θα αξιζε τον κοπο.
Τουλαχιστον αυτα τα παιδια εδω το ρισκο τους το ειχαν παρει και οποιο και να ηταν το αποτελεσμα, σημασια εχει παντα η προσπαθεια.

Χαμενος οπως παντα στο κοσμο μου, ο Γιαννης με συνεφερε με χαμογελο. 
- Εχεις φαει πιτσα γυρο;
- Πιτσα ..τι;
- Καλα ρε, πλακα με κανεις, δεν εχεις φαει πιτσα με γυρο; Κατσε να δεις. λεει και φερνει 2 μεγαλες λαχταριστες πιτσες με ...γυρο επανω! Μουρλια! Ο Γιαννης με ειχε κατασκλαβωσει με την ευγενεια και την περιποιηση του!

Η ωρα περνουσε και ηταν καιρος να πηγαινουμε, οχι ομως πριν τις απαραιτητες αναμνηστικες φωτογραφιες. Γεια σας ρε παιδια! Να ειστε παντα καλα, χαμογελαστοι και να εχετε οτι καλυτερο στο διαβα σας... Τετοιοι φιλοι ειναι ο αληθινος θησαυρος της ζωης!

\photo{129.jpg}

Στο μαγαζι ο Γερμανος ειχε αλλαξει τα λαστιχα οχι ομως αναιμακτα: στο μεσα μερος της πισω ζαντας ειχε σπασει ενα μικρο κομματακι χρωματος μεχρι το μεταλλο, ισως χτυποντας το με καποιο εργαλειο οταν πηγε να βαλει τα βαρακια της ζυγοσταθμισης. Και να σκεφτει κανεις οτι η ζαντα ειχε γινει βαφη πριν ενα χρονο! Χαλαστηκα λιγο αλλα δεν εδωσα συνεχεια -μικρο το κακο εξαλλου.

Εξω η βροχη δεν ελεγε να κοψει. ``Ριχνει ετσι εναμισι μηνα τωρα. Ειναι απο τα πιο βροχερα καλοκαιρια της χωρας εδω και χρονια.''
Γ@μω τη τυχη μου. Κανω το πιο μεγαλο ταξιδι μου τη χρονια που ο Θεος αποφασισε να ξεπλυνει τις αμαρτιες των ανθρωπων. Κυριολεκτικα.

Ο Γιαννης στο μαγαζι με ειχε προσγειωσει πολυ αποτομα σε μια ζορικη πραγματικοτητα:

\dialogue{Ποσα χιλιομετρα εχεις κανει μεχρι εδω φιλε;}
\dialogue{Ε περιπου 2 με 2μισι χιλιαδες.}
\dialogue{Και ποσα εχεις συνολο; 12.000; Πωπωω ρε φιλε, δηλαδη εχεις αλλα 10.000 χιλιομετρα μπροστα σου;;; Χαρα στο κουραγιο σου!}
\dialogue{........}

Σκατα. Ενοιωσα την πραγματικοτητα του ταξιδιου που υψωνοταν τωρα μπροστα μου θεορατο και λιγοψυχησα. Δεν ηθελα να συνεχισω. Η σκεψη να εμενα εκει για μερικες μερες και τελικα να γυριζα πισω δεν με αφηνε σε ησυχια. Ευτυχως ο Κωστας ηταν εκει να με στηριξει με το καλοσυνατο και χαμογελαστο τροπο του: ``Νικολα, μην σκας φιλε! Κανεις ενα υπεροχο ταξιδι. Πηγαινε και ζηστο. Ολα θα πανε καλα.''

Η ωρα ειχε παει 5 το απογευμα και ο Κωστας επρεπε να παει στη δουλεια. Πηρα τηλεφωνο στο hostel που ειχα κλεισει στο Aarhus οτι θα αργουσα: ``Κανενα προβλημα. Θα αφησουμε τη καρτα για το δωματιο σε μια θυριδα στην εισοδο. Πατηστε τον ταδε κωδικο για να το παρετε.''

Ξεκινησα να μαζευω τα πραγματα δεν ενοιωθα ομως ομορφα. Ειχα καθυστερησει ηδη πολυ να ξεκινησω και ειχα 740 ολοκληρα χιλιομετρα μπροστα μου με αυτο το κωλοκαιρο; 
Εχοντας αποφασισει οτι μου εφτανε η πιεση του ταξιδιου και μιας που δεν ηθελα να εχω και το ζορι να σφηνωνω καθε μερα τα πραγματα στις βαλιτσες, δανειστηκα ενα τεραστιο αδιαβροχο σακο απο τον Κωστα για να βαζω μεσα ευκολα τη σκηνη, τον υπνοσακο και λοιπα πραγματα πρωτης αναγκης. 
Για αλλη μια φορα με σκλαβωνε αυτο το παλικαρι! Ομως ενοιωθα ασχημα να του παρω το σακκο στην Ελλαδα παροτι θα τον εστελνα πισω με ταχυδρομειο.

Αποχαιρετησα τον Κωστα και τη Σαρα και καβαλησα, αλλα αυτη τη φορα κανοντας μεγαλη προσπαθεια να πεισω τον εαυτο μου να φυγει... To ταξιδι ηταν μπροστα μου αλλα για αλλη μια φορα επρεπε να αφησω πισω αγαπημενα ατομα...

\photo{130.jpg}

Βγηκα στην εθνικη οδο αλλα κατι με ετρωγε. Η βροχη ειχε σταματησει τωρα και ενας απογευματινος ηλιος εφτιαχνε το ιδανικο σκηνικο για να ταξιδεψω, ομως δεν ενοιωθα καλα. Ηταν και αυτος ο τεραστιος σακκος πισω μου...

Και τοτε μου ηρθε η ιδεα! Το πρωι ειχα δει ενα φυλλαδιο απο ενα μεγαλο μαγαζι με ειδη μοτο και ειχε ενα πολυ ωραιο μακροστενο αδιαβροχο σακκιδιο με ανοιγμα απο πανω και το ειχα βαλει στο ματι αλλα λογω χρονου με τον Κωστα δεν ειχαμε παει να το παρουμε.
Ε θα πηγαινα να το παρω! Θα εχανα τουλαχιστον αλλη μια ωρα και ηδη ειχα αργησει τραγικα να ξεκινησω! Αυτο θα ηταν εντελως τρελο! Ω ναι!

Εκανα δεξια στο πρωτο βενζιναδικο που βρηκα για να βρω τη διευθυνση του μαγαζιου. Εψαχνα στα μενου του GPS οταν ακουσα διπλα μου ενα μεγαλο τετρακυλιδρο να πλησιαζει. Γυρισα και ειδα ενα Suzuki RF 900 τιγκα φορτωμενο με ενα νεαρο Γερμανο, τον Ion. Μου επιασε κουβεντα για το απο που ειμαι και που παω. 
Αυτος γυριζε απο tour 2 βδομαδων μονος σε Ισπανια - Γαλλια. Ωραιος...
Του ειπα για το ταξιδι μου μεχρι εκει, τις αναποδιες με το πεσιμο στο πλοιο, το ασχημα πρησμενο δαχτυλο στο δεξι χερι που δεν ελεγε να περασει, τη ζημια στη ζαντα, την ατελειωτη βροχη και τα ασχημα προγνωστικα...

Ο Ion γυρισε και μου εδειξε προς την μηχανη του: ενας απο τους ιμαντες που εδεναν τα πραγματα περνουσε μεσα απο τα πλαστικα της ουρας που ηταν κομμενα καθετα περα ως περα!
Μου ειπε: "Το βλεπεις αυτο; Αυτο εγινε τη πρωτη μερα του ταξιδιου μου. Ο ιμαντας πιαστηκε στη πισω ροδα και εκοψε τελειως την ουρα της μηχανης στα δυο. Με χαλασε παρα πολυ αλλα μετα σκεφτηκα οτι αφου δεν σκοτωθηκα ολα τα αλλα περισσευαν. Ετσι πιεσα τον εαυτο μου και πηγα στο ταξιδι και μαλιστα περασα υπεροχα! \textbf{``So remember. Keep riding and the sun will shine again!''}

Ειχε δικιο! Μπροστα μου ειχα μια μεγαλη περιπετεια και θα την χαιρομουν ΟΤΙ και να συνεβαινε! Αποχαιρετησα το νεο μου φιλο και πηγα στο μαγαζι για το σακκο. Το κοστος του ηταν ελαχιστο και σιγουρα λιγοτερο απο το ταχυδρομειο του αλλου σακκου απο Ελλαδα στη Γερμανια. 
Λιγη ωρα μετα βρισκομουν παλι στο μαγαζι του φιλου. Ο Γιαννης γουρλωσε με εκπληξη τα ματια του σαν να εβλεπε εξωγηινο. 

\dialogue{Επ! Νικο;; Δεν εφυγες ακομα ρε φιλε;}
\dialogue{Οχι, αργησαμε λιγο με τα λαστιχα και το μαζεμα, συν οτι αποφασισα να παρω και ενα σακκο για να μην παρω του Κωστα μαζι, οποτε ηρθα να στον αφησω να του τον δωσεις.}
\dialogue{Και θα φυγεις; Τωρα;}
\dialogue{Ναι, πρεπει να ειμαι αυριο στο Hirtshals στη Δανια.}\\

\noindent Με κοιταξε σαν εξωγηινο.\\

\dialogue{Εισαι τελειως τρελος ρε αδερφε! Πας να κανεις τετοια αποσταση οπως κανουμε εμεις μια βολτιτσα παραδιπλα;!; Πως αντεχεις; Ελα παμε λιγο εξω να χαζεψω το εργαλειο.}\\

Ο Γιαννης εκατσε πανω στην Αυρα ποζαροντας με καμαρι και η συζυγος πανταχου παρουσα εβγαλε μερικες ακομα αναμνηστικες φωτογραφιες προτου αναχωρησω και παλι, αυτη τη φορα ομως τα πραγματα ηταν πολυ δυσκολα. Η οθονη εγραφε 748 χλμ και η ωρα ηταν πλεον 7.30 το βραδυ! Αποψε θα χρειαζοταν να ξεπερασω ολα μου τα ορια... Το να χασω το πλοιο για Νορβηγια την επομενη μερα απλα δεν ηταν επιλογη. 
Στα ακουστικα εβαλα το \href{http://www.youtube.com/watch?v=g5vYEAUIcLM}{Highway of light} των Madrugada τερμα...

Αποψε θα ημουν στη Δανια η πουθενα...

\photo{131.jpg}

Αν ολα αυτα δεν ηταν αρκετα, στην autobahn με περιμενε ενα νεο προβλημα: εργα. Πολλα εργα. Οι τρεις λωριδες της εθνικης γινονταν μια και κατι και τα αυτοκινητα εκαναν ουρες χιλιομετρων! Στριμωχνα την μηχανη στο περισσευμα της λωριδας που ηταν διαθεσιμη, αναμεσα στο στηθαιο του δρομου και τα σχεδον σταματημενα αυτοκινητα και παρακαλουσα τους θεους του ταξιδιου να μην ανοιγε καποιος τη πορτα του... Οι ταχυτητες ηταν αναγκαστικα μικρες τουλαχιστον μεχρι να βγουμε απο τα εργα και να ανοιξουμε παλι γκαζι μεχρι το επομενο σημειο μποτιλιαρισματος.

Περασαν πανω απο 350 δυσκολα χιλιομετρα ετσι. ``Κανενα προβλημα, εχεις αλλα 400 να τα απολαυσεις μεγαλε'' σκεφτηκα και γελαγα με τον εαυτο μου!
Οταν οι διαδρομες ειναι τοσο μεγαλες και εισαι ηδη απο ωρα στο δρομο τα χιλιομετρα φευγουν βασανιστικα...
Η κουραση ειχε κανει εντονη τη παρουσια της απο ωρα και οι δυναμεις μου με εγκατελλειπαν. Κρατουσα σταθερη την Αυρα στην ευθεια του δρομου και προσπαθουσα με καθε τροπο να κραταω τον εαυτο μου σε ενγρηγορση ενω το μυαλο ειχε κατεβασει τις ασφαλειες προ πολλου.

Τα μεσανυχτα με βρηκαν εξω απο το λιμανι του Αμβουργου, παγωμενο, εξαντλημενο, σε κατασταση οριακη. Το κρυο της νυχτας ηταν πλεον αφορητο. Πονουσα παντου, χερια, ποδια, αυχενας, κρυωνα παρα τα αδιαβροχα και το μονο που ηθελα να κανω ηταν να κοιμηθω. Ομως δεν γινοταν ακομα -ειχα 300 χιλιομετρα μπροστα μου και επρεπε να βγουν.

Οι σκεψεις μου παραξενες αυτη την ωρα, μπλεγμενες σε μια ομιχλη σαν αυτη που τυλιγε το εφιαλτικο, παγωμενο βιομηχανικο σκηνικο γυρω μου.

Οι τεραστιοι γερανοι του λιμανιου μου εφεραν στο μυαλο την διασημη εικονα των \href{http://upload.wikimedia.org/wikipedia/en/0/03/Pinkfloydhammers.jpg}{παρελαυνοντων σφυριων των Pink Floyd} και στο mp3 ξεκινησε το \href{http://www.youtube.com/watch?v=jySUpMqmzd4}{Comfortably Numb} -Tοσο ταιριαστο υπο τις συνθηκες...

\begin{verse}
Hello,
Is there anybody in there?
Just nod if you can hear me
Is there anyone home?
\end{verse}

Η συνεχεια ηταν θαμπη και συγκεκχυμενη μεσα στο μυαλο μου -εικονες φευγαλεες περνουσαν και χανονταν μεσα σε μια ομιχλη ζαλης, κουρασης και υπνηλιας. Η εξοδος απο την Autobahn... Τα συνορα με Δανια που ολο και περιμενα και δεν ερχονταν ποτε... Ενας παραξενος σκουρος μπλε ουρανος... Μα τετοια ωρα; Ποτε νυχτωνει εντελως; 

Πλησιαζοντας τα συνορα Γερμανιας-Δανιας, 10.42μμ...

\photo{134.jpg}

Ατελειωτες ευθειες χιλιομετρων αναμεσα σε δαση... Ανεφοδιασμοι για βενζινη σε πρατηρια ...καπου... Σκοταδι πια και ταξιδι στο πουθενα...

\photo{135.jpg}

Η θερμοκρασια στο οργανο φλερταρε πλεον με μονοψηφια νουμερα αλλα τωρα καλοδεχομουν τον παγωμενο αερα γιατι με κρατουσε ξυπνιο. Καταφερα να κραταω ενα σταθερο ρυθμο και η Αυρα καταπιε τα τελευταια χιλιομετρα γρηγορα... 
Το Aarhus ηταν μπροστα μου -τα ειχα καταφερει! Δρομοι ερημοι, λουσμενοι σε ενα κιτρινο φως απο τις νυχτερινες λαμπες, φαναρια αναβοσβηναν πορτοκαλι περιμενοντας το ξημερωμα για να ζωντανεψει παλι η πολη... 
Το GPS με οδηγησε γρηγορα στο hostel στο κεντρο. 
Παρκαρα τη μηχανη στο πεζοδρομο κοντα στα τελευταια κλαμπακια που ακομα ειχαν ζωη και ανεβηκα στο πολυ μοντερνο και προσεγμενο ξενωνα. 
Επεσα στο κρεββατι με ενα μεγαλο χαμογελο. Το ρολοι εγραφε 2.30 το πρωι. 
Μπορει να ειχα να σηκωθω σε 4 ωρες αλλα δεν με ενοιαζε γιατι σε λιγο ξημερωνε η μεγαλη μερα: η Νορβηγια με περιμενε μυστηριωδης, αγνωστη! 
Το Ονειρο ξεκινουσε... Και ουτε στα πιο τρελα μου ονειρα θα μπορουσα να ειχα φανταστει τι θα εβλεπα στη συνεχεια...
