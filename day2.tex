\chapter{Day 2 -- Como (IT) - Guebwiller (FR) 430 km}

Το ξημερωμα με βρηκε πτωμα. Ο αθλιος υπνος που ειχα κανει μεσα στο Superfast και τα 500+ χιλιομετρα μεσα στον πρωτοφανη για την Ιταλια καυσωνα ειχαν κανει αισθητη την επιδραση τους σε ενα σωμα που ειχε μεχρι προτινος καλομαθει σε ολες τις καθημερινες ανεσεις. Η ωρα ηταν περασμενες 8 και παροτι ηθελα να συνεχισω να κοιμαμαι ηξερα οτι επρεπε να σηκωθω αν ηθελα να βγαλω το προγραμμα της ημερας οπως ειχα σχεδιασει. Εξαλλου η πρωινη ζωη του camping που ειχε ξεκινησει εδω και ωρα δεν θα με αφηνε να κοιμαμαι για πολυ ακομα και να ηθελα...

\photo{19.jpg}

Ο προορισμος μου για σημερα ηταν η Γαλλια και συγκεκριμενα το Guebwiller στη νοτιοανατολικη Αλσατια οπου θα συναντουσα τη Christie: ενα μελος του Couchsurfing που μου ειχε προσφερει με χαρα να με φιλοξενησει στο σπιτι της αφου θα περνουσα απο την περιοχη. Γιατι ομως εκει; Διοτι σε αποσταση λιγων χιλιομετρων ηταν το περιφημο \href{http://en.wikipedia.org/wiki/Colmar}{Colmar}: καποιοι το ανεφεραν ως τη πιο ομορφη πολη της Ευρωπης και δεν θα αφηνα να παει χαμενη η ευκαιρια να διαπιστωσω αν ηταν αυτο ηταν αληθεια...

Πριν απο αυτο ομως σημερα θα επαιρνα την εκδικηση μου για τα βαρετα χιλιομετρα εθνικης που ειχα γραψει χθες! Ειχα να διασχισω την Ελβετια και σκοπευα να κινηθω οσο μπορουσα σε επαρχιακους δρομους και να απολαυσω τη μεγαλειωση φυση της χωρας αλλα βεβαια το μεγαλυτερο κερασακι ηταν αλλο: τα διασημα πασα Furkapass και Grimselpass στα 2.400 μετρα υψομετρο στην ραχη των Ελβετικων Αλπεων.
Οταν εχεις ενα τετοιο προγραμμα ξυπνας πιο ευχαριστα οσο να ναι...

\photo{20.jpg}

Πρωτα απο ολα ομως καφες. Οι Ουαλλοι διπλα ειχαν σηκωθει απο ωρα και επιναν ηδη το τσαι τους χαμογελαστοι, ενω δεν εχασαν ευκαιρια να μου την πουν οτι παρακοιμηθηκα κιολας! Οι παλιολυκοι μου εβαλαν τα γυαλια!
Ζεστανα νερο στο κουζινακι και εφτιαξα τον καφε μου και εκατσα να τον απολαυσω πλαι στους χαρτες μου και με παρεα αλλους μηχανοβιους ταξιδιωτες.
Το εχω πει πολλες φορες και θα το πω αλλες τοσες: δεν υπαρχει καλυτερος τροπος απο το να ξεκινας ετσι τη μερα σου!

\photo{21.jpg}

Οι Ουαλλοι μου ελεγαν οτι θα ακολουθησουν πανω κατω την ιδια διαδρομη με εμενα, τουλαχιστον μεχρι τα πρωτα πασα. Εκανα τη σκεψη να παμε μεχρι εκει παρεα ομως τους ειδα να βιαζονται να ξεκινησουν: εγω δεν ειχα προλαβει να πιω ακομα τον καφε μου και αυτοι ειχαν ξεστησει τις σκηνες τους και ειχαν σχεδον πακεταρει τα παντα! 
Αρχισα να μαζευω τα πραγματα βιαστικα αλλα καταλαβα οτι απλα θα τους καθυστερουσα χωρις λογο και εγω θα αγχωνομουν αδικα. Ανταλλαξαμε συμβουλες για τη διαδρομη και χαιρετηθηκαμε -μπορει να τα λεγαμε και στη πορεια εξαλλου!

Το πακεταρισμα παντα ηταν για εμενα ενα θεμα: μου τρωει χρονο. Πολυ χρονο. Μεχρι να μαζεψω υπνοσακους, υποστρωματα, σκηνη, τεντες, ρουχα και λοιπα συμπραγκαλα οι περισσοτεροι γειτονες ειχαν φυγει ηδη! Οκ, βεβαια πολλοι ηταν με τροχοσπιτα και αυτοκινητα αλλα οπως και να το κανεις ειναι αποκαρδιωτικο να παλευεις να σφηνωσεις ολα τα μπαγκαζια στις βαλιτσες και να βλεπεις τους αλλους να ξεστηνουν τα παντα χαμογελαστοι στο χρονο που σου παιρνει εσενα να κλεισεις την μια βαλιτσα! 
Βεβαια ειχα φερει μαζι μου και κατι παραπανω οσο να ναι... (Μιας που για να φας στη Νορβηγια πρεπει να παρεις ενα μικρο στεγαστικο δανειο ειχα προετοιμαστει καταλληλα)

\photo{22.jpg}

Βγηκα στους δρομους του Como και ο ηλιος εδειχνε οτι θα ειναι μια υπεροχη -και πολυ ζεστη- μερα. 
Απο ψηλα η λιμνη και τα σπιτια αμφιθεατρικα στο βουνο εφτιαχναν ενα πολυ ομορφο θεαμα -για αλλη μια φορα εδωσα υποσχεση στον εαυτο μου να ξαναρθω σε αυτα τα μερη για μια δευτερη αναγνωση...

\photo{23.jpg}

Ενω περνουσα μεσα στους δρομους του χωριου, ξαφνικα.... συνορα;!; Τα συνορα με την Ελβετια ειναι μεσα στη πολη! 
Δεν το ειχα ξαναδει αυτο ποτε. Οκ, η προσβαση ειναι ελευθερη στους Ευρωπαιους πολιτες, αλλα οπως και να το κανεις, το να βλεπεις συνοριακο φυλακιο, αστυνομικους στη μεση ενος δρομου και σημαιες μιας αλλης χωρας ειναι κατι εντελως ασυνηθιστο. 
Απο την μια μερα της πολης Ιταλια και απο την αλλη Ελβετια!

\photo{24.jpg}

Αν δειτε το χαρτη της περιοχης θα παρατηρησετε οτι τα συνορα κοβουνε το χωριο στη μεση ουσιαστικα. Απο τη μερια της Ιταλιας ειναι το Como και απο τη μερια της Ελβετιας το Chiasso. Βεβαια η γλωσσα παρεμενε η Ιταλικη, τα μαγαζια συνεχιζαν να παιρνουν ευρω αλλα καταλαβαινες οτι κατι ανεπαισθητο εχει αλλαξει... Πολυ παραξενο συναισθημα! 

Με κατι τετοια κτηρια ομως θα μπορουσα να πιστεψω οχι μονο οτι ειμαι σε αλλη χωρα αλλα και σε αλλο ...πλανητη! Αυτος ο ιπταμενος δισκος τεραστιων διαστασεων ηταν το εμπορικο κεντρο της περιοχης.

\photo{25.jpg}

Ομως εγω τωρα ηθελα να χαθω σε επαρχιακα δρομακια... 
Να δω χωρια, να χαρω τη φυση, να πιασω μια ροτα στο περιπου και να βγω οπου με παει ο δρομος.

\photo{26.jpg}

Ο δρομος αρχισε να ανηφοριζει στο βουνο με ποιοτητα ασφαλτου που θα ζηλευε και πιστα....

\photo{27.jpg}

...χωριουδακια τριγυρω σαν ψευτικα....

\photo{28.jpg}

...δαση και ατελειωτο πρασινο οπου και να γυριζε το ματι....

\photo{29.jpg}

....και μια θεα που σε εκανε απλα να στεκεσαι και να κοιτας σαν χαμενος!

\photo{30.jpg}

Ο δρομος συντομα με εβγαλε στις οχθες της λιμνης Lugano και της ομωνυμης πολης που βρισκεται εκει...

\photo{31.jpg}

Η αρχιτεκτονικη ηταν εμφανως Ιταλικη, κατι αναμενομενο αν σκεφτει κανεις οτι ολη αυτη η περιοχη ανηκε επι αιωνες στους Ιταλους. Ημουν σιγουρος οτι δεν θα ηταν και πολυ χαρουμενοι που τωρα ηταν μερος της Ελβετιας, αλλα ακομα και ετσι ηταν ουσιαστικα ιταλικη περιοχη. 
Κοιταζοντας αυτες τις διαχωριστικες γραμμες στο χαρτη να πασχιζουν να ορισουν περιοχες με εκανε να σκεφτομαι ξανα τη ματαιοδοξια και γελοιοτητα του να προσπαθεις να χωρισεις γη και πολιτισμο σε ταμπελες: ``Ιταλια'', ``Ελβετια'', εμεις, εσεις. Μονοι εμεις οι ανθρωποι καταφερνουμε κατι τοσο αδιαιρετο οσο η Γη να το κοβουμε σε κομματια και να σκοτωνομαστε κιολας για αυτα...

\photo{32.jpg}

Ολες ομως αυτες οι σκεψεις εξατμιστηκαν με μιας οταν η θεα της λιμνη ξεδιπλωθηκε μπροστα μου... 
Τι μαγευτικο θεαμα!

\photo{33.jpg}
\photo{34.jpg}

Το προγραμμα απο την αρχη σημερα ηταν να κανω οσο πιο πολλα χιλιομετρα μπορουσα σε επαρχιακους δρομους. 
Ομως τωρα ημουν λιγο εξω απο την Bellinzona και ειχα ηδη κανει μιαμιση ωρα διαδρομης για μολις 70 χιλιομετρα -τα χωριουδακια και οι φιδωτοι ορεινοι δρομοι μπορει να ηταν ακρως γραφικοι αλλα δεν βοηθουσαν να γραψω χιλιομετρα.

\photo{35.jpg}
\photo{36.jpg}

Κοιταξα το GPS. Αν ηθελα να ειμαι στη Γαλλια νωρις το βραδυ θα επρεπε να βαλω λιγο νερο στο κρασι μου και να πιασω την εθνικη, τουλαχιστον μεχρι τα πασα. Η Christie με περιμενε στο σπιτι της και θα ηταν μεγαλη αγενεια να φτασω μεσα στη μαυρη νυχτα και απλα να πεσω να κοιμηθω σε ενα καναπε για να αναχωρησω την επομενη μερα.

Ετσι μετα απο μια σταση για ανασυγκροτηση προμηθευτηκα την Ελβετικη βινιετα οσο και αν πονουσε -27 ευρω ειναι αυτα!- και βγηκα στην εθνικη...
Αν εκρινα ομως απο τη θεα που ανοιγοταν μπροστα μου κατι μου ελεγε οτι δεν θα βαριομουν καθολου!

\photo{37.jpg}

Ολα πηγαιναν περιφημα. Η Αυρα ειχε την ευκαιρια να ξεμουδιασει λιγο, τα χιλιομετρα εφευγαν γρηγορα και εγω χαζευα την καταπρασινη φυση γυρω μου με ενα χαμογελο μεχρι τα αυτια! Και ολα θα συνεχιζαν να πηγαινουν περιφημα αν δεν υπηρχε κατι που ακουγε στο ονομα Galleria stradale del San Gottardo. 

Καποια στιγμη στην εθνικη οδο ειδα στο βαθος κινηση. Το ειδος της κινησης που ξερεις οτι κατι συμβαινει. Τα αυτοκινητα ηταν ακινητοποιημενα σε ολες τις λωριδες και περιμεναν. Σταματησα στην ακρη δεξια και προσπαθησα να καταλαβω τι γινεται. Ατυχημα; 
Τι αλλο θα μπορουσε να μπλοκαρει ολες τις λωριδες σε μια τετοια εθνικη οδο; Η ζεστη του μεσημεριου ηταν ανελεητη και τα δερματινα δεν βοηθουσαν ιδιαιτερα τη κατασταση. Ευτυχως ομως συντομα καποιοι αλλοι μοτοσυκλετιστες αρχισαν να περνανε σιγα σιγα απο την ΛΕΑ με αλαρμ και αποφασισα να τους ακολουθησω. Φτιαχνοντας στιχακια στο μυαλο μου ("ΛΕΑ ΛΕΑ και η ζωη ειναι ωραια" κλπ) και μετα απο μερικα χιλιομετρα πορειας αναμεσα σε λεωφορεια και κολωνακια εφτασα στην κεφαλη της ουρας: τα αυτοκινητα περιμεναν σε φαναρι! 

Μπροστα μου ανοιγοταν η εισοδος ενος τουνελ και η κυκλοφορια γινοταν σε δοσεις. Αυτο το ειχα δει στο παρελθον στην Αυστρια οποτε δεν εδωσα ιδιαιτερη σημασια και περιμενα το πρασινο για να μπω στο τουνελ. Λαθος.
Το ποσο μεγαλο λαθος ηταν αυτο το ανακαλυψα καθως εμπαινα στο τουνελ και ειδα την πινακιδα: Galleria stradale del S. Gottardo, 16.400 m. Ποσα;; 16;;; Δεκαεξι χ ι λ ι ο μ ε τ ρ α;;; Καπου εκει θυμηθηκα ενα βραδυ που κοιταζοντας τους χαρτες στο σπιτι ειχα δει μια τεραστια ευθεια γραμμη καπου στις Ελβετικες Αλπεις και σκεφτηκα οτι δεν μπορουσε να ειναι δρομος αυτο το πραγμα. Εντελως ευθεια και τοσο μεγαλο σε μηκος... Δεν μπορει. 
Και ομως. Αυτη τη στιγμη βρισκομουν στο τουνελ του St. Gottard, το τριτο μεγαλυτερο τουνελ στον κοσμο με μηκος πανω απο 16 χιλιομετρα απο ακρη σ' ακρη!

\photo{38.jpg}

Ζωντας εδω στην Ελλαδα που τα μεγαλυτερα τουνελ σπανια ξεπερνανε τα 1-2 χιλιομετρα δεν ειναι ευκολο να καταλαβει κανεις τι σημαινει να διασχιζεις ενα τοσο τερατωδες τουνελ. 
Τα χιλιομετρα αρχισαν να περνανε το ενα μετα το αλλο αργα και βασανιστικα και συντομα καταλαβα οτι θα ειχα προβλημα. Εκλεισα τη ζελατινα καλα και ολους τους αεραγωγους σε μια απεγνωσμενη προσπαθεια να προφυλαχτω. Το θερμομετρο εδειχνε 52 βαθμους εξωτερικη θερμοκρασια και τα καυσαερια απο τα οχηματα εκαναν την ατμοσφαιρα να θυμιζει θαλαμο αεριων. Δραματικη κατασταση... Παρολαυτα ομως δεν μπορουσα να μην θαυμασω το τι εφτιαξαν οι ανθρωποι. Ακρως εντυπωσιακο αν και εξαιρετικα δυσκολο στη διασχιση για ενα μηχανοβιο... Μετα απο 20 λεπτα που φανηκαν σαν αιωνας βγηκα στην αλλη πλευρα! Ουφ! Βαθειες ανασσες και ο δροσερος πλεον αερας εμοιαζε τοσο γλυκος τωρα!

\photo{39.jpg}

Η πινακιδα που μολις ειχα περασει εγραφε τα μαγικα γραμματα: Furkapass δεξια. 
Χαμογελασα. Τα πασα με περιμεναν...

\photo{40.jpg}
\photo{41.jpg}

Το τοπιο πλεον ειχε αλλαξει δραματικα. Οι χαμηλες καταπρασινες πλαγιες της βορειας Ιταλιας ειχαν δωσει τη θεση τους σε τεραστιους βραχωδεις ογκους που ορθωνονταν γυρω μου σε ενα θεαμα πραγματικα μεγαλειωδες...
Κοιτες ποταμων περνουσαν αναμεσα σε πανυψηλες οροσειρες, ενω ο δρομος εμοιαζε με ενα γκριζο φιδακι που ξεδιπλωνοταν αναμεσα σε βραχους, δεντρα και νερο...

\photo{42.jpg}

Στο βαθος τα συννεφα χορευαν πανω στις καταπρασινες κορυφες, σε ενα αεναο παιχνιδι της γης με τον ουρανο, φτιαχνοντας μια μοναδικη εικονα απο αυτες που σε συνοδευουν για παντα και σου θυμιζουν οτι η ομορφια βρισκεται εκει εξω, στα πιο απλα, στα πιο μικρα... 
Και μεσα σε ολα αυτα, ο δρομος μπροστα που ανοιγοταν μεχρι τον οριζοντα... 

\photo{43.jpg}

Εδω ανοιγαν τα ματια, η ψυχη και το μυαλο και προσπαθουσαν να χωρεσουν μεσα τους το μεγαλειο αυτο... Τιποτε αλλο...
Και οπως εβλεπα δεν ημουν ο μονος που σκεφτοταν ολα αυτα τριγυρω... 

\photo{44.jpg}

O δρομος συντομα αρχισε να ανηφοριζει μεσα απο μικρους και γραφικους οικισμους, ολοενα και πιο στενος, αλλα τιποτα δεν μαρτυρουσε αυτο που βρισκοταν μπροστα...

\photo{45.jpg}

...ωσπου οι πρωτες στροφες του πιο διασημου Ελβετικου πασου ανοιχτηκαν μπροστα μου. 
``Stairway to heaven'', σκεφτηκα χαμογελοντας και το παρτυ ξεκινουσε!

\photo{46.jpg}

Αρχισα να ανεβαινω και η μια φουρκετα διαδεχοταν την αλλη σε ενα γιγαντιο rollercoaster, απανωτες ανηφορικες στροφες ξανα και ξανα και ξανα, με μια ασφαλτο που σε προκαλουσε να ξυσεις μεχρι και τα γκριπ! 
Και ολα αυτα ενω η θεα κατω πραγματικα εκοβε την ανασσα...

\photo{47.jpg}

Τρεχουμενα νερα παντου, ποταμια που κατεβαιναν απο τις πλαγιες του βουνου και μεγεθη που δεν θα χωρουσαν στον καλυτερο ευρυγωνιο φακο του κοσμου... Τι να σου κανει μια ταλαιπωρη φωτογραφικη οταν εχεις μπροστα σου τετοιες εικονες;

\photo{48.jpg}
\photo{49.jpg}

Ανεβαινοντας, απο ενα σημειο και μετα η μπαλα χαθηκε τελειως... Το μυαλο μηδενισε, κλειδωσε, πεταξε τα κλειδια στον αγυριστο και παρεδωσε στα ματια και την ψυχη. 
Αυτο που εβλεπα μπροστα μου απλα δεν μπορουσα να το περιγραψω με λογια... 
Εικονες βγαλμενες απο τα καλυτερα μου ονειρα...!

\photo{50.jpg}
\photo{51.jpg}

Το GPS εδειχνε ηδη 2.000 μετρα υψομετρο και τα συννεφα κυλουσαν συμπαγη πανω στην ασφαλτο σαν να ηταν κατι ζωντανο...

\photo{52.jpg}

... ενω εκαναν ολο το δρομο να μοιαζει σαν να εχει παρει φωτια...!

\photo{53.jpg}

Ωσπου τελικα.... Ο προορισμος:

\photo{54.jpg}

Εχοντας δει πολλες φορες ταξιδιωτικα με τα γνωστα parking και τα μεγαλα τουριστικα κιοσκια στα πασα των Αλπεων, θεωρουσα οτι ετσι θα ηταν και εδω. Οχι ακριβως. Στο υψηλοτερο σημειο του πασου που βρισκομουν αυτη τη στιγμη δεν υπηρχε τιποτα παρα μονο ενα παλιο και ερειπωμενο πλεον πανδοχειο, πραγματικα στη μεση του πουθενα... 

\photo{55.jpg}

Ποσες φορες στη ζωη μας μπορουμε να βαζουμε στοχους και να ειμαστε στη θεση να τους φτανουμε, να τους αγγιζουμε με τα ιδια μας τα χερια; Αυτο το ταξιδι ηταν ακριβως αυτο: στοχοι, σημεια στο χαρτη ενος μυαλου, που ηθελε να τα δει απο κοντα, να τα αγγιξει, να νοιωσει τη χαρα της κατακτησης, οτι και αυτος ηταν εκει! 
Το αν αξιζε η αναμονη και ο κοπος θα φαινοταν ``στο χειροκροτημα'' ομως αν νομιζατε οτι ειχαμε τελειωσει ετσι απλα, απατασθε...

\photo{56.jpg}

Το κρυο σε τετοιο υψος δεν αστειευοταν παρα τον ηλιο που εκανε φιλοτιμες προσπαθειες να με ζεστανει. Αν κατακαλοκαιρο με ηλιο ειχε τετοια ψυχρα, το χειμωνα ηθελα καν να σκεφτω πως θα ηταν εδω.

Καπου εκει ακουσα ηχο μηχανων και ειδα μια μεγαλη ομαδα μηχανοβιων να περνανε χαιρετωντας και αποφασισα να τους ακολουθησω κατεβαινοντας πλεον απο την αλλη πλευρα του βουνου. Δεν ειχα προλαβει να κανω πανω απο 3 χιλιομετρα οταν ειδα το λογο που οι περισσοτερες μηχανες δεν σταματουσαν στην κορυφη οπως εγω αλλα συνεχιζαν πιο κατω...

\photo{57.jpg}

Σταματησα στην ακρη της αλανας και καθισα εκει σωπηλος. Χαιδεψα το ντεποζιτο της Αυρας και θαρρεις την ακουσα να μου ψυθιριζει ``Σε τετοια μερη θελω παντα να με φερνεις, να γεμιζω απο τις ομορφιες του κοσμου...''

\photo{58.jpg}

Δεν ηξερα τι να πω. Η θεα ηταν τοσο εκθαμβωτικη που χαμογελουσα σαν χαζος μπροστα σε αυτο που δεν χωρουσε μεσα στα ματια μου...

\photo{59.jpg}

Στο mp3 αρχισε να παιζει το Freebird των Lynyrd Skynyrd και αρχισα να σιγοτραγουδαω...

\begin{verse}
\begin{center}
\textit{If I leave here tomorrow\\
Would you still remember me?}

\textit{For I must be traveling on, now,\\
Cause there's too many places I've got to see.}

\textit{But, if I stayed here with you, girl,\\
Things just couldn't be the same.}

\textit{Cause I'm as free as a bird now,\\
And this bird you can not change...}
\end{center}
\end{verse}


\photo{60.jpg}

Καθομουν εκει πολυ ωρα. Καποια στιγμη το μηνυμα στο κινητο με ξυπνησε απο το ονειρο. Ηταν η Christie που ρωτουσε που ημουν. Η ωρα πλησιαζε 4 το απογευμα και ειχα αλλα 250 χιλιομετρα μεχρι το Guebwiller -oσο και να μην ηθελα επρεπε να πηγαινω.
Καβαλησα τη μηχανη και ξεκινησα να κατεβαινω το πασο ομως ειχα πολλα ακομα να δω...

\photo{61.jpg}

Εξαλλου ο δρομος που τοση ωρα χαζευα απο ψηλα περνουσε ετσι και αλλιως απο το αλλο διασημο πασο της Ελβετιας, το Grimselpass.

\photo{62.jpg}

Οι ρυθμοι ραθυμοι πλεον -ηθελα να απολαυσω καθε στιγμη που βρισκομουν σε αυτα τα μοναδικα μερη και το ιδιο εκαναν και οι συνταξιδιωτες μου...

\photo{63.jpg}

Οσο ανεβαιναμε ομως ο καιρος αρχιζε πλεον να δειχνει τα δοντια του: ο ηλιος χαθηκε μεσα στα συννεφα και το κρυο τωρα ηταν πολυ τσουχτερο... 
Οντας μεσα σε αρκετη ομιχλη και με το ψιλοβροχο να πεφτει δεν ειχα πολλα να δω: ο δρομος συντομα με εβγαλε στη κορυφη οπου περιμενε το κλασσικο τουριστικο περιπτερο, με καποιους παραξενους εκπροσωπους της τοπικης πανιδας...

\photo{64.jpg}

...αλλα και μερικους ακομα πιο περιεργους συνταξιδιωτες!
Grimselpass, 2012

\photo{65.jpg}

Εκανα σταση να ξεμουδιασω και περπατησα τριγυρω. Το μερος ειχε πολυ πλακα. Οι παραξενες σιδερενιες κατασκευες ηταν διασπαρτες τριγυρω. Ποιος τις ειχε φτιαξει αραγε; Παντως οποιος και να ηταν ειχε πολυ ταλεντο και μερακι...

Οπως περπατουσα χαζευοντας το επιβλητικο ορεινο τοπιο, μεσα στην πυκνη ομιχλη ξαφνικα ειδα.... νερο;! Απιστευτο! Μια μεγαλη λιμνη στα 2.100 μετρα! 

\photo{66.jpg}

Και αυτο δεν ηταν τιποτα...
Η πορεια συνεχιστηκε προς το βορρα μεσα στη καταχνια, ωσπου μεσα απο την ομιχλη εμφανιστηκε μια εικονα που δεν θα ξεχασω ποτε: το φως λες και επαιζε με τα συννεφα και επεφτε πανω μια μεγαλη λιμνη που απλωνοταν τωρα μπροστα μου γεμιζοντας την με χρωματα! 
Η ``μοναδικη στιγμη'' -εκεινο το σημειο στο χρονο που ολα συμπιπτουν για να δημιουργησουν κατι πανεμορφο- και εγω ημουν τοσο τυχερος να ειμαι εδω τωρα!

\photo{67.jpg}

Ο δρομος μπροστα ανοιγοταν ατελειωτος ξανα και εγω παρακαλουσα αυτο το ταξιδι να μην τελειωσει ποτε...

\photo{68.jpg}

Δρομοι και εικονες βγαλμενες θαρρεις απο ονειρα... Η Ελβετικη φυση ηταν πραγματικα συγκλονιστικη!

\photo{69.jpg}

Ο δρομος απλωνοταν μπροστα μου ανοιχτος μεχρι εκει που εβλεπε το ματι και τα τοπια ηταν σαν πινακας ζωγραφικης...

\photo{70.jpg}
\photo{71.jpg}
\photo{72.jpg}

Στο παρελθον ειχα κανει πολλες φορες χιλιομετρα σε πανεμορφους επαρχιακους δρομους. Ομως αυτη τη φορα πραγματικα δεν ηξερα που να πρωτοκοιταξω απο την ομορφια -η διαδρομη απο το Grimsel μεχρι το Lungern ηταν ανετα μια απο τις πιο ομορφες που ειχα δει ποτε! Και επειδη καποιες φορες ακομα και οι εικονες δεν μπορουν να δειξουν τη μαγεια... απολαυστε! \href{http://www.youtube.com/watch?v=UuPvab4E5ZI}{Swiss Alps}

Συντομα ειχα βγει στο τελευταιο κομματι της απιθανης αυτης διαδρομης πριν την εθνικη οδο.
Ο δρομος περνουσε μεσα απο ενα πυκνο δασος, φτιαχνοντας ενα ατελειωτο καταπρασινο τουνελ που με προκαλουσε για ατελειωτο παιχνιδι... 

\photo{73.jpg}

Και καπου εδω ηταν που κρυμμενη αναμεσα στις φυλλωσιες των δεντρων βρισκοταν ισως η πιο ομορφη εικονα της ημερας...
.
.
.
.
.
.
.
Τα λογια ειναι απλα περιττα νομιζω...

\photo{74.jpg}

Δεν θα μπορουσα να ειχα ζητησει καλυτερο κλεισιμο για το κομματι των ορεινων και επαρχιακων δρομων της Ελβετιας.
Το ρολοι της μηχανης εδειχνε ηδη περασμενες 5 και ηταν ωρα να πιασω την εθνικη οδο για να γραψω τα τελευταια χιλιομετρα μεχρι το προορισμο μου. Οπως ολα τα αλλα σε αυτη τη χωρα, η εθνικη οδος τους ηταν απολυτως τελεια σε ...εκνευριστικο βαθμο!

\photo{75.jpg}

Συντομα ειχα βρεθει για μια ακομα φορα σε μια ακομη συνοριακη γραμμη: Vive la France!

\photo{76.jpg}

Μολις 40 χιλιομετρα εμεναν για το προορισμο μου και ενω σκεφτομουν το ποσο καλο καιρο ειχε κανει σημερα ο Μερφυ εκανε το θαυμα του: ξαφνικα και απο το πουθενα επιασε μια απιστευτα δυνατη μπορα ενω απο πανω ειχε ηλιο! Δεν ειχε ουτε ενα συννεφο απο πανω αλλα εγω ειχα γινει μουσκεμα. Μα απο ΠΟΥ με βρεχει ρε παιδια;
Συνεχισα μεσα στη βροχη απτοητος -ετσι και αλλιως ειχα σχεδον φτασει και η μπορα δεν θα κρατουσε για πολυ.
Οντως δεκα λεπτα αργοτερα η βροχη σταματησε και εγω εμπαινα στη μικρη κωμοπολη του Guebwiller.

\photo{77.jpg}
\photo{78.jpg}

Ολα πεντακαθαρα, μικρα δρομακια, χρωμα και πανεμορφες λεπτομερειες στα σπιτια παντου. 
Πολυ ομορφο μερος!

\photo{79.jpg}

Καπου εδω ηταν που με περιμενε η Christie -μια θεοπαλαβη, πολυ καλοστεκουμενη και χαμογελαστη 55αρα, με μια απιστευτη θετικη διαθεση και ορεξη για ζωη! Εμενε σε ενα πανεμορφο παλιο σπιτι Αλσατικου τυπου μαζι με το συντροφο της Francois και τη φοβερη γατα τους τη Lolotte.
Τακτοποιησα τη μηχανη και ανεβηκαμε πανω. Ηταν τοσο ομορφο συναισθημα να καταληγεις σε ενα ζεστο σπιτικο μετα απο μια ολοκληρη μερα στο δρομο!

Στη τραπεζαρια το τραπεζι ηταν στρωμενο και το φαγητο περιμενε. Οι ανθρωποι ειχαν ετοιμασει δειπνο για λογαριασμο μου! Ενοιωσα μεγαλη τιμη και μονο στη σκεψη οτι καποιος εμπαινε σε τετοιο κοπο για εμενα...
Η βραδια κυλισε υπεροχα με καλη παρεα, κουβεντουλα, εξαιρετικο γαλλικο κρασι και πολλα χαμογελα, ειδικα οταν εγω και Francois πιαναμε συζητηση: μπορει τα αγγλικα του να ηταν οσο καλα οσο και τα Γαλλικα μου (κοινως χαλια) αλλα αυτο δεν μας εμποδιζε να καταλαβαινομαστε μια χαρα! 
Οι πρωτες πρωινες ωρες μας βρηκαν σε φιλοσοφικες συζητησεις παρεα με μερικα ποτηρακια του εθνικου ποτου των Γαλλων: Pastis (κατι σαν το δικο μας ουζο).

Ο πιο ομορφος τροπος να κλεισεις μια τοσο τελεια ημερα...
